To construct our dataset we have combined PTA data from the ``Design of
International Trade Agreements'' (DESTA) (Dür, Andreas, Leonardo Baccini
and Manfred Elsig 2014) and from the CEPII ``Trade and Production
Database'' (TradeProd) (Thierry Mayer, Gianluca Santoni, Vincent Vicard
2023). The DESTA database aims to aggregate all agreements that have the
potential to liberalise trade, including all agreements notified to the
World Trade Organisation (WTO) and other agreements from a wide range of
sources, covering 880 agreements for 204 countries since 1948 to 2023 in
the last updated version.

Our sample consists of PTAs signed between the years 2000 to 2010 and
the country members to these PTAs, totalling 154 agreements and 143
member countries. For ease of estimation, and to get a sense of
geographical differences, we estimate our models by PTA region for five
main regions: Africa, Americas, Asia, Europe and Intercontinental (We
exclude Oceania {[}11 countries and 1 agreement{]} for lack of
sufficient trade data for our estimations). Each region has the
following samples of agreements and countries: Intercontinental (114
countries and 64 agreements), Europe (42 countries and 41 agreements),
Asia (35 countries and 33 agreements), Americas (15 countries and 13
agreements) and Africa (10 countries and 2 agreements).

For all countries in our sample, we get international trade and domestic
trade flows from the TradeProd database, which has been created
specifically for estimating gravity models and combines trade data from
the UN Commodity Trade Statistics Database (COMTRADE) and production
data from UNIDO Industrial Statistics database (INDSTAT). We also
download export data directly from COMTRADE for all countries in our
sample to construct our export product unit value measurements. For
estimations on trade flows, we use international trade flow data as
reported by importer. In order to measure the appropriate lags for the
effects of each agreement, our period of interest for international flow
data is between 1995 to 2015, and since we are estimating in 5-year
intervals, we get trade flow data for the years 1995, 2000, 2005, 2010
and 2015. Finally, as mentioned before, export product unit values are
constructed using the total value exported per product per year divided
by the net weight exported of said product for said year at the HS
2-digit code level for the 84 and 85 codes for manufacturing products.
As it is not possible to get data for product unit values for domestic
trade, the estimations using this measure as the dependent variable will
suffer from bias as the estimation does not include intra-trade effects.
However, the direction of bias is important as not including intra-trade
measures is expected to bias the effects of PTAs downwards (Yotov et al.
2016), so we use this estimates as illustrative conservative
measurements of the effects of PTAs on the unit value of exported
products.