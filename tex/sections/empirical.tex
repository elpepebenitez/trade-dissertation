The preference of a type of partner in a TAs then becomes an empirical
question. Do South-South TAs promote trade and industrial development
among their members? The empirical literature overall reports positive
effects of TAs on the trade of member countries, but with considerable
heterogeneity on the estimation coefficients. For example, a
meta-analysis of research papers on the effects of TAs on member trade,
encompassing 85 papers and 1827 estimates, finds an average of 0.59 (an
80\% increase in trade), with a median of 0.38 (a 46\% increase in
trade), a wide range of coefficient estimates (-9.01 to 15.41), and only
312 out of 1827 estimates reported as negative (\cite{cipollina_reciprocal_2010} Cipollina and Salvatici,
2010). Furthermore, a survey of the empirical research on the effect of
economic integration agreements on international trade flows, as well as
using the most modern econometric techniques to address biases, found an
increase of 50\% on international trade, but with significant variation
in the effects of specific agreements (\cite{kohl_we_2014} Kohl, 2014). However, much of the
empirical research is focused on the effects of TAs on or including the
most advanced economies. Empirical research focused exclusively on the
effects of South-South TAs or comparing them to the effects of
North-North or North-South TAs, is much less prevalent in the
literature. Although, several research papers do control for the type of
agreement (North-South or South-South) and have found positive and
significant effects of South-South TAs (\cite{medvedev_preferential_2006} Medvedev, 2006; \cite{mayda_south-south_2007} Mayda and
Steinberg, 2007; \cite{dahi_preferential_2013} Dahi and Demir, 2013; \cite{deme_trade-creation_2017} Deme and Ndrianasy, 2017), these
articles tend to be limited in their scope, sample size or only focus on
trade volumes.


Using firm-level data, empirical literature studying
trade outcomes using unit values of exports reports evidence of the
value per unit increasing as the income level of the importing nation
increases (\cite{hallak_product_2006} Hallak, 2006; \cite{bastos_quality_2010} Bastos and Silva, 2010). Relevant to our
analysis, one article finds evidence that the same firms export their
products at a higher value per unit the higher the income level of the
importing nation (\cite{manova_export_2012} Manova and Zhang, 2012). Beyond providing evidence
that the direction of trade has immediate repercussions, this could also
provide evidence in favour of North-South TAs, as they can generate more
revenue and promote quality upgrading (\cite{dahi_south-south_2017} Dahi and Demir, 2017). At the
same time, other strands of the empirical research literature emphasise
the importance of similarities in trade structure and preferences and
provide evidence that countries of similar levels of income, technology
and endowments have higher levels of trade, and importantly, more
potential for convergence and spillovers (\cite{hallak_product-quality_2010} Hallak, 2010). Important for
our discussion of the structure of product space, empirical research
finds that trade between similarly endowed countries have more
diversified exports between them, relative to trade with countries with
different endowments (\cite{regolo_export_2013} Regolo, 2013), and that countries with neighbours
with shared or similar comparative advantages will experience an
increase in the export of similar products to the neighbouring country
(\cite{bahar_neighbors_2014} Bahar, Hausmann and Hidalgo, 2014). If similarity between countries is
highly relevant for knowledge transfer, South-South TAs can potentially
be more beneficial for developing countries.