The Gravity Model of Trade

Often referred as the ``workhorse'' of international trade, the gravity
model is prominent in the empirical literature of applied international
trade analysis. Among the arguments that could explain/support the use
of the gravity model, there are four that are particularly relevant for
our purposes. First, the gravity model of trade is intuitive to
understand. Following the metaphor of Newton's Law of Universal
Gravitation, it predicts that international trade between two countries
is directly proportional to the product of their economic size, and
inversely proportional to trade frictions between them. In simpler
words, the bigger (smaller) the economies of two countries, and the
easier (harder) it is for them to trade with each other, the more (less)
we expect them to trade. Second, it is referred to as a structural model
with solid theoretical foundations, which makes it appropriate for
counterfactual analysis, such as measuring the effects of trade policies
as we aim to do with the effects of South-North versus South-South
agreements. Third, model has a flexible structure, which will allow us
to construct a specification tailored to our research. Finally, fourth,
it holds consistent and remarkable predictive power, both with aggregate
and sectoral data (Yotov et al. 2016).

Through the decades, the gravity equation has been regularly upgraded in
the theoretical and empirical literature. Of relevance, the simple
intuition of the gravity model was theoretically extended by Anderson to
note that, after controlling for size, the increase or decrease is
\emph{relative} to the average barriers of the two countries with all
their partners, which are referred as ``multilateral resistance''
(Anderson 1979). The more trade barriers or resistance to trade exists
with other countries relative to a given partner, the more a country is
pushed to trade with said partner. Anderson also introduced the
assumptions of product differentiation by place of origin, and Constant
Elasticity of Substitution (CES) expenditures ,or the Armington-CES
assumption (Yotov et al. 2016; Chatzilazarou and Dadakas 2023), which
led us to today's generalized form of the gravity equation, as developed
and popularised by Anderson and van Wincoop (Anderson and van Wincoop
2003).

Do I need to add more on the theory underlying the Gravity Models of
Trade?

Equally important, several empirical developments have strengthened the
gravity model and inform our choice of methodology: Exporter-time and
importer-time fixed effects are used to account for the multilateral
resistance terms in a gravity estimation with panel data (Olivero and
Yotov 2012); As the gravity model is often estimated with an OSL
estimator, zero-trade flows were dropped from the sample when trade was
transformed into a logarithmic form. Also, trade data is recognized to
suffer from heteroscedasticity (Yotov et al. 2016). To solve for
zero-trade flows and heteroscedasticity, the Poisson Pseudo Maximum
Likelihood (PPML) estimator has been proposed to estimate the gravity
model, avoiding potential biases (Silva and Tenreyro 2006; Santos Silva
and Tenreyro 2011); Country-pair fixed effects has been proposed to
account for the unobserved endogeneity of trade policy (Baier and
Bergstrand 2007). It is worth nothing that the inclusion of
exporter-time and importer-time fixed effects will absorb all observable
and unobservable time-varying country-specific characteristics that
could affect the dependent variable, while the country-pair fixed
effects will absorb observable and unobservable bilateral time-invariant
characteristics that could affect trade costs; The inclusion of
intra-trade flows as well as international trade flows is proposed to
correctly estimate the effects of non-discriminatory trade policy,
allowing for consumers to choose products from both international and
domestic sources (Dai, Yotov, and Zylkin 2014; Heid, Larch, and Yotov
2017); Year-intervals instead of data pooled over consecutive years
should be used to allow for adjustment of trade flows to policies that
might not have immediate effects (Baier and Bergstrand 2007; Anderson
and Yotov 2016); And finally, to account for the effects of
globalization forces that may biased the estimates of trade policies, a
set of globalization dummies are recommended to control for the effects
of globalization in the gravity model (Yotov 2012; Bergstrand, Larch,
and Yotov 2015). Based on the theoretical and empirical best-practices
found in the relevant literature, we employ the following gravity
equation using a PPML estimator and a balanced panel data approach with
multiple exporters, multiple importers and time as our benchmark model:

\[(1)\ \ \ \ \ X_{ij,t} = \ exp(\eta_{i,t} + \psi_{j,t} + \gamma_{\binom{-}{ij}} + \beta_{1}{PTA}_{ij,t} + \beta_{2}{PTA}_{ij,t - 5} + \sum_{t}^{}b_{t}) + \epsilon_{ij,t}\]

Where \(X_{ij,t}\) denotes the value of exports from an origin country
\(i\) to a destination country \(j\); \(\eta_{i,t}\) and \(\psi_{j,t}\)
are, respectively, exporter-time and importer-time fixed-effects;
\(\gamma_{\binom{-}{ij}}\) is a country-pair fixed-effect;
\({PTA}_{ij,t}\) and \({PTA}_{ij,t - 5}\) are our main variables of
interest, which, respectively indicate if \(i\) and \(j\) are members of
a PTA at time \(t\) and, to account for potential ``phase-in'' effects
over time of the PTA, at time \(t - 5\); \(\sum_{t}^{}b_{t}\) is a set
of dummies that equal 1 for international trade and 0 for domestic trade
observations at each time \(t\); and \(\epsilon_{ij,t}\) is an error
term.

In contrast with our main interest of research, which are the potential
heterogenous effects of PTAs on different members for different types of
agreements, this benchmark model, specifically
\(\beta = \beta_{1} + \beta_{2}\) , would provide the average ``total''
partial effect of PTAs on trade after accounting for lagged effects, but
it cannot provide the effects for a given agreement, country-pairs o
specific country members to a specific agreement. As such, three
successive expansions can be implemented to capture heterogeneity in PTA
effects as proposed by Baier \emph{et al}. (Baier, Yotov, and Zylkin
2019):

\[(2)\ \ \ \ \ X_{ij,t} = \ exp(\eta_{i,t} + \psi_{j,t} + \gamma_{\binom{-}{ij}} + \sum_{A}^{}{\beta_{1,A}{PTA}_{ij,t}} + \sum_{A}^{}{\beta_{2,A}{PTA}_{ij,t - 5}} + \sum_{t}^{}b_{t}) + \epsilon_{ij,t}\]

Equation (2) can be implemented to account for heterogeneous effects of
PTAs at the level of the specific agreement, by allowing for distinct
average partial effects for each individual agreement, using superscript
\(A\) to index by agreement and also allowing for agreement-specific
lags: \(\beta_{A} = \beta_{1,A} + \beta_{2,A}\).

Extended Benchmark Model with PTA Types

You want to modify the benchmark model to allow for different types of
PTAs (NN, NS, SS) while still keeping the same overall structure. Here's
how you can adjust the model:

Equation (1) with PTA Types:

\[(3)\ \ \ \ \ X_{ij,t} = \ exp(\eta_{i,t} + \psi_{j,t} + \gamma_{\binom{-}{ij}} + \beta_{1NN}{PTA\_ NN}_{ij,t} + \beta_{2NN}{PTA\_ NN}_{ij,t - 5} + + \beta_{1NS}{PTA\_ NS}_{ij,t} + \beta_{2NS}{PTA\_ NS}_{ij,t - 5} + \beta_{1SS}{PTA\_ SS}_{ij,t} + \beta_{2SS}{PTA\_ SS}_{ij,t - 5} + \sum_{t}^{}b_{t}) + \epsilon_{ij,t}\]

Where:

\begin{itemize}
\item
  \(X_{ij,t}\)\hspace{0pt} denotes the value of exports from country
  \(i\) to country \(j\) at time \(t\).
\item
  \(\eta_{i,t}\) and \(\psi_{j,t}\ \)are exporter-time and importer-time
  fixed effects, respectively.
\item
  \(\gamma_{\binom{-}{ij}}\) is a country-pair fixed effect.
\item
  \hspace{0pt}\(\beta_{1NN}\) and \(\beta_{2NN}\) are the coefficients
  for the immediate and lagged effects of a North-North PTA
  (\(PTA\_ NN\)).
\item
  \hspace{0pt}\hspace{0pt}\(\beta_{1NS}\) and \(\beta_{2NS}\) are the
  coefficients for the immediate and lagged effects of a North-South PTA
  (\(PTA\_ SN\)).
\item
  \hspace{0pt}\hspace{0pt}\(\beta_{1SS}\) and \(\beta_{2SS}\) are the
  coefficients for the immediate and lagged effects of a South-South PTA
  (\(PTA\_ SS\)).
\item
  \(\sum_{t}^{}b_{t}\) is a set of time dummies accounting for
  international trade-specific effects at each time \(t\).
\item
  \(\epsilon_{ij,t}\) is the error term.
\end{itemize}

Extended Model with PTA Heterogeneity and Types

Next, you want to extend the model allowing for PTA heterogeneity to
also distinguish between different types of PTAs.

Equation (2) with PTA Types:

\[(4)\ \ \ \ \ X_{ij,t} = \ exp(\eta_{i,t} + \psi_{j,t} + \gamma_{\binom{-}{ij}} + \sum_{A}^{}{(\beta_{1,A,NN}{PTA\_ NN}_{ij,t}\  + \ \beta_{2,A,NN}{PTA\_ NN}_{ij,t - 5})} + \sum_{A}^{}{(\beta_{1,A,NS}{PTA\_ NS}_{ij,t}\  + \ \beta_{2,A,NS}{PTA\_ NS}_{ij,t - 5})} + \sum_{A}^{}{(\beta_{1,A,SS}{PTA\_ SS}_{ij,t}\  + \ \beta_{2,A,SS}{PTA\_ SS}_{ij,t - 5})} + \sum_{t}^{}b_{t}) + \epsilon_{ij,t}\]

Where:

\begin{itemize}
\item
  \(X_{ij,t}\)\hspace{0pt} denotes the value of exports from country
  \(i\) to country \(j\) at time \(t\).
\item
  \(\eta_{i,t}\) and \(\psi_{j,t}\ \)are exporter-time and importer-time
  fixed effects, respectively.
\item
  \(\gamma_{\binom{-}{ij}}\) is a country-pair fixed effect.
\item
  The summations \hspace{0pt}\(\sum_{}^{}A\) denote the sum over
  different agreements \(A\) for:

  \begin{itemize}
  \item
    \(\beta_{1,A,NN}\) and \(\beta_{2,A,NN}\): Coefficients for the
    immediate and lagged effects of North-North PTAs
    \hspace{0pt}(\(PTA\_ NN\)).
  \item
    \(\beta_{1,A,NS}\) and \(\beta_{2,A,NS}\): Coefficients for the
    immediate and lagged effects of North-South PTAs (\(PTA\_ SN\)).
  \item
    \(\beta_{1,A,SS}\) and \(\beta_{2,A,SS}\): Coefficients for the
    immediate and lagged effects of South-South PTAs (\(PTA\_ SS\)).
  \end{itemize}
\item
  \(\sum_{t}^{}b_{t}\) is a set of time dummies accounting for
  trade-specific effects at each time \(t\).
\item
  \(\epsilon_{ij,t}\) is the error term.
\end{itemize}

Variables:

\begin{enumerate}
\def\labelenumi{\arabic{enumi}.}
\item
  \({PTA\_ NN}_{ij,t}\): Dummy variable that takes the value of 1 if the
  trade pair \((i,j)\) is part of a North-North PTA at time \(t\), and 0
  otherwise.
\item
  \({PTA\_ NN}_{ij,t - 5}\): Dummy variable that takes the value of 1 if
  the trade pair \((i,j)\) was part of a North-North PTA at time
  \(t\)\emph{-5}, and 0 otherwise.
\item
  \({PTA\_ NS}_{ij,t}\): Dummy variable that takes the value of 1 if the
  trade pair \((i,j)\) is part of a North-South PTA at time \(t\), and 0
  otherwise.
\item
  \({PTA\_ NS}_{ij,t - 5}\): Dummy variable that takes the value of 1 if
  the trade pair \((i,j)\) was part of a North-South PTA at time
  \(t\)\emph{-5}, and 0 otherwise.
\item
  \({PTA\_ SS}_{ij,t}\): Dummy variable that takes the value of 1 if the
  trade pair \((i,j)\) is part of a South-South PTA at time \(t\), and 0
  otherwise.
\item
  \({PTA\_ SS}_{ij,t - 5}\):: Dummy variable that takes the value of 1
  if the trade pair \((i,j)\) was part of a South-South PTA at time
  \(t\)\emph{-5}, and 0 otherwise.
\end{enumerate}

These models allow you to capture the differentiated impacts of PTAs
depending on whether they are between two developed countries (NN),
between a developed and a developing country (NS), or between two
developing countries (SS).

References
