Inspired by other strands of the international trade literature, we also
test our models using ``Unit Values'' of the products exported, by
dividing the total value exported by the total weight exported in
kilograms (Latzer and Mayneris 2021; Manova and Zhang 2012; Bastos and
Silva 2010). Using the unit value as the dependent variable in our
estimations allow us to analyse if the value per unit exported is
affected by TAs. To be consistent in our effort to understand the
potentially heterogenous effects of TAs according to the different
category of the members in trade volume, but also in quality upgrading
and industrialization development of countries, we focus on
manufacturing products (Chatzilazarou and Dadakas 2023) with HS 2-digit
codes 84 (Nuclear reactors, boilers, machinery and mechanical
appliances; parts thereof ) and 85 (Electrical machinery and equipment
and parts thereof; sound recorders and reproducers, television image and
sound recorders and reproducers, and parts and accessories of such
articles) which are part of the ``Machinery and mechanical appliances;
electrical equipment; parts thereof; sound recorders and reproducers,
television image and sour sound recorders and reproducers, and parts and
accessories of such articles'' category from the World Customs
Organization. Our aim is to compare the effects of TAs on trade volumes
against the effects on the unit value of manufacturing products
exported.
