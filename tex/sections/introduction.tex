Background and context

Research problem

Significance of study

Research questions

no conclusive answer to the research questions of whether South-South
TAs act as building blocks or stumbling blocks to developing countries,
or if they are preferable to North-South agreements.

Set up the relevance of the topic.

Proliferation of TAs

In the recent decades, alongside a liberalization of foreign trade,
developing countries of all sizes and regions have actively made efforts
to enter into Trade Agreements (TAs), with both other developing
countries as well as already developed nations.

What are we studying (Research question(s)), on what countries, and why.

Heterogeneity within agreements and not only across. Different effects
of an agreement on its members.

What methodology is used in the paper and why.

How does this paper relate to the wider literature. What are we
contributing.

Gravity literature. Export complexity.

Stylized facts on the historical evolution of the adoption of TAs and
the diversity of products exported.

Ideas:

Total number of SS, NS and NN TAs

Share of SS, NS and NN TAs

Total exports by S and N countries

Share of total exports by S and N countries

Total exports of manufactured by S and N countries

Share of total exports of manufactured by S and N countries

Number of products exported by S and N countries

Section. The paper proceeds in section 2 with a review of the relevant
theoretical and empirical literature on the effects of TAs on
North-South and South-South trade and potential development
implications. Section 3 introduces our empirical methodology and data.
Section 4 presents and describes our main findings, and Section 5
analyses and discusses their potential implications and how they fit
within the relevant literature. Finally, Section 6 concludes.
