Some potential determinant mechanisms of the effects of TAs in the
academic literature are related to the content of the TA, and the extent
to which it removes trade barriers. TAs should have more potential for
larger effects when they remove trade frictions imposed by other trade
policies and regulations, domestic or foreign (\cite{baier_widely_2019} Baier et al., 2019).
Moreover, unilateral trade policies can create a terms-of-trade
inefficiency externality when a government introduces a higher trade
barrier, shifting the cost to foreign exporters (\cite{bagwell_economic_1999} Bagwell \& Staiger,
1999). Since foreign exporters bear the cost of the inefficiency, there
is a tendency by governments to set barriers at a higher level than it
would be politically efficient. TAs can act as a mechanism to remove or
lower said inefficiencies, resulting in better trade and welfare
outcomes, or in an observed higher effect of a TA relative to
non-members in our case. Through trade diversion, there is a theoretical
possibility that the proliferation of TAs can harm the terms of trade of
non-TA-members and create significant inefficiencies in the world
trading system (\cite{anderson_terms_2016} Anderson \& Yotov, 2016), but empirical evidence so far
finds that TAs negligibly harm non-members and global efficiency rises.

Another strand of the relevant literature emphasises the extensive
margin of trade ex-ante the signature of a TA as an important
determinant of its effects. In particular, that TAs have an important
effect in the growth of the extensive margin of trade, which in turn is
a significant factor in the overall growth of total trade (\cite{kehoe_how_2013} Kehoe \&
Ruhl, 2013). If a TA is signed between country members with low
diversity of traded goods, it is expected that we will see a bigger
effects of the TA in trade growth driven by the increase in the number
of goods traded and in the volume of trade of the least-traded products
(\cite{kehoe_using_2015} Kehoe et al., 2015). Interestingly, empirical research shows that the
number of products exported ex-ante is positively related to the trade
creation after a TA, but when heterogeneous effects of TAs within
agreements and country-pairs is taken into consideration, the extensive
margin of trade does account for differences in trade creation (\cite{baier_widely_2019} Baier et
al., 2019).

There is also evidence that different types of agreements, such as
non-reciprocal preferential trade agreements (NRPTAs), preferential
trade agreements (PTAs), free trade agreements (FTA), customs unions
(CU), common markets (CMs) and economic unions (EUs), can have different
levels and time horizons of trade effects (\cite{baier_economic_2014} Baier et al., 2014; \cite{magee_new_2008-1} Magee,
2008). This can occur because different types of TAs can induce
different unobservable effects that reduce trade costs, as we observe
that modern TAs not only reduce tariffs, but also regulate all kinds of
non-tariff issues in what is called ``deep integration'' (\cite{anderson_terms_2016} Anderson \&
Yotov, 2016). The deeper the integration, the more effective we expect
TAs to be (\cite{kohl_we_2014} Kohl, 2014). It has also been shown that the design of TAs
matters, in terms of institutional design and legal enforceability, with
more comprehensive agreements being better at stimulating positive trade
outcomes (\cite{kohl_trade_2013} Kohl et al., 2013).

The differences in market power across member countries could also be
important, as countries with less market power relative to other TA
members over their terms of trade are expected to grant smaller
concessions when they negotiate agreements. Agreements between countries
with relatively similar market power over each other's term of trade
potentially have higher potential to eliminate inefficiencies and
achieve higher effects (\cite{baier_widely_2019} Baier et al., 2019).

Based on the academic arguments mentioned, can we expect that TAs will
have more potential for effectively improve trade outcomes when the
bilateral relationship is South-South vs North-South? It would appear
that it depends highly on the terms of trade inefficiencies and the
potential for increases in the extensive margin of trade ex-ante the
agreement is in place, as well as in the design and depth of the
agreement. These are considerations that should be taken on a bilateral
basis, rather than in an aggregated matter. Moreover, as more
South-South TAs are signed, and more of the share of global trade
happens among South countries, the North-South distinction also starts
to lose relevance. Evidence appears to show that the ``South'' is
splitting into groups, with the ``Emerging South'' growing at an
accelerated pace and even challenging the hegemony that developed
economies that enjoyed since the Post-World War II period. It could be
the case that the same power dynamics observed between developed and
developing countries by the classical development literature also occur
in South-South relationships, and that they become a threat for the
development of the least-developed economies in the South (\cite{dahi_south-south_2017} Dahi \&
Demir, 2017). It is clear that more research focused on South-South
dynamics is needed in order to guide the policy decisions of different
groups of countries.

It appears clear from the literature studied and from the empirical
analysis carried out in this paper, that TAs have significant potential
and that they can be an effective development policy tool for South
countries based on their dynamic effects on the structure of production
capacity, as long as a proper analysis of current capabilities and
identification of related products and industries is carried out. South
countries should strive to acquire new capabilities close-by in
relatedness to the capabilities already in place and choose appropriate
partner countries to do so. For more immediate concerns of trade
creation and trade diversion, it should be taken into consideration low
traded and non-traded products between potential partners to increase
the chances of trade creation, as well as striving for deep integration
in the design of the agreement as much as possible.