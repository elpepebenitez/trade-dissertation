This paper empirically analysed the effects of TAs on the volume of
trade of exports and on the value per unit of manufacturing products
exported of member-countries to agreements signed between the years 2000
and 2015, with an ample dataset comprised of 154 agreements and 143
countries, using a gravity model of trade, updated with the best
practices in the literature, and subsequent extensions to capture the
heterogeneous effects of TAs on its members, and on their disaggregated
bilateral trade relationships classified as North-North, North-South and
South-South, relative to non-TA-members. We found coefficient magnitudes
consistent with the empirical literature and high degrees of
heterogeneity on the effects of TAs, and no conclusive answer to the
research questions of whether South-South TAs act as building blocks or
stumbling blocks to developing countries, or if they are preferable to
North-South agreements. We proposed some potential mechanisms driving
the heterogeneity of the effects of TAs, and also cautioned against
threating the ``South'' as a homogeneous group.

In this paper we have proposed several methodological innovations to
advance the literature on the effects of TAs. We use a modern data set,
comprised of data between the years 1995 and 2015, with data on both
international and domestic trade. We do not focus our sample on
particular regions or groups of countries, nor on specific agreements.
We try to cover as many countries and TAs as possible, without over
representation of developed or ``North'' countries or of the biggest
agreements. We extend traditional gravity estimations to capture
heterogenous effects of TAs instead of the average ``total'' partial
effect as is common in the literature, as well as heterogenous effects
of TAs on the different categories of bilateral trade relationships
(North-North, North-South and South-South). Finally, we complement our
main estimations by replacing bilateral trade volume with the export
product unit value of manufacturing products (HS codes 84 and 85).

Future research on the heterogenous effects of TAs using gravity models
is promising, as the empirical methods continue to improve, and they are
applied to get more detailed and nuanced estimates that can better guide
the developmental decisions and policies of developing countries. Some
potential areas for future research on ``South'' countries include
research on the dynamic effects of TAs on the industrialization process
and on technology absorption and upgrading; extending gravity models to
capture effects of country-pairs member to a TA, and to capture effects
on individual countries of a country-pair member to a TA (Baier et al.,
2019); extending the gravity model to capture effects of different types
of TAs depending on their depth and content; different
sub-classifications of ``South'' countries should be explored to further
understand the limits to South-South cooperation in trade; and, beyond
trade volume, the measure of export product unit value can be used to
capture the increase or decrease of the value per unit commodities and
goods in specific industries.