Despite their apparent importance for the government of developing
countries, a common view in the academic literature is that South-South
TAs are not as effective as North-South TAs, and that they do not
achieve significant effects, making them largely symbolic (\cite{gamso_leveling-up_2022} Gamso \&
Postnikov, 2022). North-South agreements, signed between developed and
developing countries, are presented as a superior alternative, more
effectively leading to increased trade for its members, and quality
upgrading through learning-by-exporting dynamics and better access to
intermediate goods for developing countries. At the same time, other
strands of the international trade literature present South-South TAs as
a more effective platform for developing countries to grow, at least to
a level where they can take advantage of North-South cooperation without
being undermined by more influential powers. This debate is often
presented as a dichotomy, where South-South TAs are either building or
stumbling blocks for developing countries. Solving it, will go a long
way in better informing developing countries into which agreements they
should enter, with what types of partners, and how should the agreement
be designed.

In this paper, we venture to analyse this dichotomy empirically through
the use of a gravity model of trade, and subsequent extensions, in order
to get estimates for the effects of specific TAs, as well as estimates
on the effects of TAs on South-South, North-South and North-North
bilateral trade relationships, relative to non-TA-members. We also
explore the use of the export product unit value (EPUV) as an
alternative to trade volume, in order to capture the effects of TAs on
the value per unit exported of manufacturing goods. As such, we explore
two specific research questions: do South-South TAs act as building
blocks or stumbling blocks to developing countries? Are they preferable
to North-South agreements?

This research is related to literature from traditional trade theory on
comparative advantage, to new trade theory and classical development
theory on the potential dynamic and scale effects of TAs, and to more
recent literature on the relevance of the structure of the product space
exported and the extensive margin of trade.

The paper proceeds in section 2 with a review of the relevant
theoretical and empirical literature on the effects of TAs on
North-South and South-South trade and potential development
implications. Section 3 introduces our empirical methodology and data.
Section 4 presents and describes our main findings, and Section 5
analyses and discusses their potential implications and how they fit
within the relevant literature. Finally, Section 6 concludes.