In contrast, classical development theory and new trade literature go
beyond the static welfare gains from trade creation and diversion
effects when analysing the effect of TAs. Developing countries can use
TAs to overcome limitations of their domestic market size in the
industrialization process (Dahi and Demir, 2013). Such potential
increases in the effective market size could help industries in
developing countries achieve economies of scale and increase the skill
content of production and exports, which in turn could improve the
market penetration of exports of developing countries in developed
markets in industrial products (Fugazza and Robert-Nicoud, 2006). Also,
due to similarities in production patterns and resource base among
developing countries, incentivising trade by lowering barriers could
facilitate appropriate technology transfer, according to the needs of
developing countries (UNIDO, 2006). Of particular relevance for
developing countries, it is argued that the products that countries
export matter for long-term economic performance. If a country exports
products from industries that are more technology-intensive, these are
likely to create input-output linkages and spillover effects in human
and physical capital accumulation and innovation (Hausmann, Hwang and
Rodrik, 2007). Furthermore, by allowing for factor accumulation, TAs can
reduce intra-block trade barriers and increase competition and access to
cheaper intermediate goods, triggering changes in industrial production
in member countries. As such, TAs among ``South'' countries can reduce
intra-South barriers and lead to industrialization of the region (Puga
and Venables, 1998). In this context, what matters are not static gains
from TAs, but dynamic gains in industrial development. If South-South
TAs truly promote industrial development of member countries, they might
be desirable even if there are short-term losses due to trade diversion
(Dahi and Demir, 2013). Other arguments in the development literature
emphasize the asymmetries in bargaining power between ``North'' and
``South'' countries, which could lead to worse outcomes for developing
countries if their policy space gets restricted (Thrasher and Gallagher,
2008). To the extent that these arguments hold true, developing
countries could be better off entering into South-South rather than
North-South agreements, or at least should pursue both kinds of
agreements.