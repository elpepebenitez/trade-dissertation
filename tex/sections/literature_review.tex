This section reviews the literature on the theoretical and empirical
effects of PTAs on exports and situates the analysis in the relevant
field of research.

\textbf{Theoretical Framework}

\textbf{Comparative advantage and trade creation and diversion}

Traditional trade theory emphasizes trade creation (allowing cheaper
products from PTA members to substitute for more expensive domestic
products) and trade diversion (substituting products from non-PTA
members that were cheaper before the PTA with products from PTA members
that are cheaper now due to the PTA reducing tariffs) (Schiff, Winters
and Schiff, 2003) and argues that the impact of PTAs depends on the
comparative advantage of member countries. In particular, it argues that
PTAs magnify the impacts of a country's comparative advantage, relative
to the world and to other member countries signatories of a common PTA.
If member countries of a PTA have a comparative advantage on a factor
endowment relative to the world, but one country also has a comparative
advantage on the same factor endowment relative to the other member
countries, the country with the ``extreme'' advantage will be more
vulnerable to trade diversion effects, while countries with
``intermediate'' advantages will gain from trade creation effects,
predicting divergence of trade outcomes, and winners and losers among
member countries. (Venables, 2003). This emphasis on the trade creation
and trade diversion effects among member countries with significant
differences in the comparative advantage of their factor endowments
relative to the world and to each other, suggests that, when the country
with the ``extreme'' comparative advantage is a high-income country,
relative to a lower-income country with an ``intermediate'' comparative
advantage, the lower-income country should seek a PTA with the other
country as it will gain more. On the contrary, if both members are
lower-income countries, the country with the ``extreme'' comparative
advantage, should not seek a PTA with the other low-income member
country as it will be vulnerable. (Sanguinetti, Siedschlag and
Martincus, 2010). This logic can be easily extended to the North-South
and South-South types of PTAs, as North countries will reasonably have
an ``extreme'' comparative advantage in skill-intensive goods relative
to South countries, while South countries will reasonably have an
``extreme'' comparative advantage in labour-intensive goods relative to
North countries. Furthermore, it is also argued that benefiting from
economies of scale through South-South economic integration is more
difficult because member countries do not have complementary production
and trade structures nor high interpenetration of each other's markets
on intra-industry trade. (Schiff, Winters and Schiff, 2003). Also, South
countries can benefit from greater technological diffusion from
North-South PTAs as the North countries have higher industrial
development as well as investment in research (Schiff and Wang, 2008).
Finally, as the trend in manufacturing has been in favour of vertical
specialization or value chain fragmentation (Krugman, 1995), North-South
PTAs are preferable as developing countries strive to capture a greater
portion of the value added. For these arguments, developing countries
should therefore be better off entering into North-South rather than
South-South agreements.

\textbf{Economies of Scale and Input-Output linkages}

In contrast, classical development theory and new trade literature go
beyond the static welfare gains from trade creation and diversion
effects when analysing the effect of PTAs.

``Preferential trade arrangements between developing countries can lead
to industrialization of the region as a whole as a consequence of the
effective market enlargement induced by reducing intra-South barriers.''
(Puga and Venables, 1998)

Other arguments

\begin{itemize}
\item
  Infant industry development
\item
  Economies of scale
\item
  Decoupling
\end{itemize}

``Trade can also arise from product differentiation and from economies
of scale that reduce costs as production grows. In these circumstances
competition between firms is weakened, and consumers lose. International
trade then offers an important means of increasing competition by
allowing new suppliers to enter markets.'' (Schiff, Winters and Schiff,
2003)

Stumbling block vs building block dichotomy

\textbf{Discussions on SS PTAs}

``\emph{The trade literature long argued that PTAs can benefit member
states through economies of scale and comparative advantage and higher
competition (Schiff, 2003). However, these arguments are generally
reserved for North--North and South--North but not South--South PTAs.
First, it is argued that similar production and trade structures in the
South make it more difficult to benefit from economies of scale. Second,
given the lower industrial development and research and development
activities in the South, greater technology diffusion for the Southern
country can be reaped from South--North integration (Schiff and Wang,
2008).3 Third, the more advanced members are argued to be the likely
winners in South--South integration, thanks to their higher industrial
and institutional development. As a result, lower income Southern
countries might be better off entering South-- North PTAs. It is also
claimed that industries with long term development potential are more
likely to move to the bigger and richer members, leading to divergence
once the barriers are lowered under South--South PTAs (Puga and
Venables, 1997; Schiff, 2003; Venables, 2003). Last but not the least,
North--South PTAs are argued to facilitate increasing vertical
specialization or value chain fragmentation, what Krugman (1995)
referred to as the slicing up of the value added.4}

\emph{In contrast, the classical development theory and new trade
literature has a more positive view of South--South PTAs, focusing on
their developmental benefits through infant industry development,
economies of scale and decoupling rather than on the static welfare
gains (from trade creation and diversion) or the `stumbling block/
building block' dichotomy. Myrdal (1956), for example, suggested that
regional integration in the South can help developing countries overcome
local market size limita- tions during industrialization. Accordingly,
given the strongly skill biased structure of output expansion in inter-
national trade (Antweiler and Trefler, 2002), increasing market size can
help developing countries enjoy scale effects and increase the skill
content of their exports while reducing the cost of intermediaries,
which in return may help stimulate export penetration into Northern mar-
kets in industrial goods (Fugazza and Robert-Nicoud, 2006). Likewise,
Lewis (1980), and more recently UNCTAD (2005) and World Bank (2008),
also pointed out that South--South trade can reduce the growth depen-
dence of the South on Northern growth, leading perhaps to decoupling
from Northern business cycles (thus helping the recovery from current
global downturn (ESCAP, 2009)). Furthermore, the structure of
South--South trade is argued to have dynamic and long term benefits for
developing countries because of its comparatively higher technology and
human capital intensive factor content (Amsden, 1987; Lall et al., 1989;
Demir and Dahi, 2011). Besides, similarity in production pattern and
resource base may facilitate appropriate technology trans- fer (Amsden,
1980, 1987; UNIDO, 2005; World Bank, 2006).}

\emph{It is also possible that South--North PTAs can yield more benefits
to Northern countries than the Southern ones because of asymmetries in
bargaining power, nego- tiating capacity and retaliatory capability.
Even though these asymmetries are also present between Southern
countries, the gap is likely to be smaller.}

\emph{Thrasher and Gallagher (2008), for example, show that South--South
PTAs leave the greatest policy space available to `deploy effective
policy for long-run diversification and develop- ment' than South--North
PTAs. We should also note that structuralist North--South models have
long discussed how interactions between countries with asymmetrical
economic structures, patterns of specialization, and devel- opment can
lead to uneven development (Findlay, 1980; Darity, 1990; Dutt, 1992; and
also see the survey articles Findlay, 1984; Dutt, 1989; Darity and
Davis, 2005).}

\emph{In addition to the debate above, the effects of PTAs on the
structure of trade are of particular importance for long term
development and growth. Development economics and the new trade theory
provide strong evidence that not all trade is equal and what you export
might matter for long term economic performance (Kaldor, 1967; An and
Iyigun, 2004; Hausmann et al., 2007). Exports in more technology
intensive industries are likely to generate lar- ger spillovers (such as
innovation and physical and human capital accumulation) and linkages for
development than lower technology and labour intensive ones (Hausman et
al., 2007). Earlier on, this point was also raised by Kaldor (1967) in
his three growth laws; which stated that there is a strong positive
relationship between the growth of manufacturing output and (i) the
growth of GDP, (ii) the growth of labour productivity in manufacturing
(i.e. the Verdoorn's law) and (iii) the growth of productivity in
nonmanufacturing sectors.}

\emph{Note that the question we raise here is different than the one
usually discussed in the literature, which is whether PTAs are trade
creating or diverting.5 To the extent that PTAs enhance manufactures
exports, we can then start evaluating the success or failure of PTAs
according to their potential long term developmental impact. Much of the
traditional PTA literature, both theoretical and empiri- cal, is taken
up by trade creation versus trade diversion debate. These questions are
not unimportant; however, there is reason to question the
disproportionate attention still given to the classic Vinerian
dichotomy.}

\emph{(...)}

\emph{Second, since North--North, South--North and North--South trade
barriers appear to be significantly lower than the ones present in
South--South trade (Kowalski and Shepherd, 2006, also see Kee et al.,
2009; Medvedev, 2010), it is unlikely that South--South PTAs are trade
diverting from the North, which has retrospectively been the main point
of contention among trade theorists on the relative costs and benefits
of South--South PTAs. In fact, consistent with Mundell (1968)'s
assertion that `a member's gain from a free-trade area will be larger,
the higher are the initial tariffs of partner countries', South--South
trade barrier reduction is found to generate a significant increase in
South--South exports, while no such effect is reported in the case of
North--South, South--North or North--North trade (Kowalski and Shepherd,
2006). Besides, there is also some empirical evidence showing that
South--South PTAs are no more trade diverting than other PTAs (Cernat,
2001). Third, since higher transportation costs and former colonial
linkages with Northern countries (which always appear to be significant
in gravity models of trade), in addition to higher trade barriers (Kee
et al., 2009), con- tinue to limit South--South trade expansion, PTAs
might be seen as a way of compensating for such trade barriers that are
lower in South--North, North--South or North-- North trade.6 Last but
not least, in the case of industrial development, what matters are
dynamic not static gains. That is to say, if South--South PTAs are found
to enhance industrial development, the long term gains may very well
outweigh the static short term losses.}'' (Dahi and Demir, 2013)

\textbf{Significance of Exports}

\textbf{Defining South and North}

\textbf{Empirical literature}

``Turning to the empirical work on PTAs, the majority of research
reports a significantly positive effect of PTAs on member trade.
Cipollina and Salvatici (2010) review 85 papers on the effects of PTAs
and find that the mean effect is 0.59 (or an 80\% increase in trade),
while the median is 0.38 (or a 46\% increase in trade). Although the
range of coefficient estimates is quite large (--9.01 to 15.41), only
312 out of 1827 coefficient estimates are reported as negative.
Nevertheless, despite the diversity of research, there are only few
studies that compare heterogeneous effects of PTAs within and between
developing and developed countries. Among the few, Medvedev (2010),
using a cross sectional analysis, reports that while North--North PTAs
are insignificant in stimulating preferential trade, North--South PTAs
increase trade by 40\% and South-- South PTAs increase them by 163\%.

Cipollina, M. and Salvatici, L. (2010) Reciprocal trade agreements in
gravity models: a meta analysis, Review of International Economics, 18,
63-80.

Medvedev, Denis, Preferential Trade Agreements and Their Role in World
Trade (October 1, 2006). World Bank Policy Research Working Paper No.
4038, Available at SSRN: \url{https://ssrn.com/abstract=938031}''

The empirical work on the structure of trade under PTAs has also been
scarce. Sanguinetti et al. (2010) examine the impact of PTAs on
South--South manufacturing production patterns in the case of MERCOSUR
for the period of 1985--1998 and find that South--South PTAs cause a
spatial regional reorganization of production along the lines of
internal comparative advantage.

Sanguinetti, P., Siedschlag, I., \& Martincus, C. V. (2010). The Impact
of South-South Preferential Trade Agreements on Industrial Development:
An Empirical Test. Journal of Economic Integration, 25(1), 69--103.
http://www.jstor.org/stable/23000966

\textbf{Stylised facts}

\textbf{Ideas:}

Total number of SS, NS and NN PTAs

Share of SS, NS and NN PTAs

Total exports by S and N countries

Share of total exports by S and N countries

Total exports of manufactured by S and N countries

Share of total exports of manufactured by S and N countries

Number of products exported by S and N countries

\textbf{References}

Dahi, O.S. and Demir, F. (2013) `Preferential trade agreements and
manufactured goods exports: does it matter whom you PTA with?',
\emph{Applied Economics}, 45(34), pp. 4754--4772. Available at:
https://doi.org/10.1080/00036846.2013.804169.

Krugman, P. (1995) `Growing World Trade: Causes and Consequences',
\emph{Brookings Papers on Economic Activity} {[}Preprint{]}.

Puga, D. and Venables, A.J. (1998) `Trading Arrangements and Industrial
Development'.

Sanguinetti, P., Siedschlag, I. and Martincus, C.V. (2010) `The Impact
of South-South Preferential Trade Agreements on Industrial Development:
An Empirical Test', \emph{Journal of Economic Integration}, 25(1), pp.
69--103.

Schiff, M. and Wang, Y. (2008) `North-South and South-South
Trade-Related Technology Diffusion: How Important Are They in Improving
TFP Growth?', \emph{The Journal of Development Studies}, 44(1), pp.
49--59. Available at: https://doi.org/10.1080/00220380701722282.

Schiff, M.W., Winters, L.A. and Schiff, M. (2003) \emph{Regional
Integration And Development}. Washington, UNITED STATES: World Bank
Publications. Available at:
http://ebookcentral.proquest.com/lib/londonschoolecons/detail.action?docID=3050563
(Accessed: 12 August 2024).

Venables, A.J. (2003) `Winners and Losers from Regional Integration
Agreements', \emph{The Economic Journal}, 113(490), pp. 747--761.
Available at: https://doi.org/10.1111/1468-0297.t01-1-00155.
