This section reviews the literature on the theoretical and empirical
potential effects of PTAs on exports and welfare and situates the
analysis in the relevant field of research.

\textbf{Theoretical Framework}

Stumbling block vs building block dichotomy

\textbf{Comparative advantage and trade creation and diversion}

Traditional trade theory emphasizes trade creation (allowing cheaper
products from PTA members to substitute for more expensive domestic
products) and trade diversion (substituting products from non-PTA
members that were cheaper before the PTA with products from PTA members
that are cheaper now due to the PTA reducing tariffs) (Schiff, Winters
and Schiff, 2003) and argues that the impact of PTAs depends on the
comparative advantage of member countries. In particular, it argues that
PTAs magnify the impacts of a country's comparative advantage, relative
to the world and to other member countries signatories of a common PTA.
If member countries of a PTA have a comparative advantage on a factor
endowment relative to the world, but one country also has a comparative
advantage on the same factor endowment relative to the other member
countries, the country with the ``extreme'' advantage will be more
vulnerable to trade diversion effects, while countries with
``intermediate'' advantages will gain from trade creation effects,
predicting divergence of trade outcomes, and winners and losers among
member countries. (Venables, 2003). This emphasis on the trade creation
and trade diversion effects among member countries with significant
differences in the comparative advantage of their factor endowments
relative to the world and to each other, suggests that, when the country
with the ``extreme'' comparative advantage is a high-income country,
relative to a lower-income country with an ``intermediate'' comparative
advantage, the lower-income country should seek a PTA with the other
high-income country as it will gain more. On the contrary, if both
members are lower-income countries, the country with the ``extreme''
comparative advantage, should not seek a PTA with the other low-income
member country as it will be vulnerable. (Sanguinetti, Siedschlag and
Martincus, 2010). This logic can be easily extended to the North-South
and South-South types of PTAs, as ``North'' countries will reasonably
have an ``extreme'' comparative advantage in skill-intensive goods
relative to ``South'' countries, while ``South'' countries will
reasonably have an ``extreme'' comparative advantage in labour-intensive
goods relative to ``North'' countries. Furthermore, it is also argued in
the literature that benefiting from economies of scale through
South-South economic integration is more difficult because member
countries do not have complementary production and trade structures, nor
high interpenetration of each other's markets on intra-industry trade.
(Schiff, Winters and Schiff, 2003). Also, South countries can benefit
from greater technological diffusion from North-South PTAs as the
``North'' countries have higher industrial development as well as
investment in research (Schiff and Wang, 2008). Finally, as the trend in
manufacturing has been in favour of vertical specialization or value
chain fragmentation (Krugman, 1995), North-South PTAs are preferable as
developing countries strive to capture a greater portion of the value
added. Based on these arguments, developing countries should therefore
be better off entering into North-South rather than South-South
agreements.

\textbf{Economies of Scale, Input-Output linkages and Products Exported}

In contrast, classical development theory and new trade literature go
beyond the static welfare gains from trade creation and diversion
effects when analysing the effect of PTAs. Developing countries can use
PTAs to overcome limitations of their domestic market size in the
industrialization process (Dahi and Demir, 2013). Such potential
increases in the effective market size could help industries in
developing countries achieve economies of scale and increase the skill
content of production and exports, which in turn could improve the
market penetration of exports of developing countries in developed
markets in industrial products (Fugazza and Robert-Nicoud, 2006). Also,
due to similarities in production patterns and resource base among
developing countries, incentivising trade by lowering barriers could
facilitate appropriate technology transfer, according to the needs of
developing countries (UNIDO, 2006). Of particular relevance for
developing countries, it is argued that the products that countries
export matter for long-term economic performance. If a country exports
products from industries that are more technology-intensive, these are
likely to create input-output linkages and spillover effects in human
and physical capital accumulation and innovation (Hausmann, Hwang and
Rodrik, 2007). Furthermore, by allowing for factor accumulation, PTAs
can reduce intra-block trade barriers and increase competition and
access to cheaper intermediate goods, triggering changes in industrial
production in member countries. As such, PTAs among ``South'' countries
can reduce intra-South barriers and lead to industrialization of the
region (Puga and Venables, 1998). In this context, what matters are not
static gains from PTAs, but dynamic gains in industrial development. If
South-South PTAs truly promote industrial development of member
countries, they might be desirable even if there are short-term losses
due to trade diversion (Dahi and Demir, 2013). Other arguments in the
development literature emphasize the asymmetries in bargaining power
between ``North'' and ``South'' countries, which could lead to worse
outcomes for developing countries if their policy space gets restricted
(Thrasher and Gallagher, 2008). To the extent that these arguments hold
true, developing countries could be better off entering into South-South
rather than North-South agreements, or at least should pursue both kinds
of agreements.

\textbf{Empirical Evidence}

The preference of a type of partner in a PTAs then becomes an empirical
question. Do South-South PTAs promote trade and industrial development
among their members? The empirical literature overall reports positive
effects of PTAs on the trade of member countries, but with considerable
heterogeneity on the estimation coefficients. For example, a
meta-analysis of research papers on the effects of PTAs on member trade,
encompassing 85 papers and 1827 estimates, finds an average of 0.59 (an
80\% increase in trade), with a median of 0.38 (a 46\% increase in
trade), a wide range of coefficient estimates (-9.01 to 15.41), and only
312 out of 1827 estimates reported as negative (Cipollina and Salvatici,
2010). Furthermore, a survey of the empirical research on the effect of
economic integration agreements on international trade flows, as well as
using the most modern econometric techniques to address biases, found an
increase of 50\% on international trade, but with significant variation
in the effects of specific agreements (Kohl, 2014). However, much of the
empirical research is focused on the effects of PTAs on or including the
most advanced economies. Empirical research focused exclusively on the
effects of South-South PTAs or comparing them to the effects of
North-North or North-South PTAs, is much less prevalent in the
literature. However, several research papers do control for the type of
agreement (North-South or South-South) and have found positive and
significant effects of South-South PTAs (Medvedev, 2006; Mayda and
Steinberg, 2007; Dahi and Demir, 2013; Deme and Ndrianasy, 2017), but
these articles tend to be limited in their scope, sample size or only
focus on trade volumes.

\textbf{Significance of Exports}