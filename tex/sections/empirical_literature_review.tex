The preference of a type of partner in a PTAs then becomes an empirical
question. Do South-South PTAs promote trade and industrial development
among their members? The empirical literature overall reports positive
effects of PTAs on the trade of member countries, but with considerable
heterogeneity on the estimation coefficients. For example, a
meta-analysis of research papers on the effects of PTAs on member trade,
encompassing 85 papers and 1827 estimates, finds an average of 0.59 (an
80\% increase in trade), with a median of 0.38 (a 46\% increase in
trade), a wide range of coefficient estimates (-9.01 to 15.41), and only
312 out of 1827 estimates reported as negative (Cipollina and Salvatici,
2010). Furthermore, a survey of the empirical research on the effect of
economic integration agreements on international trade flows, as well as
using the most modern econometric techniques to address biases, found an
increase of 50\% on international trade, but with significant variation
in the effects of specific agreements (Kohl, 2014). However, much of the
empirical research is focused on the effects of PTAs on or including the
most advanced economies. Empirical research focused exclusively on the
effects of South-South PTAs or comparing them to the effects of
North-North or North-South PTAs, is much less prevalent in the
literature. However, several research papers do control for the type of
agreement (North-South or South-South) and have found positive and
significant effects of South-South PTAs (Medvedev, 2006; Mayda and
Steinberg, 2007; Dahi and Demir, 2013; Deme and Ndrianasy, 2017), but
these articles tend to be limited in their scope, sample size or only
focus on trade volumes.