This is the analysis and discussion

``Several themes emerge from this newly bourgeoning literature. First,
South--South trade and finance is now a significant economic and
political force for South countries as well as for the global economy.
There is a near consensus therefore that South--South economic relations
do matter and that they have the potential to have a significant
developmental impact. Moreover, this impact may be positive or negative,
that is, that it may help or hinder the long-term developmental goals of
exchanging parties. Second, much of South--South manufactures trade is
concentrated in high-technology-and-skill content, opening the door for
potential long-run dynamic gains from trade. However, these gains are
being increasingly concentrated within a small number of South
countries. The global South is, in fact, splitting into two groups,
which we refer to as the Emerging South and the Rest of South with very
different outcomes. While there is evidence for gains through
South--South trade, there is also evidence that the Emerging South is
rising at the expense of the Rest of South. Finally, the South--South
exchanges have expanded significantly to cover issues including
financial flows and technology transfer, among other topics. The overall
conclusion of this diverse literature is that while it does matter who
is exchanging what and with whom, South--South trade is not a panacea
for the development challenges in Southern countries. On the contrary,
South--South exchange themselves may become a potential threat for
development for some of the Southern countries.'' (Dahi \& Demir, 2017)

References

Dahi, O. S., \& Demir, F. (2017). South-South and North-South Economic
Exchanges: Does It Matter Who Is Exchanging What and with Whom?
\emph{Journal of Economic Surveys}, \emph{31}(5), 1449--1486.
https://doi.org/10.1111/joes.12225
