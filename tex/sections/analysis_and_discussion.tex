Our analysis finds evidence of positve, negative and not significant
effects of PTAs on both South-South and North-South trade relationships,
on trade volumes and on the value per unit of manufacturing products
exported. The magnitudes of our findings are similar to the estimates in
the empirical literature on the effects of PTAs on trade.

Our findings on the heterogeneous of effects of PTAs appear to indicate
that PTAs can have positive and negative effects on North-South and
South-South bilateral trade relationships, and that declaring them as
stumbling or building blocks of industrial development and growth is not
straight forward.

Potential Mechanismss

What PTAs cover might matter

As more South-South PTAs are signed, and more of the share of global
trade happens among South countries, the North-South distinction also
starts to loose relevance. Evidence appears to show that the ``South''
is spliting into groups, with the ``Emerging South'' growing at an
accelerated pace and even challenging the hegemony that developed
economies that enjoyed since the Post-World War II period. It could be
the case that the same power dynamics observed between developed and
developing countries by the classical development literature also occur
in South-South relationships, and that they become a threat for the
development of the least-developed economies in the South (Dahi \&
Demir, 2017). It is clear that more research focused on South-South
dynamics is needed in order to guide the policy decisions of different
groups of countries.

Policy recommendations?

Assessment of country's current capabilities and analysis of related
products and industries. Push to acquire new capabilities close-by in
relatedness to the capabilities already in place.

Limitations

Although the predictive power of the Gravity Model of Trade is well
established in the relevant literature, and we have done our best to
follow the best practices to avoid endogeneity when studying the effects
of preferential agreements on international trade, it is important to
note that out empirical analysis does not claim to achieve a causal
inference on the effects of PTAs. There could be other policies and
forces driving the effects described in our estimates. Also, since the
period estudied comprehends the global financial crisis of 2007-2008, it
is possible that running the same models for other periods of time could
find different results. Our estimates could also be constrained by the
quality of the data and reportig or measurement error in trade flows,
particularly in South countries without robust institutional capacity
and estatistical infrastructure. By using relatively modern data we hope
to mitigate this concern, but we ackowledge that the data of the first
half of our period studied (1995-2005) might be less accurate than the
later period (2005-2015). Still, this research provides useful insights,
even if they are just illustrative, on the hteterogeneous effects of
PTAs, and their development potential and use by developing countries.