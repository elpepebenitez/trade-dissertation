As a compliment to trade theory, new trade theory and classical
development theory, there is a recent strand of the academic literature
that emphasises the importance of the structure of the product space
exported by each country in the structural transformation process.
Beyond factor endowments of physical, human and institutional capital,
and their subsequent evolution through accumulation processes, as the
basis for the comparative advantage of countries, this literature
proposes and finds evidence of patterns of path dependence depending on
the current capabilities of countries, and the relatedness of current
capabilities to the capabilities required to produce new products in the
future (Hausmann and Klinger, 2007). As it is observed that human
capital for one product is imperfectly substitutable for other products,
and the degree of substitutability determines the relatedness of
products, the implication is that, as countries experience a strong
tendency to move into products related to that require the capabilities
that a country already has or that are similar, the opportunities for
future transformation are dictated by the current product space and its
proximity to related products. Moreover, it also implies that there is a
positive exponential relationship between the returns to the
accumulation of new capabilities and the capabilities present in a
country. (Hausmann and Hidalgo, 2010). The more diverse the product
structure of a country, the higher the returns to accumulate new
capabilities. Inversely, we can find a ``trap of economic stasis'', in
which countries with few capabilities have little incentives to
accumulate new capabilities as they will have negligible or no returns,
predicting a world of divergence in industrial development. Furthermore,
this literature suggests that countries converge to the level of income
determined by their productive structures and how complex they are
(Hidalgo and Hausmann, 2009). An underlying assumption of this
literature is that what a country exports matter and signals valuable
information about a country's comparative advantage and productive
structure, not only on its current industries and capabilities, but also
on a component of the evolution of its comparative advantage based on
the relatedness to other industries and capabilities (Hausmann \emph{et
al.}, 2014). The implications of this literature to the effects of TAs
appears to be relevant to the extent that TAs help countries acquire new
capabilities and diversify their structure of product space. Logically,
although North-South trade has the largest potential to allow South
countries to acquire new capabilities, we expect that the highest
returns should be made by acquiring capabilities in industries and
products related to the current capabilities of countries, which in the
context of our research should occur between countries with related
productive structures. As such, South-South trade could function as a
building block for developing countries to acquire new capabilities and
diversify the structure of their product space, before they can take
advantage of acquiring new capabilities through North-South trade.
