Often referred as the ``workhorse'' of international trade, the gravity
model is prominent in the empirical literature of applied international
trade analysis. Among the arguments that could support the use of the
gravity model, there are four that are particularly relevant for our
purposes. First, the gravity model of trade is intuitive to understand.
Following the metaphor of Newton's Law of Universal Gravitation, it
predicts that international trade between two countries is directly
proportional to the product of their economic size, and inversely
proportional to trade frictions between them. In simpler words, the
bigger (smaller) the economies of two countries, and the easier (harder)
it is for them to trade with each other, the more (less) we expect them
to trade. Second, it is referred to as a structural model with solid
theoretical foundations, which makes it appropriate for counterfactual
analysis, such as measuring the effects of trade policies as we aim to
do with the effects of North-South versus South-South agreements. Third,
model has a flexible structure, which will allow us to construct a
specification tailored to our research. Finally, fourth, it holds
consistent and remarkable predictive power, both with aggregate and
sectoral data (Yotov et al. 2016).

Through the decades, the gravity equation has been regularly upgraded in
the theoretical and empirical literature. Of relevance, the simple
intuition of the gravity model was theoretically extended by Anderson to
note that, after controlling for size, the increase or decrease is
\emph{relative} to the average barriers of the two countries with all
their partners, which are referred as ``multilateral resistance''
(Anderson 1979). The more trade barriers or resistance to trade exists
with other countries relative to a given partner, the more a country is
pushed to trade with said partner. Anderson also introduced the
assumptions of product differentiation by place of origin, and Constant
Elasticity of Substitution (CES) expenditures, or the Armington-CES
assumption (Yotov et al. 2016; Chatzilazarou and Dadakas 2023), which
led us to today's generalized form of the gravity equation, as developed
and popularised by Anderson and van Wincoop (Anderson and van Wincoop
2003).

Equally important, several empirical developments have strengthened the
gravity model and inform our choice of methodology: Exporter-time and
importer-time fixed effects are used to account for the multilateral
resistance terms in a gravity estimation with panel data (Olivero and
Yotov 2012); As the gravity model is often estimated with an OSL
estimator, zero-trade flows were dropped from the sample when trade was
transformed into a logarithmic form. Also, trade data is recognized to
suffer from heteroscedasticity (Yotov et al. 2016). To solve for
zero-trade flows and heteroscedasticity, the Poisson Pseudo Maximum
Likelihood (PPML) estimator has been proposed to estimate the gravity
model, avoiding potential biases (Silva and Tenreyro 2006; Santos Silva
and Tenreyro 2011); Country-pair fixed effects has been proposed to
account for the unobserved endogeneity of trade policy (Baier and
Bergstrand 2007). It is worth nothing that the inclusion of
exporter-time and importer-time fixed effects will absorb all observable
and unobservable time-varying country-specific characteristics that
could affect the dependent variable, while the country-pair fixed
effects will absorb observable and unobservable bilateral time-invariant
characteristics that could affect trade costs; The inclusion of
intra-trade flows as well as international trade flows is proposed to
correctly estimate the effects of non-discriminatory trade policy,
allowing for consumers to choose products from both international and
domestic sources (Dai, Yotov, and Zylkin 2014; Heid, Larch, and Yotov
2017); Year-intervals instead of data pooled over consecutive years
should be used to allow for adjustment of trade flows to policies that
might not have immediate effects (Baier and Bergstrand 2007; Anderson
and Yotov 2016); And finally, to account for the effects of
globalization forces that may biased the estimates of trade policies, a
set of globalization dummies are recommended to control for the effects
of globalization in the gravity model (Yotov 2012; Bergstrand, Larch,
and Yotov 2015).
