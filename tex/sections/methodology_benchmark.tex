Based on the theoretical and empirical best-practices found in the
relevant literature, we employ the following gravity equation using a
PPML estimator and a balanced panel data approach with multiple
exporters, multiple importers and time as our benchmark model:

\begin{multline}
    X_{ij,t} = \exp\left(\eta_{i,t} + \psi_{j,t} + \gamma_{\binom{-}{ij}} + \beta_{1} \, TA_{ij,t} \right. + \beta_{2} \, TA_{ij,t-5} + \left. \sum_{t} b_{t} \right) + \epsilon_{ij,t}
\end{multline}

Where \(X_{ij,t}\) denotes the value of exports from an origin country
\(i\) to a destination country \(j\); \(\eta_{i,t}\) and \(\psi_{j,t}\)
are, respectively, exporter-time and importer-time fixed-effects;
\(\gamma_{\binom{-}{ij}}\) is a country-pair fixed-effect;
\({TA}_{ij,t}\) and \({TA}_{ij,t - 5}\) are our main variables of
interest, which, respectively indicate if \(i\) and \(j\) are members of
a TA at time \(t\) and, to account for potential ``phase-in'' effects
over time of the TA, at time \(t - 5\); \(\sum_{t}^{}b_{t}\) is a set of
dummies that equal 1 for international trade and 0 for domestic trade
observations at each time \(t\); and \(\epsilon_{ij,t}\) is an error
term.
