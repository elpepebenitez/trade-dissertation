Defining which countries belong to the ``North'' and ``South''
categories is a key step in order to properly analyse the impact of PTAs
on different bilateral export relationships. However, it is important to
consider that any way in which we categorize countries can be criticised
for not taking into consideration the diverse and heterogenous
characteristics of individual countries within each group. Furthermore,
especially since our focus is to analyse South-South relationships, it
is possible to further disaggregate from the ``South'' group the
emerging economies which are becoming more relevant at the political and
economic world stage and are challenging the hegemony of traditional
developed economies. The level of disaggregation, as well as the level
of attention to heterogenous characteristics among and within groups,
depends on the research question at hand. For the purposes of this
paper, we will not consider such heterogeneity within groups, and just
focus on categorising countries as ``North'' and ``South'', but by no
means does this assumes that countries are homogenous within groups.
This is just a useful distinction to study heterogeneity across PTA
effects.

One intuitive approach could be to categorize countries based on their
income level, but this approach would need to deal with a dynamic list
of groups, as countries change their category through time. Also,
high-income countries include non-industrialized small-nations which we
do not expect to generate significant effects on the industrial
development as well as technology- and skills-upgrading of trade-partner
countries. For such reasons, we have decided to use the same
categorization of countries as Dahi \& Demir (Dahi and Demir 2017) which
takes into consideration characteristics such as incomes, production and
trade structures, factor endowments, and human and institutional
development to construct a list of ``North'' and ``South'' countries,
and also keeps the groups consistent over time. This results in 23
countries categorized as ``North'', and the rest as ``South''. A
detailed list of the countries and their categories can be found in the
Appendix.