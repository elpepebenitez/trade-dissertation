Although the predictive power of the Gravity Model of Trade is well
established in the relevant literature, and we have done our best to
follow the best practices to avoid endogeneity when studying the effects
of preferential agreements on international trade, it is important to
note that out empirical analysis does not claim to achieve a causal
inference on the effects of TAs. There could be other policies and
forces driving the effects described in our estimates. Also, since the
period studied comprehends the global financial crisis of 2007-2008, it
is possible that running the same models for other periods of time could
find different results. Our estimates could also be constrained by the
quality of the data and reporting or measurement error in trade flows,
particularly in South countries without robust institutional capacity
and statistical infrastructure. By using relatively modern data we hope
to mitigate this concern, but we acknowledge that the data of the first
half of our period studied (1995-2005) might be less accurate than the
later period (2005-2015). Still, this research provides useful insights,
even if they are just illustrative, on the heterogeneous effects of TAs,
and their development potential and use by developing countries.