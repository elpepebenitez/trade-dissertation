Traditional trade theory emphasizes trade creation (allowing cheaper
products from PTA members to substitute for more expensive domestic
products) and trade diversion (substituting products from non-PTA
members that were cheaper before the PTA with products from PTA members
that are cheaper now due to the PTA reducing tariffs) (Schiff, Winters
and Schiff, 2003) and argues that the impact of PTAs depends on the
comparative advantage of member countries. In particular, it argues that
PTAs magnify the impacts of a country's comparative advantage, relative
to the world and to other member countries signatories of a common PTA.
If member countries of a PTA have a comparative advantage on a factor
endowment relative to the world, but one country also has a comparative
advantage on the same factor endowment relative to the other member
countries, the country with the ``extreme'' advantage will be more
vulnerable to trade diversion effects, while countries with
``intermediate'' advantages will gain from trade creation effects,
predicting divergence of trade outcomes, and winners and losers among
member countries. (Venables, 2003). This emphasis on the trade creation
and trade diversion effects among member countries with significant
differences in the comparative advantage of their factor endowments
relative to the world and to each other, suggests that, when the country
with the ``extreme'' comparative advantage is a high-income country,
relative to a lower-income country with an ``intermediate'' comparative
advantage, the lower-income country should seek a PTA with the other
high-income country as it will gain more. On the contrary, if both
members are lower-income countries, the country with the ``extreme''
comparative advantage, should not seek a PTA with the other low-income
member country as it will be vulnerable. (Sanguinetti, Siedschlag and
Martincus, 2010). This logic can be easily extended to the North-South
and South-South types of PTAs, as ``North'' countries will reasonably
have an ``extreme'' comparative advantage in skill-intensive goods
relative to ``South'' countries, while ``South'' countries will
reasonably have an ``extreme'' comparative advantage in labour-intensive
goods relative to ``North'' countries. Furthermore, it is also argued in
the literature that benefiting from economies of scale through
South-South economic integration is more difficult because member
countries do not have complementary production and trade structures, nor
high interpenetration of each other's markets on intra-industry trade.
(Schiff, Winters and Schiff, 2003). Also, South countries can benefit
from greater technological diffusion from North-South PTAs as the
``North'' countries have higher industrial development as well as
investment in research (Schiff and Wang, 2008). Finally, as the trend in
manufacturing has been in favour of vertical specialization or value
chain fragmentation (Krugman, 1995), North-South PTAs are preferable as
developing countries strive to capture a greater portion of the value
added. Based on these arguments, developing countries should therefore
be better off entering into North-South rather than South-South
agreements.