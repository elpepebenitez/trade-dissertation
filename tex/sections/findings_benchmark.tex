We begin by briefly discussing the results of our benchmark estimation
by region, contained in Table 1. We immediately see that the average total or
``cumulative'' effects of TAs on trade flows, relative to non-TA-members, after accounting for
phase-in effects (the sum of the current and lagged TA estimates), is
heterogenous across regions. Only Americas, Europe and Intercontinental
TAs have statistically significant results, with all coefficients being
positive and generally similar to the results we would expect according
to the literature. The smallest effect, that of Intercontinental TAs,
has a statistically significant coefficient at the 5\% of 0.203 with a
standard error of (0.106). We interpret this coefficient as
Intercontinental TAs having an average a partial effect of
(exp(0.203)-1)x100\%. = 22.5\% increase in trade flows. The largest
effect, that of Europe's TAs, has a statistically significant
coefficient at the 1\% of 0.475 with a standard error of (0.025). We
interpret this coefficient as Europe's TAs having an average a partial
effect of (exp(0.475)-1)x100\%. = 60.8\% increase in trade flows. On the
other hand, Africa and Asia does not have statistically significant
results, with Asia's coefficient taking a negative value. Interestingly,
Africa's TA coefficient is highly significant and positive, and TA Lag
is not significant and negative, while Asia's TA coefficient is not
significant and positive, and TA Lag is highly significant and negative.
