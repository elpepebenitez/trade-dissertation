\documentclass[12pt]{article}%
\usepackage[T1]{fontenc}%
\usepackage[utf8]{inputenc}%
\usepackage{lmodern}%
\usepackage{textcomp}%
\usepackage{lastpage}%
\usepackage[backend=biber]{biblatex}%
\usepackage{fancyhdr}%
\usepackage{lastpage}%
\usepackage{ragged2e}%
\usepackage{pdfpages}%
\usepackage{hyperref}%
\usepackage{setspace}%
\usepackage{booktabs}%
\usepackage{float}%
\usepackage{threeparttable}%
\usepackage{amssymb}%
\usepackage{amsmath}%
\usepackage{adjustbox}%
\usepackage{longtable}%
\usepackage{breqn}%
\usepackage{tabularx}%
\usepackage{graphicx}%
\usepackage{placeins}%
\usepackage{geometry}%
\usepackage{fancyhdr}%
%
\geometry{top=1in, bottom=1in, left=1in, right=1in, headheight=14.5pt}%
\addbibresource{references.bib}%
\hypersetup{colorlinks=true, linkcolor=blue, urlcolor=blue}%
\fancypagestyle{frontmatter}{%
\renewcommand{\headrulewidth}{0pt}%
\renewcommand{\footrulewidth}{0pt}%
\fancyhead{%
}%
\fancyfoot{%
}%
\fancyhf{}%
\fancyhead[L]{DV410}%
\fancyhead[C]{\thepage}%
\fancyhead[R]{23802}%
}%
\fancypagestyle{mainmatter}{%
\renewcommand{\headrulewidth}{0pt}%
\renewcommand{\footrulewidth}{0pt}%
\fancyhead{%
}%
\fancyfoot{%
}%
\fancyhf{}%
\fancyhead[L]{DV410}%
\fancyhead[C]{Page \thepage\ of \pageref{LastPage}}%
\fancyhead[R]{23802}%
}%
\setstretch{1.5}%
\justifying%
\pagenumbering{roman}%
\pagestyle{fancy}%
\fancyhf{}%
\fancyhead[L]{DV410}%
\fancyhead[C]{Page \thepage\ of \pageref{LastPage}}%
\fancyhead[R]{23802}%
%TC:ignore%
\title{
        \begin{flushright}
        \large \textbf{Candidate Number: 23802}
        \end{flushright}
        \vspace*{30mm}
        \begin{center}
        \large MSc in Development Management 2023 \\
        \large (Applied Development Economics Specialism) \\
        \vspace*{5mm}
        Dissertation submitted in partial fulfilment of the requirements of the degree. \\
        \vspace*{35mm}
        \Huge \textbf{Title title title} \\
        \vspace*{20mm}
        \end{center}
    }%
\date{}%
%TC:endignore%
%
\begin{document}%
\normalsize%
\includepdf[pages=-, offset=0 -1cm, frame]{DV410_Dissertation Cover Sheet_ Consent Form_Front Page_2023-24.pdf}%
\pagestyle{frontmatter}%
\maketitle%

\vfill
\begin{center}\textbf{Word Count: 7182}\end{center}
\newpage%
%TC:ignore%
\vspace*{\fill}%
\begin{center}%
\begin{minipage}{0.8\textwidth}%
\begin{center}%
\section*{Abstract}%
\end{center}%
\justify%
This is the abstract text. It should be centered on the page, and the text should be justified. This is the abstract text. It should be centered on the page, and the text should be justified. This is the abstract text. It should be centered on the page, and the text should be justified. This is the abstract text. It should be centered on the page, and the text should be justified. This is the abstract text. It should be centered on the page, and the text should be justified. %
\end{minipage}%
\end{center}%
\vspace*{\fill}%
%TC:endignore%
\newpage%
\tableofcontents%
\newpage%
%TC:ignore%
\section*{Abbreviations}%
\label{sec:Abbreviations}%
\begin{tabbing}%
\hspace{3cm} \= \kill%
\textbf{AI} \> Artificial Intelligence \\%
\textbf{API} \> Application Programming Interface \\%
\textbf{CPU} \> Central Processing Unit \\%
\textbf{GPU} \> Graphics Processing Unit \\%
\textbf{IoT} \> Internet of Things \\%
\textbf{ML} \> Machine Learning \\%
\textbf{NLP} \> Natural Language Processing \\%
\textbf{RAM} \> Random Access Memory \\%
\textbf{UI} \> User Interface \\%
\textbf{UX} \> User Experience \\%
\end{tabbing}

%
%TC:endignore%
\newpage%
%TC:ignore%
\listoffigures%
%TC:endignore%
\newpage%
%TC:ignore%
\listoftables%
%TC:endignore%
\newpage%
\pagenumbering{arabic}%
\pagestyle{mainmatter}%
\section{Introduction}%
\label{sec:Introduction}%
This is the introduction

\textbf{Stylised facts}

\textbf{Ideas:}

Total number of SS, NS and NN PTAs

Share of SS, NS and NN PTAs

Total exports by S and N countries

Share of total exports by S and N countries

Total exports of manufactured by S and N countries

Share of total exports of manufactured by S and N countries

Number of products exported by S and N countries

The organization of this article is as follows: Section II provides a
brief literature review of the PTAs, South--South trade and the
importance of the structure of trade. Section III introduces the
methodology and data. Section IV presents the empirical results fol-
lowed by a discussion of the robustness tests. Section V concludes.
%
This is the introduction section. Here is a citation: %
\cite{dahi_preferential_2013}

%
\section{Literature Review}%
\label{sec:LiteratureReview}%
This section reviews the literature on the theoretical and empirical potential effects of PTAs on exports and welfare and situates the analysis in the relevant field of research.%
\subsection{Theoretical Framework}%
\label{subsec:TheoreticalFramework}%
Stumbling block vs building block dichotomy.%
\subsubsection{Comparative advantage and trade creation and diversion}%
\label{ssubsec:Comparativeadvantageandtradecreationanddiversion}%

%
Traditional trade theory emphasizes trade creation (allowing cheaper
products from PTA members to substitute for more expensive domestic
products) and trade diversion (substituting products from non-PTA
members that were cheaper before the PTA with products from PTA members
that are cheaper now due to the PTA reducing tariffs) (Schiff, Winters
and Schiff, 2003) and argues that the impact of PTAs depends on the
comparative advantage of member countries. In particular, it argues that
PTAs magnify the impacts of a country's comparative advantage, relative
to the world and to other member countries signatories of a common PTA.
If member countries of a PTA have a comparative advantage on a factor
endowment relative to the world, but one country also has a comparative
advantage on the same factor endowment relative to the other member
countries, the country with the ``extreme'' advantage will be more
vulnerable to trade diversion effects, while countries with
``intermediate'' advantages will gain from trade creation effects,
predicting divergence of trade outcomes, and winners and losers among
member countries. (Venables, 2003). This emphasis on the trade creation
and trade diversion effects among member countries with significant
differences in the comparative advantage of their factor endowments
relative to the world and to each other, suggests that, when the country
with the ``extreme'' comparative advantage is a high-income country,
relative to a lower-income country with an ``intermediate'' comparative
advantage, the lower-income country should seek a PTA with the other
high-income country as it will gain more. On the contrary, if both
members are lower-income countries, the country with the ``extreme''
comparative advantage, should not seek a PTA with the other low-income
member country as it will be vulnerable. (Sanguinetti, Siedschlag and
Martincus, 2010). This logic can be easily extended to the North-South
and South-South types of PTAs, as ``North'' countries will reasonably
have an ``extreme'' comparative advantage in skill-intensive goods
relative to ``South'' countries, while ``South'' countries will
reasonably have an ``extreme'' comparative advantage in labour-intensive
goods relative to ``North'' countries. Furthermore, it is also argued in
the literature that benefiting from economies of scale through
South-South economic integration is more difficult because member
countries do not have complementary production and trade structures, nor
high interpenetration of each other's markets on intra-industry trade.
(Schiff, Winters and Schiff, 2003). Also, South countries can benefit
from greater technological diffusion from North-South PTAs as the
``North'' countries have higher industrial development as well as
investment in research (Schiff and Wang, 2008). Finally, as the trend in
manufacturing has been in favour of vertical specialization or value
chain fragmentation (Krugman, 1995), North-South PTAs are preferable as
developing countries strive to capture a greater portion of the value
added. Based on these arguments, developing countries should therefore
be better off entering into North-South rather than South-South
agreements.%
\subsubsection{Economies of Scale, Input{-}Output linkages and Products Exported}%
\label{ssubsec:EconomiesofScale,Input{-}OutputlinkagesandProductsExported}%

%
In contrast, classical development theory and new trade literature go
beyond the static welfare gains from trade creation and diversion
effects when analysing the effect of PTAs. Developing countries can use
PTAs to overcome limitations of their domestic market size in the
industrialization process (Dahi and Demir, 2013). Such potential
increases in the effective market size could help industries in
developing countries achieve economies of scale and increase the skill
content of production and exports, which in turn could improve the
market penetration of exports of developing countries in developed
markets in industrial products (Fugazza and Robert-Nicoud, 2006). Also,
due to similarities in production patterns and resource base among
developing countries, incentivising trade by lowering barriers could
facilitate appropriate technology transfer, according to the needs of
developing countries (UNIDO, 2006). Of particular relevance for
developing countries, it is argued that the products that countries
export matter for long-term economic performance. If a country exports
products from industries that are more technology-intensive, these are
likely to create input-output linkages and spillover effects in human
and physical capital accumulation and innovation (Hausmann, Hwang and
Rodrik, 2007). Furthermore, by allowing for factor accumulation, PTAs
can reduce intra-block trade barriers and increase competition and
access to cheaper intermediate goods, triggering changes in industrial
production in member countries. As such, PTAs among ``South'' countries
can reduce intra-South barriers and lead to industrialization of the
region (Puga and Venables, 1998). In this context, what matters are not
static gains from PTAs, but dynamic gains in industrial development. If
South-South PTAs truly promote industrial development of member
countries, they might be desirable even if there are short-term losses
due to trade diversion (Dahi and Demir, 2013). Other arguments in the
development literature emphasize the asymmetries in bargaining power
between ``North'' and ``South'' countries, which could lead to worse
outcomes for developing countries if their policy space gets restricted
(Thrasher and Gallagher, 2008). To the extent that these arguments hold
true, developing countries could be better off entering into South-South
rather than North-South agreements, or at least should pursue both kinds
of agreements.

%
\subsection{Empirical Evidence}%
\label{subsec:EmpiricalEvidence}%

%
The preference of a type of partner in a PTAs then becomes an empirical
question. Do South-South PTAs promote trade and industrial development
among their members? The empirical literature overall reports positive
effects of PTAs on the trade of member countries, but with considerable
heterogeneity on the estimation coefficients. For example, a
meta-analysis of research papers on the effects of PTAs on member trade,
encompassing 85 papers and 1827 estimates, finds an average of 0.59 (an
80\% increase in trade), with a median of 0.38 (a 46\% increase in
trade), a wide range of coefficient estimates (-9.01 to 15.41), and only
312 out of 1827 estimates reported as negative (Cipollina and Salvatici,
2010). Furthermore, a survey of the empirical research on the effect of
economic integration agreements on international trade flows, as well as
using the most modern econometric techniques to address biases, found an
increase of 50\% on international trade, but with significant variation
in the effects of specific agreements (Kohl, 2014). However, much of the
empirical research is focused on the effects of PTAs on or including the
most advanced economies. Empirical research focused exclusively on the
effects of South-South PTAs or comparing them to the effects of
North-North or North-South PTAs, is much less prevalent in the
literature. However, several research papers do control for the type of
agreement (North-South or South-South) and have found positive and
significant effects of South-South PTAs (Medvedev, 2006; Mayda and
Steinberg, 2007; Dahi and Demir, 2013; Deme and Ndrianasy, 2017), but
these articles tend to be limited in their scope, sample size or only
focus on trade volumes.%
\subsection{Significance of Exports}%
\label{subsec:SignificanceofExports}%

%
Significance of Exports

%
\section{Methodology}%
\label{sec:Methodology}%
\subsection{Empirical Strategy}%
\label{subsec:EmpiricalStrategy}%
\subsubsection{The Gravity Model of Trade}%
\label{ssubsec:TheGravityModelofTrade}%

%
Often referred as the ``workhorse'' of international trade, the gravity
model is prominent in the empirical literature of applied international
trade analysis. Among the arguments that could support the use of the
gravity model, there are four that are particularly relevant for our
purposes. First, the gravity model of trade is intuitive to understand.
Following the metaphor of Newton's Law of Universal Gravitation, it
predicts that international trade between two countries is directly
proportional to the product of their economic size, and inversely
proportional to trade frictions between them. In simpler words, the
bigger (smaller) the economies of two countries, and the easier (harder)
it is for them to trade with each other, the more (less) we expect them
to trade. Second, it is referred to as a structural model with solid
theoretical foundations, which makes it appropriate for counterfactual
analysis, such as measuring the effects of trade policies as we aim to
do with the effects of North-South versus South-South agreements. Third,
model has a flexible structure, which will allow us to construct a
specification tailored to our research. Finally, fourth, it holds
consistent and remarkable predictive power, both with aggregate and
sectoral data (Yotov et al. 2016).

Through the decades, the gravity equation has been regularly upgraded in
the theoretical and empirical literature. Of relevance, the simple
intuition of the gravity model was theoretically extended by Anderson to
note that, after controlling for size, the increase or decrease is
\emph{relative} to the average barriers of the two countries with all
their partners, which are referred as ``multilateral resistance''
(Anderson 1979). The more trade barriers or resistance to trade exists
with other countries relative to a given partner, the more a country is
pushed to trade with said partner. Anderson also introduced the
assumptions of product differentiation by place of origin, and Constant
Elasticity of Substitution (CES) expenditures, or the Armington-CES
assumption (Yotov et al. 2016; Chatzilazarou and Dadakas 2023), which
led us to today's generalized form of the gravity equation, as developed
and popularised by Anderson and van Wincoop (Anderson and van Wincoop
2003).

Equally important, several empirical developments have strengthened the
gravity model and inform our choice of methodology: Exporter-time and
importer-time fixed effects are used to account for the multilateral
resistance terms in a gravity estimation with panel data (Olivero and
Yotov 2012); As the gravity model is often estimated with an OSL
estimator, zero-trade flows were dropped from the sample when trade was
transformed into a logarithmic form. Also, trade data is recognized to
suffer from heteroscedasticity (Yotov et al. 2016). To solve for
zero-trade flows and heteroscedasticity, the Poisson Pseudo Maximum
Likelihood (PPML) estimator has been proposed to estimate the gravity
model, avoiding potential biases (Silva and Tenreyro 2006; Santos Silva
and Tenreyro 2011); Country-pair fixed effects has been proposed to
account for the unobserved endogeneity of trade policy (Baier and
Bergstrand 2007). It is worth nothing that the inclusion of
exporter-time and importer-time fixed effects will absorb all observable
and unobservable time-varying country-specific characteristics that
could affect the dependent variable, while the country-pair fixed
effects will absorb observable and unobservable bilateral time-invariant
characteristics that could affect trade costs; The inclusion of
intra-trade flows as well as international trade flows is proposed to
correctly estimate the effects of non-discriminatory trade policy,
allowing for consumers to choose products from both international and
domestic sources (Dai, Yotov, and Zylkin 2014; Heid, Larch, and Yotov
2017); Year-intervals instead of data pooled over consecutive years
should be used to allow for adjustment of trade flows to policies that
might not have immediate effects (Baier and Bergstrand 2007; Anderson
and Yotov 2016); And finally, to account for the effects of
globalization forces that may biased the estimates of trade policies, a
set of globalization dummies are recommended to control for the effects
of globalization in the gravity model (Yotov 2012; Bergstrand, Larch,
and Yotov 2015).%
\subsubsection{Benchmark Model}%
\label{ssubsec:BenchmarkModel}%

%
Based on the theoretical and empirical best-practices found in the
relevant literature, we employ the following gravity equation using a
PPML estimator and a balanced panel data approach with multiple
exporters, multiple importers and time as our benchmark model:

% \[(1)\ \ \ \ \ X_{ij,t} = \ exp(\eta_{i,t} + \psi_{j,t} + \gamma_{\binom{-}{ij}} + \beta_{1}{PTA}_{ij,t} + \beta_{2}{PTA}_{ij,t - 5} + \sum_{t}^{}b_{t}) + \epsilon_{ij,t}\]
\begin{multline}
    X_{ij,t} = \exp\left(\eta_{i,t} + \psi_{j,t} + \gamma_{\binom{-}{ij}} + \beta_{1} \, PTA_{ij,t} \right. + \beta_{2} \, PTA_{ij,t-5} + \left. \sum_{t} b_{t} \right) + \epsilon_{ij,t}
\end{multline}

Where \(X_{ij,t}\) denotes the value of exports from an origin country
\(i\) to a destination country \(j\); \(\eta_{i,t}\) and \(\psi_{j,t}\)
are, respectively, exporter-time and importer-time fixed-effects;
\(\gamma_{\binom{-}{ij}}\) is a country-pair fixed-effect;
\({PTA}_{ij,t}\) and \({PTA}_{ij,t - 5}\) are our main variables of
interest, which, respectively indicate if \(i\) and \(j\) are members of
a PTA at time \(t\) and, to account for potential ``phase-in'' effects
over time of the PTA, at time \(t - 5\); \(\sum_{t}^{}b_{t}\) is a set
of dummies that equal 1 for international trade and 0 for domestic trade
observations at each time \(t\); and \(\epsilon_{ij,t}\) is an error
term.%
\subsubsection{PTA Heterogeneity Model}%
\label{ssubsec:PTAHeterogeneityModel}%

%
In contrast with our main interest of research, which are the potential
heterogenous effects of PTAs on different members for different types of
agreements, this benchmark model, specifically
\(\beta = \beta_{1} + \beta_{2}\) , would provide the average ``total''
partial effect of PTAs on trade after accounting for lagged effects, but
it cannot provide the effects for a given agreement. As such, an
expansion can be implemented to capture heterogeneity in PTA effects as
proposed by Baier \emph{et al}. (Baier, Yotov, and Zylkin 2019):

% \[(2)\ \ \ \ \ X_{ij,t} = \ exp(\eta_{i,t} + \psi_{j,t} + \gamma_{\binom{-}{ij}} + \sum_{A}^{}{\beta_{1,A}{PTA}_{ij,t}} + \sum_{A}^{}{\beta_{2,A}{PTA}_{ij,t - 5}} + \sum_{t}^{}b_{t}) + \epsilon_{ij,t}\]
\begin{multline}
    X_{ij,t} = \exp\left(\eta_{i,t} + \psi_{j,t} + \gamma_{\binom{-}{ij}} + \sum_{A} \beta_{1,A} \, PTA_{ij,t} \right. + \sum_{A} \beta_{2,A} \, PTA_{ij,t-5} + \left. \sum_{t} b_{t} \right) + \epsilon_{ij,t}
\end{multline}

Equation (2) can be implemented to account for heterogeneous effects of
PTAs at the level of the specific agreement, by allowing for distinct
average partial effects for each individual agreement, using superscript
\(A\) to index by agreement and also allowing for agreement-specific
lags: \(\beta_{A} = \beta_{1,A} + \beta_{2,A}\).%
\subsubsection{North{-}North, North{-}South and South{-}South PTAs}%
\label{ssubsec:North{-}North,North{-}SouthandSouth{-}SouthPTAs}%

%
In order to analyse the differentiated effects of North-North,
North-South and South-South PTAs, we extend both models to get estimates
for each type of PTA. Our benchmark model is extended as follows:

% \[(3)\ \ \ \ \ X_{ij,t} = \ exp(\eta_{i,t} + \psi_{j,t} + \gamma_{\binom{-}{ij}} + \beta_{1NN}{PTA\_ NN}_{ij,t} + \beta_{2NN}{PTA\_ NN}_{ij,t - 5} + + \beta_{1NS}{PTA\_ NS}_{ij,t} + \beta_{2NS}{PTA\_ NS}_{ij,t - 5} + \beta_{1SS}{PTA\_ SS}_{ij,t} + \beta_{2SS}{PTA\_ SS}_{ij,t - 5} + \sum_{t}^{}b_{t}) + \epsilon_{ij,t}\]
\begin{multline}
    X_{ij,t} = \exp\left(\eta_{i,t} + \psi_{j,t} + \gamma_{\binom{-}{ij}} + \beta_{1NN} \, PTA\_NN_{ij,t} + \beta_{2NN} \, PTA\_NN_{ij,t-5} \right. \\
    + \beta_{1NS} \, PTA\_NS_{ij,t} + \beta_{2NS} \, PTA\_NS_{ij,t-5} + \beta_{1SS} \, PTA\_SS_{ij,t} + \beta_{2SS} \, PTA\_SS_{ij,t-5} \\
    + \left. \sum_{t} b_{t} \right) + \epsilon_{ij,t}
\end{multline}

Where \(X_{ij,t}\)\hspace{0pt} denotes the value of exports from country
\(i\) to country \(j\) at time \(t\); \(\eta_{i,t}\) and
\(\psi_{j,t}\ \)are exporter-time and importer-time fixed effects,
respectively; \(\gamma_{\binom{-}{ij}}\) is a country-pair fixed effect;
\hspace{0pt}\(\beta_{1NN}\) and \(\beta_{2NN}\) are the coefficients for
the immediate and lagged effects of a North-North PTA (\(PTA\_ NN\));
\hspace{0pt}\hspace{0pt}\(\beta_{1NS}\) and \(\beta_{2NS}\) are the
coefficients for the immediate and lagged effects of a North-South PTA
(\(PTA\_ SN\)); \hspace{0pt}\hspace{0pt}\(\beta_{1SS}\) and
\(\beta_{2SS}\) are the coefficients for the immediate and lagged
effects of a South-South PTA (\(PTA\_ SS\)); \(\sum_{t}^{}b_{t}\) is a
set of time dummies accounting for international trade-specific effects
at each time \(t\); and \(\epsilon_{ij,t}\) is the error term.

Equation (2) also gets extended to capture the heterogeneous effects of
the different types of PTAs as follows:

% \[(4)\ \ \ \ \ X_{ij,t} = \ exp(\eta_{i,t} + \psi_{j,t} + \gamma_{\binom{-}{ij}} + \sum_{A}^{}{(\beta_{1,A,NN}{PTA\_ NN}_{ij,t}\  + \ \beta_{2,A,NN}{PTA\_ NN}_{ij,t - 5})} + \sum_{A}^{}{(\beta_{1,A,NS}{PTA\_ NS}_{ij,t}\  + \ \beta_{2,A,NS}{PTA\_ NS}_{ij,t - 5})} + \sum_{A}^{}{(\beta_{1,A,SS}{PTA\_ SS}_{ij,t}\  + \ \beta_{2,A,SS}{PTA\_ SS}_{ij,t - 5})} + \sum_{t}^{}b_{t}) + \epsilon_{ij,t}\]
\begin{multline}
    X_{ij,t} = \exp\left(\eta_{i,t} + \psi_{j,t} + \gamma_{\binom{-}{ij}} + \sum_{A}\left(\beta_{1,A,NN} \, PTA\_NN_{ij,t} + \beta_{2,A,NN} \, PTA\_NN_{ij,t-5}\right) \right. \\
    + \sum_{A}\left(\beta_{1,A,NS} \, PTA\_NS_{ij,t} + \beta_{2,A,NS} \, PTA\_NS_{ij,t-5}\right) + \sum_{A}\left(\beta_{1,A,SS} \, PTA\_SS_{ij,t} + \beta_{2,A,SS} \, PTA\_SS_{ij,t-5}\right) \\
    + \left. \sum_{t} b_{t} \right) + \epsilon_{ij,t}
\end{multline}

Where \(X_{ij,t}\)\hspace{0pt} denotes the value of exports from country
\(i\) to country \(j\) at time \(t\); \(\eta_{i,t}\) and
\(\psi_{j,t}\ \)are exporter-time and importer-time fixed effects,
respectively; \(\gamma_{\binom{-}{ij}}\) is a country-pair fixed effect;
The summations \hspace{0pt}\(\sum_{}^{}A\) denote the sum over different
agreements \(A\) for: \(\beta_{1,A,NN}\) and \(\beta_{2,A,NN}\):
Coefficients for the immediate and lagged effects of North-North PTAs
\hspace{0pt}(\(PTA\_ NN\)); \(\beta_{1,A,NS}\) and \(\beta_{2,A,NS}\):
Coefficients for the immediate and lagged effects of North-South PTAs
(\(PTA\_ SN\)); \(\beta_{1,A,SS}\) and \(\beta_{2,A,SS}\): Coefficients
for the immediate and lagged effects of South-South PTAs (\(PTA\_ SS\));
\(\sum_{t}^{}b_{t}\) is a set of time dummies accounting for
trade-specific effects at each time \(t\); and \(\epsilon_{ij,t}\) is
the error term.

For both extended models we use the following variables:
\({PTA\_ NN}_{ij,t}\) is a dummy variable that takes the value of 1 if
the trade pair \((i,j)\) is North-North and part of a PTA at time \(t\),
and 0 otherwise; \({PTA\_ NN}_{ij,t - 5}\) is a dummy variable that
takes the value of 1 if the trade pair \((i,j)\) is North-North and was
part of a PTA at time \(t\)\emph{-5}, and 0 otherwise;
\({PTA\_ NS}_{ij,t}\) is a dummy variable that takes the value of 1 if
the trade pair \((i,j)\) is North-South and part of a PTA at time \(t\),
and 0 otherwise; \({PTA\_ NS}_{ij,t - 5}\) is a dummy variable that
takes the value of 1 if the trade pair \((i,j)\) is North-South and was
part of a PTA at time \(t\)\emph{-5}, and 0 otherwise;
\({PTA\_ SS}_{ij,t}\) is a dummy variable that takes the value of 1 if
the trade pair \((i,j)\) is South-South and part of a PTA at time \(t\),
and 0 otherwise; \({PTA\_ SS}_{ij,t - 5}\) is a dummy variable that
takes the value of 1 if the trade pair \((i,j)\) is South-South and was
part of a PTA at time \(t\)\emph{-5}, and 0 otherwise;

The extended models allow us to capture the differentiated effects of
PTAs on bilateral exports depending on whether the pair country are two
``North'' countries (NN), a ``North'' and a ``South'' country (NS), or
two ``South'' countries (SS).

%
\subsection{Export Product Unit Value}%
\label{subsec:ExportProductUnitValue}%

%
Inspired by other strands of the international trade literature, we also
test our models using ``Unit Values'' of the products exported, by
dividing the total value exported by the total weight exported in
kilograms (Latzer and Mayneris 2021; Manova and Zhang 2012; Bastos and
Silva 2010). Using the unit value as the dependent variable in our
estimations allow us to analyse if the value per unit exported is
affected by PTAs. To be consistent in our effort to understand the
potentially heterogenous effects of PTAs according to the different
category of the members in trade volume, but also in quality upgrading
and industrialization development of countries, we focus on
manufacturing products (Chatzilazarou and Dadakas 2023) with HS 2-digit
codes 84 (Nuclear reactors, boilers, machinery and mechanical
appliances; parts thereof ) and 85 (Electrical machinery and equipment
and parts thereof; sound recorders and reproducers, television image and
sound recorders and reproducers, and parts and accessories of such
articles) which are part of the ``Machinery and mechanical appliances;
electrical equipment; parts thereof; sound recorders and reproducers,
television image and sour sound recorders and reproducers, and parts and
accessories of such articles'' category from the World Customs
Organization. Our aim is to compare the effects of PTAs on trade volumes
against the effects on the unit value of manufacturing products
exported.%
\subsection{Defining North and South}%
\label{subsec:DefiningNorthandSouth}%

%
Defining which countries belong to the ``North'' and ``South''
categories is a key step in order to properly analyse the impact of PTAs
on different bilateral export relationships. However, it is important to
consider that any way in which we categorize countries can be criticised
for not taking into consideration the diverse and heterogenous
characteristics of individual countries within each group. Furthermore,
especially since our focus is to analyse South-South relationships, it
is possible to further disaggregate from the ``South'' group the
emerging economies which are becoming more relevant at the political and
economic world stage and are challenging the hegemony of traditional
developed economies. The level of disaggregation, as well as the level
of attention to heterogenous characteristics among and within groups,
depends on the research question at hand. For the purposes of this
paper, we will not consider such heterogeneity within groups, and just
focus on categorising countries as ``North'' and ``South'', but by no
means does this assumes that countries are homogenous within groups.
This is just a useful distinction to study heterogeneity across PTA
effects.

One intuitive approach could be to categorize countries based on their
income level, but this approach would need to deal with a dynamic list
of groups, as countries change their category through time. Also,
high-income countries include non-industrialized small-nations which we
do not expect to generate significant effects on the industrial
development as well as technology- and skills-upgrading of trade-partner
countries. For such reasons, we have decided to use the same
categorization of countries as Dahi \& Demir (Dahi and Demir 2017) which
takes into consideration characteristics such as incomes, production and
trade structures, factor endowments, and human and institutional
development to construct a list of ``North'' and ``South'' countries,
and also keeps the groups consistent over time. This results in 23
countries categorized as ``North'', and the rest as ``South''. A
detailed list of the countries and their categories can be found in the
Appendix.%
\subsection{Data}%
\label{subsec:Data}%

%
To construct our dataset we have combined PTA data from the ``Design of
International Trade Agreements'' (DESTA) (Dür, Andreas, Leonardo Baccini
and Manfred Elsig 2014) and from the CEPII ``Trade and Production
Database'' (TradeProd) (Thierry Mayer, Gianluca Santoni, Vincent Vicard
2023). The DESTA database aims to aggregate all agreements that have the
potential to liberalise trade, including all agreements notified to the
World Trade Organisation (WTO) and other agreements from a wide range of
sources, covering 880 agreements for 204 countries since 1948 to 2023 in
the last updated version.

Our sample consists of PTAs signed between the years 2000 to 2010 and
the country members to these PTAs, totalling 154 agreements and 143
member countries. For ease of estimation, and to get a sense of
geographical differences, we estimate our models by PTA region for five
main regions: Africa, Americas, Asia, Europe and Intercontinental (We
exclude Oceania {[}11 countries and 1 agreement{]} for lack of
sufficient trade data for our estimations). Each region has the
following samples of agreements and countries: Intercontinental (114
countries and 64 agreements), Europe (42 countries and 41 agreements),
Asia (35 countries and 33 agreements), Americas (15 countries and 13
agreements) and Africa (10 countries and 2 agreements).

For all countries in our sample, we get international trade and domestic
trade flows from the TradeProd database, which has been created
specifically for estimating gravity models and combines trade data from
the UN Commodity Trade Statistics Database (COMTRADE) and production
data from UNIDO Industrial Statistics database (INDSTAT). We also
download export data directly from COMTRADE for all countries in our
sample to construct our export product unit value measurements. For
estimations on trade flows, we use international trade flow data as
reported by importer. In order to measure the appropriate lags for the
effects of each agreement, our period of interest for international flow
data is between 1995 to 2015, and since we are estimating in 5-year
intervals, we get trade flow data for the years 1995, 2000, 2005, 2010
and 2015. Finally, as mentioned before, export product unit values are
constructed using the total value exported per product per year divided
by the net weight exported of said product for said year at the HS
2-digit code level for the 84 and 85 codes for manufacturing products.
As it is not possible to get data for product unit values for domestic
trade, the estimations using this measure as the dependent variable will
suffer from bias as the estimation does not include intra-trade effects.
However, the direction of bias is important as not including intra-trade
measures is expected to bias the effects of PTAs downwards (Yotov et al.
2016), so we use this estimates as illustrative conservative
measurements of the effects of PTAs on the unit value of exported
products.

%
\section{Findings}%
\label{sec:Findings}%
This section presents and describes the results of estimating our gravity models.

%
\subsection{Benchmark Results}%
\label{subsec:BenchmarkResults}%
We begin by briefly discussing the results of our benchmark estimation
by region, contained in Table 1. We immediately see that the average
``cumulative'' effects of PTAs on trade flows after accounting for
phase-in effects (the sum of the current and lagged PTA estimates), is
heterogenous across regions. Only Americas, Europe and Intercontinental
PTAs have statistically significant results, with all coefficients being
positive and generally similar to the results we would expect according
to the literature. The smallest effect, that of Intercontinental PTAs,
has a statistically significant coefficient at the 5\% of 0.203 with a
standard error of (0.106). We interpret this coefficient as
Intercontinental PTAs having an average a partial effect of
(exp(0.203)-1)x100\%. = 22.5\% increase in trade flows. The largest
effect, that of Europe's PTAs, has a statistically significant
coefficient at the 1\% of 0.475 with a standard error of (0.025). We
interpret this coefficient as Europe's PTAs having an average a partial
effect of (exp(0.475)-1)x100\%. = 60.8\% increase in trade flows. On the
other hand, Africa and Asia does not have statistically significant
results, with Asia's coefficient taking a negative value. Interestingly,
Africa's PTA coefficient is highly significant and positive, and PTA Lag
is not significant and negative, while Asia's PTA coefficient is not
significant and positive, and PTA Lag is highly significant and
negative.
%
\begin{table}[htbp]
    \centering
    \caption{Benchmark Model Regional Results}
    \label{tab:benchmark_region_analysis} % This allows you to reference the table in the text with \ref{tab:pta_analysis}
    \begin{adjustbox}{max width=\textwidth}
    \begin{tabular}{l@{\extracolsep{1pt}}ccccc}
    \hline
    & \multicolumn{1}{c}{(1)} & \multicolumn{1}{c}{(2)} & \multicolumn{1}{c}{(3)} & \multicolumn{1}{c}{(4)} & \multicolumn{1}{c}{(5)} \\
    \hline
    \textbf{Variables} &  &  &  &  &  \\
    \hline
     & PPML & PPML & PPML & PPML & PPML \\
     & Africa & Americas & Asia & Europe & Intercontinental \\
    \hline
    PTA & 0.578$^{\ast\ast\ast}$ & 0.287$^{\ast\ast\ast}$ & 0.064 & 0.237$^{\ast\ast\ast}$ & 0.015 \\
    & (0.154) & (0.071) & (0.083) & (0.019) & (0.093) \\

    PTA Lag & -0.278 & 0.146 & -0.167$^{\ast\ast\ast}$ & 0.238$^{\ast\ast\ast}$ & 0.188$^{\ast\ast\ast}$ \\
    & (0.300) & (0.149) & (0.056) & (0.022) & (0.043) \\

    PTA + PTA Lag & 0.301 & 0.433$^{\ast\ast\ast}$ & -0.103 & 0.475$^{\ast\ast\ast}$ & 0.203$^{\ast}$ \\
    & (0.295) & (0.140) & (0.094) & (0.025) & (0.106) \\
    \hline
    Exporter-Year FE & Yes & Yes & Yes & Yes & Yes \\
    Importer-Year FE & Yes & Yes & Yes & Yes & Yes \\
    Country-Pair FE & Yes & Yes & Yes & Yes & Yes \\
    R-Squared & 0.997 & 0.999 & 0.999 & 0.997 & 0.998 \\
    Observations & 5838 & 10997 & 25308 & 28168 & 73930 \\
    \hline
    \multicolumn{6}{l}{\footnotesize{Notes: Robust standard errors clustered at the country-pair in parentheses. Significance levels are indicated as follows: $^{\ast}$p$<$0.1; $^{\ast\ast}$p$<$0.05; $^{\ast\ast\ast}$p$<$0.01.}} \\
    \end{tabular}
    \end{adjustbox}
\end{table}    %
\FloatBarrier

%
\subsection{PTA Heterogeneity Results}%
\label{subsec:PTAHeterogeneityResults}%
The results of our model allowing for heterogenous effects of PTAs is
shown in Table 2 through Table 6. Again, we can observe significant
heterogeneity across regions and PTAs. Africa in Table 2 has no
statistically significant effect for any PTA. Americas in Table 3 has
ten PTAs with statistically significant and positive coefficients, two
with no statistically significant effect, and one PTA with a
statistically significant and negative coefficient. Asia in Table 4 has
eight PTAs with statistically significant and positive coefficients,
nine with no statistically significant effect, and four PTAs with
statistically significant and negative coefficients. Europe in Table 5
has eighteen PTAs with statistically significant and positive
coefficients, nine with no statistically significant effect, and one PTA
with a statistically significant and negative coefficient. And finally,
Intercontinental in Table 6 has twenty-eight PTAs with statistically
significant and positive coefficients, twenty with no statistically
significant effect, and six PTAs with statistically significant and
negative coefficients. Across the regions, 64 out of 118 (54.24\%)
coefficients have significant and positive effects, 42 out of 118
(35.59\%) have no significant effects, and 12 out of 118 (10.17\%) have
significand and negative effects.
%
\begin{table}[htbp]
    \centering
    \caption{TA + TA Lag Coefficients for Africa Region}
    \label{tab:pta_africa}
    \begin{adjustbox}{max width=\textwidth}
    \begin{tabular}{lcc}
    \hline
    \textbf{Statistically Insignificant} &  &  \\
    \hline
    TA ID & Estimate & SE \\
    \hline
    670 & 0.326 & (0.410) \\
    787 & 0.304 & (0.233) \\
    \hline
    Exporter-Year FE & Yes \\
    Importer-Year FE & Yes \\
    Country-Pair FE & Yes \\
    R-Squared & 0.997 \\
    Observations & 5838 \\
    \hline
    \multicolumn{3}{l}{\footnotesize{Notes: Robust standard errors clustered at the country-pair level in parentheses.}} \\
    \multicolumn{3}{l}{\footnotesize{Significance levels are indicated as follows: $^{\ast}$p$<$0.1; $^{\ast\ast}$p$<$0.05; $^{\ast\ast\ast}$p$<$0.01.}} \\
    \end{tabular}
    \end{adjustbox}
\end{table}
%
\begin{table}[htbp]
    \centering
    \caption{TA + TA Lag Coefficients for Americas Region}
    \label{tab:pta_americas}
    \begin{adjustbox}{max width=\textwidth}
    \begin{tabular}{lcc}
    \hline
    \textbf{Positive and Statistically Significant} &  &  \\
    \hline
    TA ID & Estimate & SE \\
    \hline
    213 & 1.342$^{\ast\ast\ast}$ & (0.434) \\
    218 & 0.879$^{\ast\ast\ast}$ & (0.173) \\
    239 & 0.571$^{\ast\ast\ast}$ & (0.173) \\
    616 & 0.488$^{\ast\ast\ast}$ & (0.044) \\
    168 & 0.410$^{\ast\ast\ast}$ & (0.113) \\
    163 & 0.342$^{\ast\ast\ast}$ & (0.096) \\
    141 & 0.265$^{\ast\ast\ast}$ & (0.024) \\
    716 & 0.732$^{\ast\ast}$ & (0.358) \\
    201 & 0.545$^{\ast\ast}$ & (0.265) \\
    612 & 0.515$^{\ast\ast}$ & (0.251) \\
    \hline
    \textbf{Statistically Insignificant} &  &  \\
    \hline
    TA ID & Estimate & SE \\
    \hline
    185 & 0.291 & (0.376) \\
    645 & 0.117 & (0.141) \\
    \hline
    \textbf{Negative and Statistically Significant} &  &  \\
    \hline
    TA ID & Estimate & SE \\
    \hline
    188 & -0.774$^{\ast\ast\ast}$ & (0.144) \\
    \hline
    Exporter-Year FE & Yes \\
    Importer-Year FE & Yes \\
    Country-Pair FE & Yes \\
    R-Squared & 0.999 \\
    Observations & 10997 \\
    \hline
    \multicolumn{3}{l}{\footnotesize{Notes: Robust standard errors clustered at the country-pair level in parentheses.}} \\
    \multicolumn{3}{l}{\footnotesize{Significance levels are indicated as follows: $^{\ast}$p$<$0.1; $^{\ast\ast}$p$<$0.05; $^{\ast\ast\ast}$p$<$0.01.}} \\
    \end{tabular}
    \end{adjustbox}
\end{table}
%
\begin{table}[htbp]
    \centering
    \caption{PTA + PTA Lag Coefficients for Asia Region}
    \label{tab:pta_asia}
    \begin{adjustbox}{max width=\textwidth}
    \begin{tabular}{lcc}
    \hline
    \textbf{Positive and Statistically Significant} &  &  \\
    \hline
    PTA ID & Estimate & SE \\
    \hline
    683 & 1.080$^{\ast\ast\ast}$ & (0.237) \\
    70  & 0.472$^{\ast\ast\ast}$ & (0.150) \\
    100 & 0.376$^{\ast\ast\ast}$ & (0.105) \\
    67  & 0.342$^{\ast\ast\ast}$ & (0.125) \\
    675 & 1.360$^{\ast\ast}$ & (0.655) \\
    475 & 0.636$^{\ast\ast}$ & (0.298) \\
    598 & 0.166$^{\ast\ast}$ & (0.083) \\
    474 & 0.419$^{\ast}$ & (0.243) \\
    \hline
    \textbf{Statistically Insignificant} &  &  \\
    \hline
    PTA ID & Estimate & SE \\
    \hline
    72  & 0.254 & (0.178) \\
    116 & 0.256 & (0.703) \\
    492 & 0.041 & (0.180) \\
    640 & 0.183 & (0.217) \\
    223 & -0.014 & (0.203) \\
    71  & -0.138 & (0.091) \\
    456 & -0.209 & (0.165) \\
    534 & -0.165 & (0.370) \\
    667 & -0.049 & (0.241) \\
    \hline
    \textbf{Negative and Statistically Significant} &  &  \\
    \hline
    PTA ID & Estimate & SE \\
    \hline
    221 & -2.955$^{\ast\ast\ast}$ & (0.727) \\
    220 & -1.215$^{\ast\ast\ast}$ & (0.093) \\
    599 & -0.967$^{\ast\ast\ast}$ & (0.191) \\
    1   & -0.732$^{\ast\ast}$ & (0.359) \\
    \hline
    Exporter-Year FE & Yes \\
    Importer-Year FE & Yes \\
    Country-Pair FE & Yes \\
    R-Squared & 0.999 \\
    Observations & 25308 \\
    \hline
    \multicolumn{3}{l}{\footnotesize{Notes: Robust standard errors clustered at the country-pair level in parentheses.}} \\
    \multicolumn{3}{l}{\footnotesize{Significance levels are indicated as follows: $^{\ast}$p$<$0.1; $^{\ast\ast}$p$<$0.05; $^{\ast\ast\ast}$p$<$0.01.}} \\
    \end{tabular}
    \end{adjustbox}
\end{table}
%
\begin{table}[htbp]
    \centering
    \caption{PTA + PTA Lag Coefficients for Europe Region}
    \label{tab:pta_europe}
    \begin{adjustbox}{max width=\textwidth}
    \begin{tabular}{lcc}
    \hline
    \textbf{Positive and Statistically Significant} &  &  \\
    \hline
    PTA ID & Estimate & SE \\
    \hline
    5   & 3.812$^{\ast\ast\ast}$ & (0.278) \\
    128 & 2.712$^{\ast\ast\ast}$ & (0.211) \\
    13  & 2.256$^{\ast\ast\ast}$ & (0.262) \\
    132 & 2.241$^{\ast\ast\ast}$ & (0.252) \\
    192 & 1.107$^{\ast\ast\ast}$ & (0.163) \\
    7   & 1.153$^{\ast\ast\ast}$ & (0.272) \\
    328 & 0.671$^{\ast\ast\ast}$ & (0.175) \\
    8   & 0.667$^{\ast\ast\ast}$ & (0.161) \\
    621 & 0.618$^{\ast\ast\ast}$ & (0.186) \\
    135 & 0.615$^{\ast\ast\ast}$ & (0.217) \\
    254 & 0.565$^{\ast\ast\ast}$ & (0.084) \\
    394 & 0.745$^{\ast\ast\ast}$ & (0.202) \\
    335 & 0.472$^{\ast\ast\ast}$ & (0.025) \\
    9   & 0.580$^{\ast\ast}$ & (0.285) \\
    11  & 0.656$^{\ast\ast}$ & (0.307) \\
    131 & 0.615$^{\ast\ast}$ & (0.281) \\
    129 & 0.553$^{\ast\ast\ast}$ & (0.206) \\
    \hline
    \textbf{Statistically Insignificant} &  &  \\
    \hline
    PTA ID & Estimate & SE \\
    \hline
    6   & 0.355 & (0.358) \\
    150 & 0.247 & (0.687) \\
    153 & 0.614 & (0.633) \\
    154 & 0.592 & (0.409) \\
    255 & 0.167 & (0.237) \\
    389 & 0.412 & (0.323) \\
    331 & 0.142 & (0.201) \\
    594 & 0.474$^{\ast}$ & (0.251) \\
    12  & -0.246 & (1.208) \\
    156 & -0.441 & (0.445) \\
    \hline
    \textbf{Negative and Statistically Significant} &  &  \\
    \hline
    PTA ID & Estimate & SE \\
    \hline
    133 & -0.772$^{\ast\ast\ast}$ & (0.248) \\
    \hline
    \multicolumn{3}{l}{\footnotesize{Notes: Robust standard errors clustered at the country-pair level in parentheses.}} \\
    \multicolumn{3}{l}{\footnotesize{Significance levels are indicated as follows: $^{\ast}$p$<$0.1; $^{\ast\ast}$p$<$0.05; $^{\ast\ast\ast}$p$<$0.01.}} \\
    \end{tabular}
    \end{adjustbox} 
\end{table}
%
\begin{center}
\small
\begin{longtable}{lcc}
    \caption{PTA+PTALag Coefficients for Intercontinental} \label{tab:pta_intercontinental} \\
    
    \hline
    \textbf{Positive and Statistically Significant} &  &  \\
    \hline
    PTA ID & Estimate & SE \\
    \hline
    \endfirsthead
    
    \multicolumn{3}{c}{{\bfseries \tablename\ \thetable{} -- continued from previous page}} \\
    \hline
    PTA ID & Estimate & SE \\
    \hline
    \endhead
    
    \hline \multicolumn{3}{r}{{Continued on next page}} \\ \hline
    \endfoot
    
    \hline
    \endlastfoot
    
    627 & 2.372$^{\ast\ast\ast}$ & (0.345) \\
    415 & 1.853$^{\ast\ast\ast}$ & (0.201) \\
    206 & 1.539$^{\ast\ast\ast}$ & (0.180) \\
    75  & 1.366$^{\ast\ast\ast}$ & (0.493) \\
    263 & 1.426$^{\ast\ast\ast}$ & (0.115) \\
    4   & 1.254$^{\ast\ast\ast}$ & (0.268) \\
    626 & 1.099$^{\ast\ast\ast}$ & (0.121) \\
    657 & 0.705$^{\ast\ast\ast}$ & (0.082) \\
    637 & 0.667$^{\ast\ast\ast}$ & (0.102) \\
    202 & 0.658$^{\ast\ast\ast}$ & (0.123) \\
    208 & 0.763$^{\ast\ast\ast}$ & (0.129) \\
    136 & 0.744$^{\ast\ast\ast}$ & (0.185) \\
    490 & 0.843$^{\ast\ast\ast}$ & (0.181) \\
    17  & 0.811$^{\ast\ast\ast}$ & (0.242) \\
    466 & 0.710$^{\ast\ast\ast}$ & (0.147) \\
    304 & 0.770$^{\ast\ast\ast}$ & (0.120) \\
    628 & 0.484$^{\ast\ast\ast}$ & (0.142) \\
    207 & 0.516$^{\ast\ast\ast}$ & (0.114) \\
    518 & 0.627$^{\ast\ast\ast}$ & (0.135) \\
    330 & 0.314$^{\ast\ast\ast}$ & (0.086) \\
    164 & 0.288$^{\ast\ast\ast}$ & (0.073) \\
    96 & 0.271$^{\ast\ast\ast}$ & (0.055) \\
    181 & 0.392$^{\ast\ast}$ & (0.178) \\
    624 & 0.388$^{\ast\ast}$ & (0.163) \\
    521 & 0.101$^{\ast\ast}$ & (0.045) \\
    384  & 0.645$^{\ast}$ & (0.355) \\
    15  & 0.313$^{\ast}$ & (0.179) \\
    227 & 0.348$^{\ast}$ & (0.186) \\
    \hline
    \textbf{Statistically Insignificant} &  &  \\
    \hline
    PTA ID & Estimate & SE \\
    \hline
    641 & 2.028 & (1.255) \\
    543 & 1.090 & (0.707) \\
    509 & 0.210 & (0.216) \\
    252 & 0.192 & (0.357) \\
    508 & 0.140 & (0.122) \\
    376 & 0.172 & (0.228) \\
    416 & 0.424 & (0.295) \\
    401 & 0.407 & (0.288) \\
    152 & 0.110 & (0.266) \\
    242 & 0.050 & (0.294) \\
    390 & 0.0471 & (0.181) \\
    396 & 0.019 & (0.379) \\
    205 & 0.0012 & (0.178) \\
    602 & -0.076 & (0.918) \\
    383 & -0.202 & (0.152) \\
    386 & -0.092 & (0.168) \\
    84  & -0.059 & (0.120) \\
    979  & -0.126 & (0.294) \\
    644  & -0.189 & (0.122) \\
    658  & -0.303 & (0.349) \\
    \hline
    \textbf{Negative and Statistically Significant} &  &  \\
    \hline
    PTA ID & Estimate & SE \\
    \hline
    399 & -0.473$^{\ast\ast\ast}$ & (0.127) \\
    104 & -0.338$^{\ast\ast\ast}$ & (0.112) \\
    677 & -1.366$^{\ast\ast\ast}$ & (0.385) \\
    679 & -1.429$^{\ast\ast\ast}$ & (0.430) \\
    323 & -0.338$^{\ast\ast}$ & (0.138) \\
    512 & -0.458$^{\ast}$ & (0.266) \\
    \hline
    Exporter-Year FE & Yes \\
    Importer-Year FE & Yes \\
    Country-Pair FE & Yes \\
    R-Squared & 0.998 \\
    Observations & 73930 \\
    \hline
    \multicolumn{3}{l}{\footnotesize{Notes: Robust standard errors clustered at the country-pair level in parentheses.}} \\
    \multicolumn{3}{l}{\footnotesize{Significance levels are indicated as follows: $^{\ast}$p$<$0.1; $^{\ast\ast}$p$<$0.05; $^{\ast\ast\ast}$p$<$0.01.}} \\
\end{longtable}
\end{center}


% \begin{table}[htbp]
%     \centering
%     \caption{PTA + PTA Lag Coefficients for Intercontinental Region}
%     \label{tab:pta_intercontinental}
%     \begin{minipage}[t]{0.48\textwidth} % Adjust the width as needed
%         \centering
%         \begin{adjustbox}{max width=\textwidth}
%         \begin{tabular}{lcc}
%         \hline
%         \textbf{Positive and Statistically Significant} &  &  \\
%         \hline
%         PTA ID & Estimate & SE \\
%         \hline
%         627 & 2.372$^{\ast\ast\ast}$ & (0.345) \\
%         415 & 1.853$^{\ast\ast\ast}$ & (0.201) \\
%         206 & 1.539$^{\ast\ast\ast}$ & (0.180) \\
%         75  & 1.366$^{\ast\ast\ast}$ & (0.493) \\
%         263 & 1.426$^{\ast\ast\ast}$ & (0.115) \\
%         4   & 1.254$^{\ast\ast\ast}$ & (0.268) \\
%         626 & 1.099$^{\ast\ast\ast}$ & (0.121) \\
%         657 & 0.705$^{\ast\ast\ast}$ & (0.082) \\
%         637 & 0.667$^{\ast\ast\ast}$ & (0.102) \\
%         202 & 0.658$^{\ast\ast\ast}$ & (0.123) \\
%         208 & 0.763$^{\ast\ast\ast}$ & (0.129) \\
%         136 & 0.744$^{\ast\ast\ast}$ & (0.185) \\
%         490 & 0.843$^{\ast\ast\ast}$ & (0.181) \\
%         17  & 0.811$^{\ast\ast\ast}$ & (0.242) \\
%         466 & 0.710$^{\ast\ast\ast}$ & (0.147) \\
%         304 & 0.770$^{\ast\ast\ast}$ & (0.120) \\
%         628 & 0.484$^{\ast\ast\ast}$ & (0.142) \\
%         207 & 0.516$^{\ast\ast\ast}$ & (0.114) \\
%         518 & 0.627$^{\ast\ast\ast}$ & (0.135) \\
%         330 & 0.314$^{\ast\ast\ast}$ & (0.086) \\
%         164 & 0.288$^{\ast\ast\ast}$ & (0.073) \\
%         96 & 0.271$^{\ast\ast\ast}$ & (0.055) \\
%         181 & 0.392$^{\ast\ast}$ & (0.178) \\
%         624 & 0.388$^{\ast\ast}$ & (0.163) \\
%         521 & 0.101$^{\ast\ast}$ & (0.045) \\
%         384  & 0.645$^{\ast}$ & (0.355) \\
%         15  & 0.313$^{\ast}$ & (0.179) \\
%         227 & 0.348$^{\ast}$ & (0.186) \\
%         \hline
%         \end{tabular}
%         \end{adjustbox}
%     \end{minipage}%
%     \hfill
%     \begin{minipage}[t]{0.48\textwidth} % Adjust the width as needed
%         \centering
%         \begin{adjustbox}{max width=\textwidth}
%         \begin{tabular}{lcc}
%         \hline
%         \textbf{Statistically Insignificant} &  &  \\
%         \hline
%         PTA ID & Estimate & SE \\
%         \hline
%         641 & 2.028 & (1.255) \\
%         543 & 1.090 & (0.707) \\
%         509 & 0.210 & (0.216) \\
%         252 & 0.192 & (0.357) \\
%         508 & 0.140 & (0.122) \\
%         376 & 0.172 & (0.228) \\
%         416 & 0.424 & (0.295) \\
%         401 & 0.407 & (0.288) \\
%         152 & 0.110 & (0.266) \\
%         242 & 0.050 & (0.294) \\
%         390 & 0.0471 & (0.181) \\
%         396 & 0.019 & (0.379) \\
%         205 & 0.0012 & (0.178) \\
%         602 & -0.076 & (0.918) \\
%         383 & -0.202 & (0.152) \\
%         386 & -0.092 & (0.168) \\
%         84  & -0.059 & (0.120) \\
%         979  & -0.126 & (0.294) \\
%         644  & -0.189 & (0.122) \\
%         658  & -0.303 & (0.349) \\
%         \hline
%         \textbf{Negative and Statistically Significant} &  &  \\
%         \hline
%         PTA ID & Estimate & SE \\
%         \hline
%         399 & -0.473$^{\ast\ast\ast}$ & (0.127) \\
%         104 & -0.338$^{\ast\ast\ast}$ & (0.112) \\
%         677 & -1.366$^{\ast\ast\ast}$ & (0.385) \\
%         679 & -1.429$^{\ast\ast\ast}$ & (0.430) \\
%         323 & -0.338$^{\ast\ast}$ & (0.138) \\
%         512 & -0.458$^{\ast}$ & (0.266) \\
%         \hline
%         \end{tabular}
%         \end{adjustbox}
%     \end{minipage}
%     \vspace{0.5cm} % Optional: Add space between the tables and notes
%     \begin{adjustbox}{max width=\textwidth}
%     \begin{tabular}{l}
%     \hline
%     Exporter-Year FE & Yes \\
%     Importer-Year FE & Yes \\
%     Country-Pair FE & Yes \\
%     R-Squared & 0.998 \\
%     Observations & 73930 \\
%     \hline
%     \multicolumn{3}{l}{\footnotesize{Notes: Robust standard errors clustered at the country-pair level in parentheses.}} \\
%     \multicolumn{3}{l}{\footnotesize{Significance levels are indicated as follows: $^{\ast}$p$<$0.1; $^{\ast\ast}$p$<$0.05; $^{\ast\ast\ast}$p$<$0.01.}} \\
%     \end{tabular}
%     \end{adjustbox}
% \end{table}%
\FloatBarrier

%
\subsection{North{-}North, North{-}South and South{-}South PTAs}%
\label{subsec:North{-}North,North{-}SouthandSouth{-}SouthPTAs}%
\subsubsection{North{-}South Benchmark Results}%
\label{ssubsec:North{-}SouthBenchmarkResults}%
We present the results of our extended models allowing us to capture the
differentiated effects of PTAs on bilateral exports depending on whether
the pair country are two ``North'' countries (NN), a ``North'' and a
``South'' country (NS), or two ``South'' countries (SS).

The results of the extended benchmark estimation by region, contained in
Table 7 again show heterogenous results across regions. It is
interesting to note that by disaggregating the PTA effects, in the case
of Americas and Europe, both of which had significant and positive
coefficients in the benchmark estimation, now again have significant and
positive coefficients for both NS PTA + Lag and SS PTA + Lag , but the
effects are larger in both cases for the SS PTA + Lag coefficient. Asia
now has a slightly significant and negative coefficient for NS PTA + Lag
while the coefficient for SS PTA + Lag remains not significant.
Intercontinental have significant and positive effects of NS Lag and SS
Lag, but NS PTA + Lag and SS PTA + Lag are both not significant now.
Africa's coefficients remain not significant, and it is the only region
with only South-South PTAs.
%
\begin{table}[htbp]
    \centering
    \caption{Regional Results by PTA Type}
    \label{tab:pta_types}
    \begin{adjustbox}{max width=\textwidth}
    \begin{tabular}{lccccc}
    \hline
     & \multicolumn{1}{c}{Africa} & \multicolumn{1}{c}{Americas} & \multicolumn{1}{c}{Asia} & \multicolumn{1}{c}{Europe} & \multicolumn{1}{c}{Intercontinental} \\
    \hline
    \textbf{Variables} &  &  &  &  &  \\
    \hline
    NN PTA &  &  &  & 0.207$^{\ast\ast\ast}$ & 0.013 \\
     &  &  &  & (0.021) & (0.072) \\
    NN PTA Lag &  &  &  & 0.192$^{\ast\ast\ast}$ & 0.016 \\
     &  &  &  & (0.023) & (0.073) \\
    NN PTA + NN PTA Lag &  &  &  & 0.399$^{\ast\ast\ast}$ & 0.029 \\
     &  &  &  & (0.026) & (0.102) \\
    \hline
    NS PTA &  & 0.199$^{\ast\ast\ast}$ & -0.089 & 0.374$^{\ast\ast\ast}$ & 0.013 \\
     &  & (0.069) & (0.089) & (0.041) & (0.144) \\
    NS PTA Lag &  & 0.234 & -0.067 & 0.349$^{\ast\ast\ast}$ & 0.231$^{\ast\ast\ast}$ \\
     &  & (0.190) & (0.060) & (0.041) & (0.061) \\
    NS PTA + NS PTA Lag &  & 0.434$^{\ast\ast}$ & -0.156$^{\ast}$ & 0.723$^{\ast\ast\ast}$ & 0.244 \\
     &  & (0.200) & (0.090) & (0.046) & (0.156) \\
    \hline
    SS PTA & 0.578$^{\ast\ast\ast}$ & 0.476$^{\ast\ast\ast}$ & 0.153 & 0.530$^{\ast\ast\ast}$ & 0.004 \\
     & (0.154) & (0.139) & (0.117) & (0.107) & (0.121) \\
    SS PTA Lag & -0.278 & -0.023 & -0.208$^{\ast\ast\ast}$ & 0.575$^{\ast\ast\ast}$ & 0.204$^{\ast\ast\ast}$ \\
     & (0.300) & (0.133) & (0.063) & (0.119) & (0.073) \\
    SS PTA + SS PTA Lag & 0.301 & 0.453$^{\ast\ast\ast}$ & -0.055 & 1.105$^{\ast\ast\ast}$ & 0.208 \\
     & (0.295) & (0.112) & (0.130) & (0.092) & (0.128) \\
    \hline
    Exporter-Year FE & Yes & Yes & Yes & Yes & Yes \\
    Importer-Year FE & Yes & Yes & Yes & Yes & Yes \\
    Country-Pair FE & Yes & Yes & Yes & Yes & Yes \\
    R-Squared & 0.997 & 0.999 & 0.999 & 0.997 & 0.998 \\
    Observations & 5838 & 10997 & 25308 & 28168 & 73930 \\
    \hline
    \multicolumn{6}{l}{\footnotesize{Notes: Robust standard errors clustered at the country-pair level in parentheses.}} \\
    \multicolumn{6}{l}{\footnotesize{Significance levels are indicated as follows: $^{\ast}$p$<$0.1; $^{\ast\ast}$p$<$0.05; $^{\ast\ast\ast}$p$<$0.01.}} \\
    \end{tabular}
    \end{adjustbox}
\end{table}
%
\FloatBarrier

%
\subsubsection{North{-}South PTA Heterogeneity Results}%
\label{ssubsec:North{-}SouthPTAHeterogeneityResults}%
The results of our extended model allowing for heterogenous effects of
PTAs is shown in Table 8 through Table 12. Africa in table 8 only has
effects for South-South PTAs and again has no statistically significant
effect for any PTA. Americas in Table 9 has five PTAs with North-South
estimates, one of which has statistically significant and negative
effects for NS PTA + Lag and statistically significant and positive
effects for SS PTA + Lag. Of the remaining four, none have estimates for
SS PTA + Lag, three are statistically significant and positive, and one
is not statistically significant. It has eight PTAs with South-South
estimates, seven of which have statistically significant and positive
effects, while one does not have statistically significant effects.
Americas does not have any coefficients for North-North. Asia in Table
10 has two PTAs with North-South estimates, one of which is
statistically significant and positive, while the other is not
statistically significant. It has nineteen PTAs with South-South
estimates, seven of which have statistically significant and positive
effects, four have statistically significant and negative coefficients,
and eight does not have statistically significant effects. Asia does not
have any coefficients for North-North. Europe in Table 11 has eight PTA
North-South estimates, five of which are statistically significant and
positive, and the others are not statistically significant. One of the
five agreements with statistically significant and positive coefficients
for NS PTA + Lag also has a statistically significant and positive
coefficient for SS PTA + Lag. None of the other agreements with a NS
coefficient have statistically significant coefficients for SS. It has
nineteen South-South estimates, thirteen are statistically significant
and positive, one is statistically significant and negative, and five
are not significant. Finally, the region has one agreement with a
North-North estimate, which also has a North-South and a South-South
estimate and they are all statistically significant and positive.
Intercontinental in Table 12 has thirty PTA North-South estimates, of
which twelve are statistically significant and positive, fifteen are not
statistically significant, and three are statistically significant and
negative for NS PTA + Lag. None of these PTAs also have coefficients for
SS PTA + Lag of which five are statistically significant and positive,
three are not statistically significant, and one is statistically
significant and negative. It has twenty-one estimates for South-South,
of which fourteen are statistically significant and positive, five are
not statistically significant, and two are statistically significant and
negative. It has three agreements with North-North estimates, two
statistically significant and positive, and one are not statistically
significant. Across the regions and PTAs, 23 out of 47 (48.94\%) NS
coefficients have significant and positive effects, 20 out of 47
(42.55\%) have no significant effects, and 4 out of 47 (8.51\%) have
significand and negative effects; 49 out of 84 (58.33\%) SS coefficients
have significant and positive effects, 27 out of 84 (32.14\%) have no
significant effects, and 8 out of 84 (9.52\%) have significand and
negative effects; and, 3 out of 4 (75\%) NN coefficients have
significant and positive effects, 1 out of 4 (25\%) have no significant
effects, and none have significand and negative effects.
%
\begin{table}[htbp]
    \centering
    \caption{Africa TA + TA Lag Coefficients by Type}
    \label{tab:africa_pta}
    \begin{adjustbox}{max width=\textwidth}
    \begin{tabular}{lccc}
    \hline
    \textbf{TA ID} & \textbf{NS TA+Lag} & \textbf{SS TA+Lag} & \textbf{NN TA+Lag} \\
    \hline
    \textbf{NS and SS (or only NS)} &  &  &  \\
    \hline
    \multicolumn{4}{c}{No agreements in this category} \\
    \hline
    \textbf{Only SS} &  &  &  \\
    \hline
    670 &  & 0.326 &  \\
     &  & (0.410) &  \\
    787 &  & 0.304 &  \\
     &  & (0.233) &  \\
    \hline
    \textbf{Agreements with NN and NS} &  &  &  \\
    \hline
    \multicolumn{4}{c}{No agreements in this category} \\
    \hline
    Exporter-Year FE & Yes \\
    Importer-Year FE & Yes \\
    Country-Pair FE & Yes \\
    R-Squared & 0.997 \\
    Observations & 5838 \\
    \hline
    \multicolumn{4}{l}{\footnotesize{Notes: Robust standard errors clustered at the country-pair level in parentheses.}} \\
    \multicolumn{4}{l}{\footnotesize{Significance levels are indicated as follows: $^{\ast}$p$<$0.1; $^{\ast\ast}$p$<$0.05; $^{\ast\ast\ast}$p$<$0.01.}} \\
    \end{tabular}
    \end{adjustbox}
\end{table}
%
\begin{table}[htbp]
    \centering
    \caption{Americas PTA + PTA Lag Coefficients by Type}
    \label{tab:americas_pta}
    \begin{adjustbox}{max width=\textwidth}
    \begin{tabular}{lccc}
    \hline
    \textbf{PTA ID} & \textbf{NS PTA + Lag} & \textbf{SS PTA + Lag} & \textbf{NN PTA + Lag} \\
    \hline
    \textbf{Agreements with NS and SS (or only NS)} &  &  &  \\
    \hline
    163 & 0.346$^{\ast\ast\ast}$ &  &  \\
    168 & 0.410$^{\ast\ast\ast}$ &  &  \\
    188 & -0.811$^{\ast\ast\ast}$ & 0.685$^{\ast\ast}$ &  \\
    218 & 0.879$^{\ast\ast\ast}$ &  &  \\
    645 & 0.117 &  &  \\
    \hline
    \textbf{Agreements with only SS} &  &  &  \\
    \hline
    141 &  & 0.265$^{\ast\ast\ast}$ &  \\
    201 &  & 0.545$^{\ast\ast}$ &  \\
    213 &  & 1.342$^{\ast\ast\ast}$ &  \\
    239 &  & 0.572$^{\ast\ast\ast}$ &  \\
    612 &  & 0.517$^{\ast\ast}$ &  \\
    616 &  & 0.488$^{\ast\ast\ast}$ &  \\
    716 &  & 0.732$^{\ast\ast}$ &  \\
    \hline
    \textbf{Agreements with NN and NS} &  &  &  \\
    \hline
    \multicolumn{4}{c}{No agreements in this category} \\
    \hline
    \multicolumn{4}{l}{\footnotesize{Notes: Robust standard errors clustered at the country-pair level in parentheses.}} \\
    \multicolumn{4}{l}{\footnotesize{Significance levels are indicated as follows: $^{\ast}$p$<$0.1; $^{\ast\ast}$p$<$0.05; $^{\ast\ast\ast}$p$<$0.01.}} \\
    \end{tabular}
    \end{adjustbox}
\end{table}
%
\begin{table}[htbp]
    \centering
    \caption{Asia PTA + PTA Lag Coefficients by Type}
    \label{tab:asia_pta}
    \begin{adjustbox}{max width=\textwidth}
    \begin{tabular}{lccc}
    \hline
    \textbf{PTA ID} & \textbf{NS PTA + Lag} & \textbf{SS PTA + Lag} & \textbf{NN PTA + Lag} \\
    \hline
    \textbf{Agreements with NS and SS (or only NS)} &  &  &  \\
    \hline
    1 &  & -0.732$^{\ast\ast}$ &  \\
    67 &  & 0.342$^{\ast\ast\ast}$ &  \\
    70 &  & 0.472$^{\ast\ast\ast}$ &  \\
    71 & -0.138 &  &  \\
    72 & 0.254 &  &  \\
    220 &  & -1.215$^{\ast\ast\ast}$ &  \\
    221 &  & -2.955$^{\ast\ast\ast}$ &  \\
    \hline
    \textbf{Agreements with only SS} &  &  &  \\
    \hline
    100 &  & 0.376$^{\ast\ast\ast}$ &  \\
    475 &  & 0.636$^{\ast\ast}$ &  \\
    640 &  & 0.183 &  \\
    675 &  & 1.360$^{\ast\ast}$ &  \\
    683 &  & 1.080$^{\ast\ast\ast}$ &  \\
    \hline
    \textbf{Agreements with NN and NS} &  &  &  \\
    \hline
    \multicolumn{4}{c}{No agreements in this category} \\
    \hline
    \multicolumn{4}{l}{\footnotesize{Notes: Robust standard errors clustered at the country-pair level in parentheses.}} \\
    \multicolumn{4}{l}{\footnotesize{Significance levels are indicated as follows: $^{\ast}$p$<$0.1; $^{\ast\ast}$p$<$0.05; $^{\ast\ast\ast}$p$<$0.01.}} \\
    \end{tabular}
    \end{adjustbox}
\end{table}
%
\FloatBarrier%
\begin{table}[htbp]
    \centering
    \caption{Europe PTA + PTA Lag Coefficients by Type}
    \label{tab:europe_pta}
    \begin{adjustbox}{max width=\textwidth}
    \begin{tabular}{lccc}
    \hline
    \textbf{PTA ID} & \textbf{NS PTA + Lag} & \textbf{SS PTA + Lag} & \textbf{NN PTA + Lag} \\
    \hline
    \textbf{Agreements with NS and SS (or only NS)} &  &  &  \\
    \hline
    8   & 0.663$^{\ast\ast\ast}$ & 0.783$^{\ast\ast\ast}$ &  \\
    9   & 0.581$^{\ast\ast}$ &  &  \\
    254 & 0.568$^{\ast\ast\ast}$ & 0.323 &  \\
    328 & 0.738$^{\ast\ast\ast}$ & 0.354 &  \\
    335 & 0.727$^{\ast\ast\ast}$ & 1.099$^{\ast\ast\ast}$ & 0.399$^{\ast\ast\ast}$ \\
    \hline
    \textbf{Agreements with only SS} &  &  &  \\
    \hline
    5   &  & 3.811$^{\ast\ast\ast}$ &  \\
    7   &  & 1.153$^{\ast\ast\ast}$ &  \\
    11  &  & 0.663$^{\ast\ast}$ &  \\
    13  &  & 2.303$^{\ast\ast\ast}$ &  \\
    128 &  & 2.773$^{\ast\ast\ast}$ &  \\
    129 &  & 0.556$^{\ast\ast\ast}$ &  \\
    132 &  & 2.241$^{\ast\ast\ast}$ &  \\
    135 &  & 0.696$^{\ast\ast\ast}$ &  \\
    150 &  & 0.444 &  \\
    153 &  & 0.817 &  \\
    192 &  & 1.199$^{\ast\ast\ast}$ &  \\
    621 &  & 0.614$^{\ast\ast\ast}$ &  \\
    394 & 0.747$^{\ast\ast\ast}$ &  &  \\
    \hline
    \textbf{Agreements with NN and NS} &  &  &  \\
    \hline
    335 & 0.727$^{\ast\ast\ast}$ & 1.099$^{\ast\ast\ast}$ & 0.399$^{\ast\ast\ast}$ \\
    6   & 0.411 &  &  \\
    132 & 0.738$^{\ast\ast\ast}$ & 2.241$^{\ast\ast\ast}$ &  \\
    \hline
    \multicolumn{4}{l}{\footnotesize{Notes: Robust standard errors clustered at the country-pair level in parentheses.}} \\
    \multicolumn{4}{l}{\footnotesize{Significance levels are indicated as follows: $^{\ast}$p$<$0.1; $^{\ast\ast}$p$<$0.05; $^{\ast\ast\ast}$p$<$0.01.}} \\
    \end{tabular}
    \end{adjustbox}
\end{table}
%
\FloatBarrier%
\begin{table}[htbp]
    \centering
    \caption{Intercontinental PTA + PTA Lag Coefficients by Type}
    \label{tab:intercontinental_pta}
    \begin{adjustbox}{max width=\textwidth}
    \begin{tabular}{lccc}
    \hline
    \textbf{PTA ID} & \textbf{NS PTA + Lag} & \textbf{SS PTA + Lag} & \textbf{NN PTA + Lag} \\
    \hline
    \textbf{Agreements with NS and SS (or only NS)} &  &  &  \\
    \hline
    17  & 0.800$^{\ast\ast\ast}$ & 1.055$^{\ast\ast\ast}$ &  \\
    75  & 1.366$^{\ast\ast\ast}$ &  &  \\
    96  & 0.271$^{\ast\ast\ast}$ &  &  \\
    202 & 0.660$^{\ast\ast\ast}$ & 0.612$^{\ast\ast}$ &  \\
    207 & 0.516$^{\ast\ast\ast}$ &  &  \\
    208 &  & 0.763$^{\ast\ast\ast}$ &  \\
    323 & -0.335$^{\ast\ast}$ & -0.360 &  \\
    330 & 0.286$^{\ast\ast\ast}$ & 0.662$^{\ast\ast\ast}$ &  \\
    512 & -0.458$^{\ast}$ &  &  \\
    518 & 0.627$^{\ast\ast\ast}$ &  &  \\
    624 &  & 0.384$^{\ast\ast}$ &  \\
    626 &  & 1.099$^{\ast\ast\ast}$ &  \\
    627 &  & 2.372$^{\ast\ast\ast}$ &  \\
    628 & 0.484$^{\ast\ast\ast}$ &  &  \\
    637 & 0.667$^{\ast\ast\ast}$ &  &  \\
    657 &  & 0.705$^{\ast\ast\ast}$ &  \\
    679 & 0.546$^{\ast}$ & -1.636$^{\ast\ast\ast}$ &  \\
    \hline
    \textbf{Agreements with only SS} &  &  &  \\
    \hline
    4   &  & 1.255$^{\ast\ast\ast}$ &  \\
    104 &  & -0.338$^{\ast\ast\ast}$ &  \\
    136 &  & 0.744$^{\ast\ast\ast}$ &  \\
    164 &  & 0.288$^{\ast\ast\ast}$ &  \\
    181 &  & 1.288$^{\ast\ast\ast}$ &  \\
    206 &  & 1.540$^{\ast\ast\ast}$ &  \\
    263 &  & 1.426$^{\ast\ast\ast}$ &  \\
    304 & 0.787$^{\ast\ast\ast}$ & 0.591$^{\ast\ast}$ &  \\
    466 &  & 0.710$^{\ast\ast\ast}$ &  \\
    490 &  & 0.843$^{\ast\ast\ast}$ &  \\
    521 & 0.102$^{\ast\ast}$ &  &  \\
    543 & 1.090 &  &  \\
    677 &  & -1.366$^{\ast\ast\ast}$ &  \\
    679 & -1.636$^{\ast\ast\ast}$ &  &  \\
    \hline
    \textbf{Agreements with NN and NS} &  &  &  \\
    \hline
    84  &  & -0.059 & -0.059 \\
    15  &  & 0.313 &  \\
    17  &  & 0.800$^{\ast\ast\ast}$ &  \\
    164 &  & 0.288$^{\ast\ast\ast}$ & 0.343$^{\ast\ast\ast}$ \\
    715 &  & 0.102$^{\ast}$ & 0.516$^{\ast\ast\ast}$ \\
    \hline
    \multicolumn{4}{l}{\footnotesize{Notes: Robust standard errors clustered at the country-pair level in parentheses.}} \\
    \multicolumn{4}{l}{\footnotesize{Significance levels are indicated as follows: $^{\ast}$p$<$0.1; $^{\ast\ast}$p$<$0.05; $^{\ast\ast\ast}$p$<$0.01.}} \\
    \end{tabular}
    \end{adjustbox}
\end{table}%
\FloatBarrier

%
\subsection{Export Product Unit Value Results}%
\label{subsec:ExportProductUnitValueResults}%
Finally, we present the results of running our estimations substituting
trade flows as our dependent variable for the unit values of products
exported, specifically under the HS 2-digit codes 84 and 85 for
manufacturing products, in order to analyse if the effect of PTAs goes
beyond trade volumes. For ease of comparison, we ran each estimation
twice for each HS code: one with trade volume as the dependent variable,
and one with the unit value of the product exported as the dependent
variable.

Tables 13 and 14, and 15 and 16, show the results of our benchmark model
for each region for trade volumes and the unit value of the product
exported, and for HS 84 and 85, respectively. We continue to observe
heterogeneous results across regions. In table 13, for the trade volume
of HS 84, none of the PTA + Lag coefficients are statistically
significant with the exception of the Intercontinental region, for which
it is statistically significant and negative. In table 14, for the unit
value of the product exported of HS 84, the effects are not significant
for Africa and Asia, they are significant and negative for Americas, and
significant and positive for Europe and Intercontinental. Interestingly,
these results suggest that Intercontinental PTAs reduced the volume of
trade of HS 84 products but increased the value per unit. In table 15,
for the trade volume of HS 85, PTA + Lag coefficients are not
statistically significant for Americas, Asia and Intercontinental, while
Africa's results are significant and positive, and Europe's are
significant and negative. In table 16, for the unit value of the product
exported of HS 85, results are only slightly significant for
Intercontinental, with a negative coefficient. The rest of the regions
do not have significant results.

Tables 17 and 18, and 19 and 20, show the results of our extended
benchmark model with North-North, North-South and South-South PTAs, for
each region for trade volumes and the unit value of the product
exported, and for HS 84 and 85, respectively. In table 17, for the trade
volume of HS 84, we observe that for North-North trade, PTA + Lag
coefficient for Intercontinental has a significant and positive
coefficient, while Europe's is not significant. For North-South trade
PTA + Lag coefficients are not significant for Asia and Europe, while
they are significant and positive for Americas, and significant and
negative for Intercontinental. For South-South trade, PTA + Lag for
Africa, Asia and Europe do not have significant coefficients, while the
coefficients of Americas and Intercontinental are significant and
negative. In table 18, for the unit value of the product exported of HS
84, for North-North trade's PTA + Lag, Europe's coefficient is
significant and positive and the coefficient of Intercontinental is not
significant. For North-South trade, none of the PTA + Lag coefficients
are significant. For South-South trade, the PTA + Lag coefficients of
Africa, Americas and Asia are not significant, while Europe and
Intercontinental have significant and positive coefficients.
Interestingly, while trade volume for North-South and South-South for
Intercontinental PTAs decreased, the value per unit of South-South trade
increased. In table 19, for the trade volume of HS 85, we observe that
for North-North trade, PTA + Lag coefficient for Intercontinental has a
significant and positive coefficient, while Europe's is not significant.
For North-South trade PTA + Lag coefficients are not significant for
Americas, Asia and Intercontinental, while they are significant and
negative for Europe. For South-South trade, PTA + Lag for Americas,
Asia, Europe and Intercontinental do not have significant coefficients,
while the coefficient of Africa is significant and positive. In table
20, for the unit value of the product exported of HS 85, for North-North
trade's PTA + Lag, Europe and Intercontinental's coefficients are not
significant. For North-South trade, the PTA + Lag coefficients for
Americas and Europe are not significant, while they are significant and
negative for Asia and Intercontinental. For South-South trade, the PTA +
Lag coefficients of Africa, Americas and Intercontinental are not
significant, while Europe has significant and negative coefficients and
Asia has significant and positive coefficients. Interestingly, for
Asia's exports, the value per unit of product exported decreased with
North-South trade but increased with South-South trade.

Finally, for illustrative purposed, in tables 21 and 22, and 23 and 24,
we include the estimates of our model allowing for PTA specific effects,
extended with North-North, North-South and South-South PTAs, for Africa
and Americas, for trade volumes and the unit value of the product
exported, and for HS 84 and 85, respectively. In table 21, for the trade
volumes of HS 84 and 85 for Africa, which only has South-South PTAs, we
can see that PTA 670 had statistically significant and negative effects
on the trade volume of HS 84, and not significant for HS 85. PTA 787 did
not have a significant impact on trade volume of HS 84, while it has
significant and positive effects on HS 85. In table 22, for the unit
value of products HS 84 and 85 exported for the region of Africa, we can
see that PTA 670 did not have significant effects on the value per unit
of products in HS 84 and 85. PTA 787 did not have a significant impact
on the value per unit of HS 84, while it has significant and positive
effects on HS 85. This is a case where we can see a that a PTA has a
significant effect on the volume of trade and in the value per unit of a
category of manufacturing products of a South-South trade relationship.

In table 23, for the trade volumes of HS 84 and 85, and table 24 for the
unit value of products HS 84 and 85, all for the region of Americas,
which has North-South and South-South PTAs, we can observe heterogeneous
effects of different PTAs on the different types of bilateral trade
relationships. One interesting example is PTA 188, which has North-South
and South-South trade among its members. It has positive and significant
effects in the trade volumes of HS 84 and 85 for South-South trade,
while it has no significant effect in the trade volume of HS 84 and 85
for North-South trade. Furthermore, it has a significant and negative
effects on the value per unit of HS 84 for both North-South and
South-South trade, and it has no significant effect on the value per
unit of HS 85 for both North-South and South-South trade.
%
\begin{table}[htbp]
    \centering
    \caption{HS 84 Trade Volume Benchmark Model Regional Results}
    \label{tab:84_trade_benchmark_region_analysis}
    \begin{adjustbox}{max width=\textwidth}
    \begin{tabular}{l@{\extracolsep{1pt}}ccccc}
    \hline
    & \multicolumn{1}{c}{(1)} & \multicolumn{1}{c}{(2)} & \multicolumn{1}{c}{(3)} & \multicolumn{1}{c}{(4)} & \multicolumn{1}{c}{(5)} \\
    \hline
    \textbf{Variables} &  &  &  &  &  \\
    \hline
     & PPML & PPML & PPML & PPML & PPML \\
     & Africa & Americas & Asia & Europe & Intercontinental \\
    \hline
    TA & -0.364 & -0.289$^{\ast}$ & -0.005 & 0.288$^{\ast\ast}$ & -0.411$^{\ast\ast\ast}$ \\
    & (0.695) & (0.162) & (0.078) & (0.146) & (0.099) \\

    TA Lag & -0.247 & -0.024 & -0.053 & -0.233$^{\ast}$ & -0.081 \\
    & (0.403) & (0.120) & (0.048) & (0.122) & (0.077) \\

    TA + TA Lag & -0.610 & -0.313 & -0.057 & 0.056 & -0.491$^{\ast\ast\ast}$ \\
    & (0.676) & (0.201) & (0.080) & (0.165) & (0.126) \\
    \hline
    Exporter-Year FE & Yes & Yes & Yes & Yes & Yes \\
    Importer-Year FE & Yes & Yes & Yes & Yes & Yes \\
    Country-Pair FE & Yes & Yes & Yes & Yes & Yes \\
    R-Squared & 0.997 & 0.997 & 0.992 & 0.986 & 0.989 \\
    Observations & 1314 & 4230 & 10778 & 18152 & 36735 \\
    \hline
    \multicolumn{6}{l}{\footnotesize{Notes: Robust standard errors clustered at the country-pair level in parentheses. Significance levels are indicated as follows: $^{\ast}$p$<$0.1; $^{\ast\ast}$p$<$0.05; $^{\ast\ast\ast}$p$<$0.01.}} \\
    \end{tabular}
    \end{adjustbox}
\end{table}
%
\begin{table}[htbp]
    \centering
    \caption{HS 84 EPUV Benchmark Model Regional Results}
    \label{tab:84_benchmark_region_analysis} % This allows you to reference the table in the text with \ref{tab:TA_analysis}
    \begin{adjustbox}{max width=\textwidth}
    \begin{tabular}{l@{\extracolsep{1pt}}ccccc}
    \hline
    & \multicolumn{1}{c}{(1)} & \multicolumn{1}{c}{(2)} & \multicolumn{1}{c}{(3)} & \multicolumn{1}{c}{(4)} & \multicolumn{1}{c}{(5)} \\
    \hline
    \textbf{Variables} &  &  &  &  &  \\
    \hline
     & PPML & PPML & PPML & PPML & PPML \\
     & Africa & Americas & Asia & Europe & Intercontinental \\
    \hline
    TA & 1.676$^{\ast\ast\ast}$ & -0.037 & 0.130 & 0.164 & -0.236 \\
    & (0.592) & (0.356) & (0.165) & (0.223) & (0.150) \\

    TA Lag & -2.388$^{\ast\ast\ast}$ & -0.615 & -0.129 & 0.238 & 0.534$^{\ast\ast\ast}$ \\
    & (0.517) & (0.484) & (0.136) & (0.188) & (0.156) \\

    TA + TA Lag & -0.712 & -0.652$^{\ast}$ & 0.001 & 0.402$^{\ast}$ & 0.298$^{\ast\ast}$ \\
    & (0.492) & (0.339) & (0.192) & (0.216) & (0.141) \\
    \hline
    Exporter-Year FE & Yes & Yes & Yes & Yes & Yes \\
    Importer-Year FE & Yes & Yes & Yes & Yes & Yes \\
    Country-Pair FE & Yes & Yes & Yes & Yes & Yes \\
    R-Squared & 0.960 & 0.986 & 0.982 & 0.956 & 0.966 \\
    Observations & 1299 & 4053 & 10223 & 18019 & 35947 \\
    \hline
    \multicolumn{6}{l}{\footnotesize{Notes: Robust standard errors clustered at the country-pair in parentheses. Significance levels are indicated as follows: $^{\ast}$p$<$0.1; $^{\ast\ast}$p$<$0.05; $^{\ast\ast\ast}$p$<$0.01.}} \\
    \end{tabular}
    \end{adjustbox}
\end{table}
%
\begin{table}[htbp]
    \centering
    \caption{HS 85 Trade Volume Benchmark Model Regional Results}
    \label{tab:85_trade_benchmark_region_analysis}
    \begin{adjustbox}{max width=\textwidth}
    \begin{tabular}{l@{\extracolsep{1pt}}ccccc}
    \hline
    & \multicolumn{1}{c}{(1)} & \multicolumn{1}{c}{(2)} & \multicolumn{1}{c}{(3)} & \multicolumn{1}{c}{(4)} & \multicolumn{1}{c}{(5)} \\
    \hline
    \textbf{Variables} &  &  &  &  &  \\
    \hline
     & PPML & PPML & PPML & PPML & PPML \\
     & Africa & Americas & Asia & Europe & Intercontinental \\
    \hline
    TA & 0.023 & -0.106 & 0.140 & -0.138 & 0.142$^{\ast}$ \\
    & (0.419) & (0.178) & (0.088) & (0.152) & (0.072) \\

    TA Lag & 1.009$^{\ast\ast}$ & 0.052 & -0.070 & -0.311$^{\ast\ast}$ & -0.209$^{\ast\ast}$ \\
    & (0.441) & (0.172) & (0.058) & (0.127) & (0.926) \\

    TA + TA Lag & 1.033$^{\ast\ast}$ & -0.055 & 0.069 & -0.449$^{\ast\ast\ast}$ & -0.067 \\
    & (0.404) & (0.253) & (0.102) & (0.165) & (0.872) \\
    \hline
    Exporter-Year FE & Yes & Yes & Yes & Yes & Yes \\
    Importer-Year FE & Yes & Yes & Yes & Yes & Yes \\
    Country-Pair FE & Yes & Yes & Yes & Yes & Yes \\
    R-Squared & 0.989 & 0.998 & 0.993 & 0.980 & 0.989 \\
    Observations & 1205 & 3836 & 10465 & 16436 & 33999 \\
    \hline
    \multicolumn{6}{l}{\footnotesize{Notes: Robust standard errors clustered at the country-pair level in parentheses. Significance levels are indicated as follows: $^{\ast}$p$<$0.1; $^{\ast\ast}$p$<$0.05; $^{\ast\ast\ast}$p$<$0.01.}} \\
    \end{tabular}
    \end{adjustbox}
\end{table}
%
\begin{table}[htbp]
    \centering
    \caption{HS 85 Benchmark Model Regional Results}
    \label{tab:85_benchmark_region_analysis} % This allows you to reference the table in the text with \ref{tab:TA_analysis}
    \begin{adjustbox}{max width=\textwidth}
    \begin{tabular}{l@{\extracolsep{1pt}}ccccc}
    \hline
    & \multicolumn{1}{c}{(1)} & \multicolumn{1}{c}{(2)} & \multicolumn{1}{c}{(3)} & \multicolumn{1}{c}{(4)} & \multicolumn{1}{c}{(5)} \\
    \hline
    \textbf{Variables} &  &  &  &  &  \\
    \hline
     & PPML & PPML & PPML & PPML & PPML \\
     & Africa & Americas & Asia & Europe & Intercontinental \\
    \hline
    TA & 2.098$^{\ast\ast}$ & -0.360 & 0.965$^{\ast\ast\ast}$ & -0.198 & -0.010 \\
    & (1.032) & (0.583) & (0.324) & (0.278) & (0.190) \\

    TA Lag & -0.478 & 0.421 & -0.299 & -0.184 &- 0.280 \\
    & (0.650) & (0.408) & (0.294) & (0.247) & (0.208) \\

    TA + TA Lag & 1.620 & 0.062 & 0.666 & -0.382 & -0.290$^{\ast}$ \\
    & (1.150) & (0.524) & (0.494) & (0.333) & (0.175) \\
    \hline
    Exporter-Year FE & Yes & Yes & Yes & Yes & Yes \\
    Importer-Year FE & Yes & Yes & Yes & Yes & Yes \\
    Country-Pair FE & Yes & Yes & Yes & Yes & Yes \\
    R-Squared & 0.939 & 0.990 & 0.992 & 0.950 & 0.956 \\
    Observations & 1130 & 3698 & 9934 & 16235 & 33070 \\
    \hline
    \multicolumn{6}{l}{\footnotesize{Notes: Robust standard errors clustered at the country-pair in parentheses. Significance levels are indicated as follows: $^{\ast}$p$<$0.1; $^{\ast\ast}$p$<$0.05; $^{\ast\ast\ast}$p$<$0.01.}} \\
    \end{tabular}
    \end{adjustbox}
\end{table}
%
\begin{table}[htbp]
    \centering
    \caption{HS 84 Trade Volume Regional Results by TA Type}
    \label{tab:84_trade_pta_types}
    \begin{adjustbox}{max width=\textwidth}
    \begin{tabular}{lccccc}
    \hline
     & \multicolumn{1}{c}{Africa} & \multicolumn{1}{c}{Americas} & \multicolumn{1}{c}{Asia} & \multicolumn{1}{c}{Europe} & \multicolumn{1}{c}{Intercontinental} \\
    \hline
    \textbf{Variables} &  &  &  &  &  \\
    \hline
    NN TA &  &  &  & -0.087 & 0.084 \\
     &  &  &  & (0.163) & (0.080) \\
    NN TA Lag &  &  &  & -0.234 & 0.187$^{\ast}$ \\
     &  &  &  & (0.191) & (0.098) \\
    NN TA + NN TA Lag &  &  &  & -0.321 & 0.272$^{\ast\ast}$ \\
     &  &  &  & (0.233) & (0.121) \\
    \hline
    NS TA &  & -0.082 & -0.001 & 0.236$^{\ast}$ & -0.455$^{\ast\ast\ast}$ \\
     &  & (0.108) & (0.096) & (0.133) & (0.104) \\
    NS TA Lag &  & 0.294$^{\ast}$ & -0.079 & -0.242$^{\ast}$ & -0.126 \\
     &  & (0.151) & (0.059) & (0.139) & (0.093) \\
    NS TA + NS TA Lag &  & 0.212$^{\ast\ast}$ & -0.080 & -0.006 & -0.580$^{\ast\ast\ast}$ \\
     &  & (0.097) & (0.112) & (0.169) & (0.111) \\
    \hline
    SS TA & -0.364 & -0.310$^{\ast}$ & -0.006 & 0.417$^{\ast}$ & -0.315$^{\ast\ast\ast}$ \\
     & (0.695) & (0.189) & (0.117) & (0.215) & (0.102) \\
    SS TA Lag & -0.247 & -0.123 & -0.037 & -0.129 & 0.057 \\
     & (0.403) & (0.112) & (0.080) & (0.160) & (0.082) \\
    SS TA + SS TA Lag & -0.610 & -0.433$^{\ast}$ & -0.043 & 0.287 & -0.258$^{\ast\ast}$ \\
     & (0.676) & (0.229) & (0.126) & (0.228) & (0.125) \\
    \hline
    Exporter-Year FE & Yes & Yes & Yes & Yes & Yes \\
    Importer-Year FE & Yes & Yes & Yes & Yes & Yes \\
    Country-Pair FE & Yes & Yes & Yes & Yes & Yes \\
    R-Squared & 0.997 & 0.997 & 0.992 & 0.986 & 0.989 \\
    Observations & 1314 & 4230 & 10778 & 18152 & 36735 \\
    \hline
    \multicolumn{6}{l}{\footnotesize{Notes: Robust standard errors clustered at the country-pair level in parentheses.}} \\
    \multicolumn{6}{l}{\footnotesize{Significance levels are indicated as follows: $^{\ast}$p$<$0.1; $^{\ast\ast}$p$<$0.05; $^{\ast\ast\ast}$p$<$0.01.}} \\
    \end{tabular}
    \end{adjustbox}
\end{table}
%
\begin{table}[htbp]
    \centering
    \caption{HS 84 Regional Results by PTA Type}
    \label{tab:84_pta_types}
    \begin{adjustbox}{max width=\textwidth}
    \begin{tabular}{lccccc}
    \hline
     & \multicolumn{1}{c}{Africa} & \multicolumn{1}{c}{Americas} & \multicolumn{1}{c}{Asia} & \multicolumn{1}{c}{Europe} & \multicolumn{1}{c}{Intercontinental} \\
    \hline
    \textbf{Variables} &  &  &  &  &  \\
    \hline
    NN PTA &  &  &  & 0.316 & 0.584 \\
     &  &  &  & (0.258) & (0.516) \\
    NN PTA Lag &  &  &  & 0.250 & -0.740$^{\ast\ast}$ \\
     &  &  &  & (0.226) & (0.296) \\
    NN PTA + NN PTA Lag &  &  &  & 0.566$^{\ast\ast}$ & -0.155 \\
     &  &  &  & (0.266) & (0.362) \\
    \hline
    NS PTA &  & 1.033$^{\ast\ast}$ & 0.403 & 0.202 & -0.345$^{\ast\ast}$ \\
     &  & (0.471) & (0.278) & (0.236) & (0.166) \\
    NS PTA Lag &  & -1.925$^{\ast\ast\ast}$ & -0.005 & 0.139 & 0.576$^{\ast\ast\ast}$ \\
     &  & (0.609) & (0.244) & (0.202) & (0.180) \\
    NS PTA + NS PTA Lag &  & -0.891 & 0.399 & 0.341 & 0.231 \\
     &  & (0.601) & (0.268) & (0.227) & (0.170) \\
    \hline
    SS PTA & 1.676$^{\ast\ast\ast}$ & -0.974$^{\ast\ast\ast}$ & -0.004 & 0.097 & -0.063 \\
     & (0.592) & (0.324) & (0.195) & (0.265) & (0.231) \\
    SS PTA Lag & -2.388$^{\ast\ast\ast}$ & 0.603$^{\ast}$ & -0.148 & 0.327 & 0.542$^{\ast\ast}$ \\
     & (0.517) & (0.311) & (0.149) & (0.232) & (0.234) \\
    SS PTA + SS PTA Lag & -0.712 & -0.371 & -0.152 & 0.424$^{\ast}$ & 0.479$^{\ast\ast}$ \\
     & (0.492) & (0.368) & (0.233) & (0.253) & (0.196) \\
    \hline
    Exporter-Year FE & Yes & Yes & Yes & Yes & Yes \\
    Importer-Year FE & Yes & Yes & Yes & Yes & Yes \\
    Country-Pair FE & Yes & Yes & Yes & Yes & Yes \\
    R-Squared & 0.960 & 0.986 & 0.982 & 0.956 & 0.966 \\
    Observations & 1299 & 4053 & 10223 & 18019 & 35947 \\
    \hline
    \multicolumn{6}{l}{\footnotesize{Notes: Robust standard errors clustered at the country-pair level in parentheses.}} \\
    \multicolumn{6}{l}{\footnotesize{Significance levels are indicated as follows: $^{\ast}$p$<$0.1; $^{\ast\ast}$p$<$0.05; $^{\ast\ast\ast}$p$<$0.01.}} \\
    \end{tabular}
    \end{adjustbox}
\end{table}
%
\begin{table}[htbp]
    \centering
    \caption{HS 85 Trade Volume Regional Results by PTA Type}
    \label{tab:85_trade_pta_types}
    \begin{adjustbox}{max width=\textwidth}
    \begin{tabular}{lccccc}
    \hline
     & \multicolumn{1}{c}{Africa} & \multicolumn{1}{c}{Americas} & \multicolumn{1}{c}{Asia} & \multicolumn{1}{c}{Europe} & \multicolumn{1}{c}{Intercontinental} \\
    \hline
    \textbf{Variables} &  &  &  &  &  \\
    \hline
    NN PTA &  &  &  & 0.041 & 0.272$^{\ast\ast}$ \\
     &  &  &  & (0.208) & (0.128) \\
    NN PTA Lag &  &  &  & -0.160 & 0.271 \\
     &  &  &  & (0.208) & (0.190) \\
    NN PTA + NN PTA Lag &  &  &  & -0.119 & 0.543$^{\ast\ast}$ \\
     &  &  &  & (0.246) & (0.274) \\
    \hline
    NS PTA &  & -0.494 & 0.158$^{\ast}$ & -0.051 & 0.154$^{\ast}$ \\
     &  & (0.345) & (0.085) & (0.152) & (0.084) \\
    NS PTA Lag &  & 0.700$^{\ast\ast\ast}$ & -0.038 & -0.315$^{\ast\ast}$ & -0.248$^{\ast\ast}$ \\
     &  & (0.270) & (0.094) & (0.153) & (0.108) \\
    NS PTA + NS PTA Lag &  & 0.206 & 0.121 & -0.366$^{\ast\ast}$ & -0.094 \\
     &  & (0.442) & (0.120) & (0.174) & (0.091) \\
    \hline
    SS PTA & 0.023 & 0.082 & 0.118 & -0.004 & 0.039 \\
     & (0.419) & (0.206) & (0.128) & (0.211) & (0.152) \\
    SS PTA Lag & 1.009$^{\ast\ast}$ & -0.176 & -0.088 & -0.142 & -0.090 \\
     & (0.441) & (0.168) & (0.081) & (0.157) & (0.176) \\
    SS PTA + SS PTA Lag & 1.033$^{\ast\ast}$ & -0.094 & 0.030 & -0.146 & -0.051 \\
     & (0.404) & (0.280) & (0.160) & (0.196) & (0.194) \\
    \hline
    Exporter-Year FE & Yes & Yes & Yes & Yes & Yes \\
    Importer-Year FE & Yes & Yes & Yes & Yes & Yes \\
    Country-Pair FE & Yes & Yes & Yes & Yes & Yes \\
    R-Squared & 0.989 & 0.998 & 0.993 & 0.980 & 0.989 \\
    Observations & 1205 & 3836 & 10465 & 16436 & 33999 \\
    \hline
    \multicolumn{6}{l}{\footnotesize{Notes: Robust standard errors clustered at the country-pair level in parentheses.}} \\
    \multicolumn{6}{l}{\footnotesize{Significance levels are indicated as follows: $^{\ast}$p$<$0.1; $^{\ast\ast}$p$<$0.05; $^{\ast\ast\ast}$p$<$0.01.}} \\
    \end{tabular}
    \end{adjustbox}
\end{table}
%
\begin{table}[htbp]
    \centering
    \caption{HS 85 EPUV Regional Results by TA Type}
    \label{tab:85_pta_types}
    \begin{adjustbox}{max width=\textwidth}
    \begin{tabular}{lccccc}
    \hline
     & \multicolumn{1}{c}{Africa} & \multicolumn{1}{c}{Americas} & \multicolumn{1}{c}{Asia} & \multicolumn{1}{c}{Europe} & \multicolumn{1}{c}{Intercontinental} \\
    \hline
    \textbf{Variables} &  &  &  &  &  \\
    \hline
    NN TA &  &  &  & 0.024 & 0.867$^{\ast}$ \\
     &  &  &  & (0.349) & (0.494) \\
    NN TA Lag &  &  &  & 0.409 & -0.847$^{\ast\ast}$ \\
     &  &  &  & (0.292) & (0.411) \\
    NN TA + NN TA Lag &  &  &  & 0.433 & 0.020 \\
     &  &  &  & (0.364) & (0.490) \\
    \hline
    NS TA &  & -0.582 & 0.076 & -0.244 & -0.133 \\
     &  & (1.139) & (0.388) & (0.332) & (0.198) \\
    NS TA Lag &  & 0.918 & -1.017$^{\ast\ast\ast}$ & 0.084 & -0.200 \\
     &  & (0.629) & (0.370) & (0.245) & (0.232) \\
    NS TA + NS TA Lag &  & 0.336 & -0.941$^{\ast\ast}$ & -0.160 & -0.333$^{\ast}$ \\
     &  & (0.851) & (0.407) & (0.356) & (0.199) \\
    \hline
    SS TA & 2.098$^{\ast\ast}$ & -0.208 & 1.662$^{\ast\ast\ast}$ & -0.218 & 0.097 \\
     & (1.032) & (0.517) & (0.481) & (0.369) & (0.301) \\
    SS TA Lag & -0.478 & 0.068 & 0.026 & -0.672$^{\ast\ast}$ & -0.316 \\
     & (0.650) & (0.493) & (0.328) & (0.336) & (0.298) \\
    SS TA + SS TA Lag & 1.620 & -0.139 & 1.688$^{\ast\ast}$ & -0.890$^{\ast\ast}$ & -0.219 \\
     & (1.150) & (0.689) & (0.679) & (0.414) & (0.250) \\
    \hline
    Exporter-Year FE & Yes & Yes & Yes & Yes & Yes \\
    Importer-Year FE & Yes & Yes & Yes & Yes & Yes \\
    Country-Pair FE & Yes & Yes & Yes & Yes & Yes \\
    R-Squared & 0.939 & 0.990 & 0.992 & 0.951 & 0.956 \\
    Observations & 1130 & 3698 & 9934 & 16235 & 33070 \\
    \hline
    \multicolumn{6}{l}{\footnotesize{Notes: Robust standard errors clustered at the country-pair level in parentheses.}} \\
    \multicolumn{6}{l}{\footnotesize{Significance levels are indicated as follows: $^{\ast}$p$<$0.1; $^{\ast\ast}$p$<$0.05; $^{\ast\ast\ast}$p$<$0.01.}} \\
    \end{tabular}
    \end{adjustbox}
\end{table}
%
\begin{table}[htbp]
    \centering
    \caption{Africa PTA + PTA Lag Coefficients by Type for Trade Volume of HS 84 and HS 85}
    \label{tab:HS_trade_africa_pta}
    \begin{adjustbox}{max width=\textwidth}
    \begin{tabular}{lcccccc}
    \hline
    \textbf{PTA ID} & \multicolumn{3}{c}{\textbf{HS 84}} & \multicolumn{3}{c}{\textbf{HS 85}} \\
    & \textbf{NS PTA+Lag} & \textbf{SS PTA+Lag} & \textbf{NN PTA+Lag} & \textbf{NS PTA+Lag} & \textbf{SS PTA+Lag} & \textbf{NN PTA+Lag} \\
    \hline
    \textbf{NS and SS (or only NS)} &  &  &  &  &  &  \\
    \hline
    \multicolumn{7}{c}{No agreements in this category} \\
    \hline
    \textbf{Only SS} &  &  &  &  &  &  \\
    \hline
    670 &  & -2.234$^{\ast\ast\ast}$ &  &  & -0.041 &  \\
    &  & (0.678) &  &  & (1.008) &  \\
    787 &  & -0.682 &  &  & 1.507$^{\ast\ast\ast}$ &  \\
    &  & (0.781) &  &  & (0.573) &  \\
    \hline
    \textbf{Agreements with NN and NS} &  &  &  &  &  &  \\
    \hline
    \multicolumn{7}{c}{No agreements in this category} \\
    \hline
    Exporter-Year FE & Yes & Yes & Yes & Yes & Yes & Yes \\
    Importer-Year FE & Yes & Yes & Yes & Yes & Yes & Yes \\
    Country-Pair FE & Yes & Yes & Yes & Yes & Yes & Yes \\
    R-Squared & 0.997 & 0.997 & 0.997 & 0.989 & 0.989 & 0.989 \\
    Observations & 1314 & 1314 & 1314 & 1205 & 1205 & 1205 \\
    \hline
    \multicolumn{7}{l}{\footnotesize{Notes: Robust standard errors clustered at the country-pair level in parentheses.}} \\
    \multicolumn{7}{l}{\footnotesize{Significance levels are indicated as follows: $^{\ast}$p$<$0.1; $^{\ast\ast}$p$<$0.05; $^{\ast\ast\ast}$p$<$0.01.}} \\
    \end{tabular}
    \end{adjustbox}
\end{table}
%
\begin{table}[htbp]
    \centering
    \caption{Africa TA + TA Lag Coefficients by Type for HS 84 and HS 85}
    \label{tab:epuv_africa_pta}
    \begin{adjustbox}{max width=\textwidth}
    \begin{tabular}{lcccccc}
    \hline
    \textbf{TA ID} & \multicolumn{3}{c}{\textbf{HS 84}} & \multicolumn{3}{c}{\textbf{HS 85}} \\
    & \textbf{NS TA+Lag} & \textbf{SS TA+Lag} & \textbf{NN TA+Lag} & \textbf{NS TA+Lag} & \textbf{SS TA+Lag} & \textbf{NN TA+Lag} \\
    \hline
    \textbf{NS and SS (or only NS)} &  &  &  &  &  &  \\
    \hline
    \multicolumn{7}{c}{No agreements in this category} \\
    \hline
    \textbf{Only SS} &  &  &  &  &  &  \\
    \hline
    670 &  & -0.975 &  &  & -0.860 &  \\
     &  & (0.802) &  &  & (1.031) &  \\
    787 &  & -0.693 &  &  & 2.760$^{\ast\ast}$ &  \\
     &  & (0.590) &  &  & (1.148) &  \\
    \hline
    \textbf{Agreements with NN and NS} &  &  &  &  &  &  \\
    \hline
    \multicolumn{7}{c}{No agreements in this category} \\
    \hline
    Exporter-Year FE & Yes & Yes & Yes & Yes & Yes & Yes \\
    Importer-Year FE & Yes & Yes & Yes & Yes & Yes & Yes \\
    Country-Pair FE & Yes & Yes & Yes & Yes & Yes & Yes \\
    R-Squared & 0.960 & 0.960 & 0.960 & 0.939 & 0.939 & 0.939 \\
    Observations & 1299 & 1299 & 1299 & 1130 & 1130 & 1130 \\
    \hline
    \multicolumn{7}{l}{\footnotesize{Notes: Robust standard errors clustered at the country-pair level in parentheses.}} \\
    \multicolumn{7}{l}{\footnotesize{Significance levels are indicated as follows: $^{\ast}$p$<$0.1; $^{\ast\ast}$p$<$0.05; $^{\ast\ast\ast}$p$<$0.01.}} \\
    \end{tabular}
    \end{adjustbox}
\end{table}
%
\begin{table}[htbp]
    \centering
    \caption{Americas PTA + PTA Lag Coefficients by Type for Trade Volume of HS 84 and HS 85}
    \label{tab:HS_trade_americas_pta}
    \begin{adjustbox}{max width=\textwidth}
    \begin{tabular}{lcccccc}
    \hline
    \textbf{PTA ID} & \multicolumn{3}{c}{\textbf{HS 84}} & \multicolumn{3}{c}{\textbf{HS 85}} \\
    & \textbf{NS PTA+Lag} & \textbf{SS PTA+Lag} & \textbf{NN PTA+Lag} & \textbf{NS PTA+Lag} & \textbf{SS PTA+Lag} & \textbf{NN PTA+Lag} \\
    \hline
    \textbf{NS and SS (or only NS)} &  &  &  &  &  &  \\
    \hline
    188 & 0.056 & 3.233$^{\ast\ast\ast}$ &  & 0.483 & 1.123$^{\ast\ast\ast}$ &  \\
    & (0.769) & (0.566) &  & (0.440) & (0.223) &  \\
    163 & 0.579$^{\ast\ast\ast}$ &  &  & -0.095 &  &  \\
     & (0.151) &  &  & (0.641) &  &  \\
    168 & 0.191$^{\ast\ast}$ &  &  & -0.514 &  &  \\
     & (0.077) &  &  & (0.334) &  &  \\
    218 & 0.401$^{\ast\ast\ast}$ &  &  & 1.765$^{\ast\ast\ast}$ &  &  \\
     & (0.124) &  &  & (0.331) &  &  \\
    645 & 0.296$^{\ast\ast}$ &  &  & -1.341$^{\ast\ast\ast}$ &  &  \\
     & (0.148) &  &  & (0.425) &  &  \\
    \hline
    \textbf{Only SS} &  &  &  &  &  &  \\
    \hline
    141 &  & -0.705$^{\ast}$ &  &  & -0.613 &  \\
     &  & (0.372) &  &  & (0.388) &  \\
    213 &  & 0.326 &  &  & 1.233$^{\ast\ast\ast}$ &  \\
     &  & (0.397) &  &  & (0.253) &  \\
    239 &  & -0.030 &  &  & 0.008 &  \\
     &  & (0.271) &  &  & (0.374) &  \\
    616 &  & -0.019 &  &  & -0.416$^{\ast\ast\ast}$ &  \\
     &  & (0.218) &  &  & (0.146) &  \\
    201 &  & 0.479$^{\ast\ast}$ &  &  & 0.971$^{\ast\ast\ast}$ &  \\
     &  & (0.213) &  &  & (0.257) &  \\
    716 &  & 0.270$^{\ast}$ &  &  & -0.349 &  \\
     &  & (0.141) &  &  & (0.391) &  \\
    612 &  & -0.704$^{\ast\ast\ast}$ &  &  & 1.089$^{\ast\ast\ast}$ &  \\
     &  & (0.180) &  &  & (0.276) &  \\
    185 &  & 0.238 &  &  & -1.303$^{\ast\ast\ast}$ &  \\
     &  & (0.399) &  &  & (0.278) &  \\
    \hline
    \textbf{Agreements with NN and NS} &  &  &  &  &  &  \\
    \hline
    \multicolumn{7}{c}{No agreements in this category} \\
    \hline
    Exporter-Year FE & Yes & Yes & Yes & Yes & Yes & Yes \\
    Importer-Year FE & Yes & Yes & Yes & Yes & Yes & Yes \\
    Country-Pair FE & Yes & Yes & Yes & Yes & Yes & Yes \\
    R-Squared & 0.997 & 0.997 & 0.997 & 0.998 & 0.998 & 0.998 \\
    Observations & 4230 & 4230 & 4230 & 3836 & 3836 & 3836 \\
    \hline
    \multicolumn{7}{l}{\footnotesize{Notes: Robust standard errors clustered at the country-pair level in parentheses.}} \\
    \multicolumn{7}{l}{\footnotesize{Significance levels are indicated as follows: $^{\ast}$p$<$0.1; $^{\ast\ast}$p$<$0.05; $^{\ast\ast\ast}$p$<$0.01.}} \\
    \end{tabular}
    \end{adjustbox}
\end{table}
%
\begin{table}[htbp]
    \centering
    \caption{Americas PTA + PTA Lag Coefficients by Type for HS 84 and HS 85}
    \label{tab:epuv_americas_pta}
    \begin{adjustbox}{max width=\textwidth}
    \begin{tabular}{lcccccc}
    \hline
    \textbf{PTA ID} & \multicolumn{3}{c}{\textbf{HS 84}} & \multicolumn{3}{c}{\textbf{HS 85}} \\
    & \textbf{NS PTA+Lag} & \textbf{SS PTA+Lag} & \textbf{NN PTA+Lag} & \textbf{NS PTA+Lag} & \textbf{SS PTA+Lag} & \textbf{NN PTA+Lag} \\
    \hline
    \textbf{NS and SS (or only NS)} &  &  &  &  &  &  \\
    \hline
    188 & -3.217$^{\ast\ast\ast}$ & -2.778$^{\ast\ast\ast}$ &  & -0.568 & 0.797 &  \\
    & (0.748) & (1.013) &  & (0.641) & (0.606) &  \\
    163 & -1.314$^{\ast}$ &  &  & 1.272$^{\ast}$ &  &  \\
     & (0.704) &  &  & (0.715) &  &  \\
    168 & 1.236$^{\ast\ast\ast}$ &  &  & 1.189 &  &  \\
     & (0.424) &  &  & (1.497) &  &  \\
    218 & -3.916$^{\ast\ast\ast}$ &  &  & 1.103 &  &  \\
     & (0.716) &  &  & (0.822) &  &  \\
    645 & -0.791 &  &  & -1.662$^{\ast\ast}$ &  &  \\
     & (0.885) &  &  & (0.658) &  &  \\
    \hline
    \textbf{Only SS} &  &  &  &  &  &  \\
    \hline
    141 &  & -0.854$^{\ast\ast}$ &  &  & 0.662 &  \\
     &  & (0.375) &  &  & (0.582) &  \\
    213 &  & -0.506 &  &  & 1.089 &  \\
     &  & (0.456) &  &  & (0.728) &  \\
    239 &  & 1.263 &  &  & 1.457 &  \\
     &  & (0.866) &  &  & (0.895) &  \\
    616 &  & -0.638 &  &  & 0.728 &  \\
     &  & (0.435) &  &  & (0.636) &  \\
    201 &  & -0.554 &  &  & 1.581$^{\ast\ast\ast}$ &  \\
     &  & (0.610) &  &  & (0.390) &  \\
    716 &  & -0.572 &  &  & 2.042 &  \\
     &  & (1.223) &  &  & (1.478) &  \\
    612 &  & -0.015 &  &  & -2.843$^{\ast\ast\ast}$ &  \\
     &  & (0.274) &  &  & (1.045) &  \\
    185 &  & 1.023 &  &  & 0.768 &  \\
     &  & (0.784) &  &  & (1.005) &  \\
    \hline
    \textbf{Agreements with NN and NS} &  &  &  &  &  &  \\
    \hline
    \multicolumn{7}{c}{No agreements in this category} \\
    \hline
    Exporter-Year FE & Yes & Yes & Yes & Yes & Yes & Yes \\
    Importer-Year FE & Yes & Yes & Yes & Yes & Yes & Yes \\
    Country-Pair FE & Yes & Yes & Yes & Yes & Yes & Yes \\
    R-Squared & 0.986 & 0.986 & 0.986 & 0.990 & 0.990 & 0.990 \\
    Observations & 4053 & 4053 & 4053 & 3698 & 3698 & 3698 \\
    \hline
    \multicolumn{7}{l}{\footnotesize{Notes: Robust standard errors clustered at the country-pair level in parentheses.}} \\
    \multicolumn{7}{l}{\footnotesize{Significance levels are indicated as follows: $^{\ast}$p$<$0.1; $^{\ast\ast}$p$<$0.05; $^{\ast\ast\ast}$p$<$0.01.}} \\
    \end{tabular}
    \end{adjustbox}
\end{table}
%
\FloatBarrier

%
\section{Analysis and Discussion}%
\label{sec:AnalysisandDiscussion}%
This is the analysis and discussion

``Several themes emerge from this newly bourgeoning literature. First,
South--South trade and finance is now a significant economic and
political force for South countries as well as for the global economy.
There is a near consensus therefore that South--South economic relations
do matter and that they have the potential to have a significant
developmental impact. Moreover, this impact may be positive or negative,
that is, that it may help or hinder the long-term developmental goals of
exchanging parties. Second, much of South--South manufactures trade is
concentrated in high-technology-and-skill content, opening the door for
potential long-run dynamic gains from trade. However, these gains are
being increasingly concentrated within a small number of South
countries. The global South is, in fact, splitting into two groups,
which we refer to as the Emerging South and the Rest of South with very
different outcomes. While there is evidence for gains through
South--South trade, there is also evidence that the Emerging South is
rising at the expense of the Rest of South. Finally, the South--South
exchanges have expanded significantly to cover issues including
financial flows and technology transfer, among other topics. The overall
conclusion of this diverse literature is that while it does matter who
is exchanging what and with whom, South--South trade is not a panacea
for the development challenges in Southern countries. On the contrary,
South--South exchange themselves may become a potential threat for
development for some of the Southern countries.'' (Dahi \& Demir, 2017)

References

Dahi, O. S., \& Demir, F. (2017). South-South and North-South Economic
Exchanges: Does It Matter Who Is Exchanging What and with Whom?
\emph{Journal of Economic Surveys}, \emph{31}(5), 1449--1486.
https://doi.org/10.1111/joes.12225

%
\section{Conclusion}%
\label{sec:Conclusion}%
This is the conclusion

%
\newpage%
%TC:ignore%
\section{References}%
\label{sec:References}%
\printbibliography

%
%TC:endignore%
\newpage%
%TC:ignore%
\section{Appendix}%
\label{sec:Appendix}%
\subsection{Subsection in Appendix}%
\label{subsec:SubsectioninAppendix}%
Content in the appendix should not be counted in the word count.

%
%TC:endignore%
\end{document}