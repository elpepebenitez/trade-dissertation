\documentclass[12pt]{article}%
\usepackage[T1]{fontenc}%
\usepackage[utf8]{inputenc}%
\usepackage{lmodern}%
\usepackage{textcomp}%
\usepackage{lastpage}%
\usepackage[backend=biber]{biblatex}%
\usepackage{fancyhdr}%
\usepackage{lastpage}%
\usepackage{ragged2e}%
\usepackage{pdfpages}%
\usepackage{hyperref}%
\usepackage{setspace}%
\usepackage{booktabs}%
\usepackage{float}%
\usepackage{threeparttable}%
\usepackage{amssymb}%
\usepackage{amsmath}%
\usepackage{adjustbox}%
\usepackage{longtable}%
\usepackage{breqn}%
\usepackage{tabularx}%
\usepackage{graphicx}%
\usepackage{geometry}%
\usepackage{fancyhdr}%
%
\geometry{left=1in, right=1in, top=1in, bottom=1in}%
\addbibresource{references.bib}%
\hypersetup{colorlinks=true, linkcolor=blue, urlcolor=blue}%
\fancypagestyle{frontmatter}{%
\renewcommand{\headrulewidth}{0pt}%
\renewcommand{\footrulewidth}{0pt}%
\fancyhead{%
}%
\fancyfoot{%
}%
\fancyhf{}%
\fancyhead[L]{DV410}%
\fancyhead[C]{\thepage}%
\fancyhead[R]{23802}%
}%
\fancypagestyle{mainmatter}{%
\renewcommand{\headrulewidth}{0pt}%
\renewcommand{\footrulewidth}{0pt}%
\fancyhead{%
}%
\fancyfoot{%
}%
\fancyhf{}%
\fancyhead[L]{DV410}%
\fancyhead[C]{Page \thepage\ of \pageref{LastPage}}%
\fancyhead[R]{23802}%
}%
\setstretch{1.5}%
\justifying%
\pagenumbering{roman}%
\pagestyle{fancy}%
\fancyhf{}%
\fancyhead[L]{DV410}%
\fancyhead[C]{Page \thepage\ of \pageref{LastPage}}%
\fancyhead[R]{23802}%
%TC:ignore%
\title{
        \begin{flushright}
        \large \textbf{Candidate Number: 23802}
        \end{flushright}
        \vspace*{30mm}
        \begin{center}
        \large MSc in Development Management 2023 \\
        \vspace*{5mm}
        Dissertation submitted in partial fulfilment of the requirements of the degree. \\
        \vspace*{35mm}
        \Huge \textbf{Title title title} \\
        \vspace*{20mm}
        \end{center}
    }%
\date{}%
%TC:endignore%
%
\begin{document}%
\normalsize%
\includepdf[pages=-, offset=0 -1cm, frame]{DV410_Dissertation Cover Sheet_ Consent Form_Front Page_2023-24.pdf}%
\pagestyle{frontmatter}%
\maketitle%

\vfill
\begin{center}\textbf{Word Count: 3886}\end{center}
\newpage%
%TC:ignore%
\vspace*{\fill}%
\begin{center}%
\begin{minipage}{0.8\textwidth}%
\begin{center}%
\section*{Abstract}%
\end{center}%
\justify%
This is the abstract text. It should be centered on the page, and the text should be justified. This is the abstract text. It should be centered on the page, and the text should be justified. This is the abstract text. It should be centered on the page, and the text should be justified. This is the abstract text. It should be centered on the page, and the text should be justified. This is the abstract text. It should be centered on the page, and the text should be justified. %
\end{minipage}%
\end{center}%
\vspace*{\fill}%
%TC:endignore%
\newpage%
\tableofcontents%
\newpage%
%TC:ignore%
\section*{Abbreviations}%
\label{sec:Abbreviations}%
\begin{tabbing}%
\hspace{3cm} \= \kill%
\textbf{AI} \> Artificial Intelligence \\%
\textbf{API} \> Application Programming Interface \\%
\textbf{CPU} \> Central Processing Unit \\%
\textbf{GPU} \> Graphics Processing Unit \\%
\textbf{IoT} \> Internet of Things \\%
\textbf{ML} \> Machine Learning \\%
\textbf{NLP} \> Natural Language Processing \\%
\textbf{RAM} \> Random Access Memory \\%
\textbf{UI} \> User Interface \\%
\textbf{UX} \> User Experience \\%
\end{tabbing}

%
%TC:endignore%
\newpage%
%TC:ignore%
\listoffigures%
%TC:endignore%
\newpage%
%TC:ignore%
\listoftables%
%TC:endignore%
\newpage%
\pagenumbering{arabic}%
\pagestyle{mainmatter}%
\section{Introduction}%
\label{sec:Introduction}%
This is the introduction

\textbf{Stylised facts}

\textbf{Ideas:}

Total number of SS, NS and NN PTAs

Share of SS, NS and NN PTAs

Total exports by S and N countries

Share of total exports by S and N countries

Total exports of manufactured by S and N countries

Share of total exports of manufactured by S and N countries

Number of products exported by S and N countries

The organization of this article is as follows: Section II provides a
brief literature review of the PTAs, South--South trade and the
importance of the structure of trade. Section III introduces the
methodology and data. Section IV presents the empirical results fol-
lowed by a discussion of the robustness tests. Section V concludes.
%
This is the introduction section. Here is a citation: %
\cite{dahi_preferential_2013}

%
\section{Literature Review}%
\label{sec:LiteratureReview}%
This section reviews the literature on the theoretical and empirical
potential effects of PTAs on exports and welfare and situates the
analysis in the relevant field of research.

\textbf{Theoretical Framework}

Stumbling block vs building block dichotomy

\textbf{Comparative advantage and trade creation and diversion}

Traditional trade theory emphasizes trade creation (allowing cheaper
products from PTA members to substitute for more expensive domestic
products) and trade diversion (substituting products from non-PTA
members that were cheaper before the PTA with products from PTA members
that are cheaper now due to the PTA reducing tariffs) (Schiff, Winters
and Schiff, 2003) and argues that the impact of PTAs depends on the
comparative advantage of member countries. In particular, it argues that
PTAs magnify the impacts of a country's comparative advantage, relative
to the world and to other member countries signatories of a common PTA.
If member countries of a PTA have a comparative advantage on a factor
endowment relative to the world, but one country also has a comparative
advantage on the same factor endowment relative to the other member
countries, the country with the ``extreme'' advantage will be more
vulnerable to trade diversion effects, while countries with
``intermediate'' advantages will gain from trade creation effects,
predicting divergence of trade outcomes, and winners and losers among
member countries. (Venables, 2003). This emphasis on the trade creation
and trade diversion effects among member countries with significant
differences in the comparative advantage of their factor endowments
relative to the world and to each other, suggests that, when the country
with the ``extreme'' comparative advantage is a high-income country,
relative to a lower-income country with an ``intermediate'' comparative
advantage, the lower-income country should seek a PTA with the other
high-income country as it will gain more. On the contrary, if both
members are lower-income countries, the country with the ``extreme''
comparative advantage, should not seek a PTA with the other low-income
member country as it will be vulnerable. (Sanguinetti, Siedschlag and
Martincus, 2010). This logic can be easily extended to the North-South
and South-South types of PTAs, as ``North'' countries will reasonably
have an ``extreme'' comparative advantage in skill-intensive goods
relative to ``South'' countries, while ``South'' countries will
reasonably have an ``extreme'' comparative advantage in labour-intensive
goods relative to ``North'' countries. Furthermore, it is also argued in
the literature that benefiting from economies of scale through
South-South economic integration is more difficult because member
countries do not have complementary production and trade structures, nor
high interpenetration of each other's markets on intra-industry trade.
(Schiff, Winters and Schiff, 2003). Also, South countries can benefit
from greater technological diffusion from North-South PTAs as the
``North'' countries have higher industrial development as well as
investment in research (Schiff and Wang, 2008). Finally, as the trend in
manufacturing has been in favour of vertical specialization or value
chain fragmentation (Krugman, 1995), North-South PTAs are preferable as
developing countries strive to capture a greater portion of the value
added. Based on these arguments, developing countries should therefore
be better off entering into North-South rather than South-South
agreements.

\textbf{Economies of Scale, Input-Output linkages and Products Exported}

In contrast, classical development theory and new trade literature go
beyond the static welfare gains from trade creation and diversion
effects when analysing the effect of PTAs. Developing countries can use
PTAs to overcome limitations of their domestic market size in the
industrialization process (Dahi and Demir, 2013). Such potential
increases in the effective market size could help industries in
developing countries achieve economies of scale and increase the skill
content of production and exports, which in turn could improve the
market penetration of exports of developing countries in developed
markets in industrial products (Fugazza and Robert-Nicoud, 2006). Also,
due to similarities in production patterns and resource base among
developing countries, incentivising trade by lowering barriers could
facilitate appropriate technology transfer, according to the needs of
developing countries (UNIDO, 2006). Of particular relevance for
developing countries, it is argued that the products that countries
export matter for long-term economic performance. If a country exports
products from industries that are more technology-intensive, these are
likely to create input-output linkages and spillover effects in human
and physical capital accumulation and innovation (Hausmann, Hwang and
Rodrik, 2007). Furthermore, by allowing for factor accumulation, PTAs
can reduce intra-block trade barriers and increase competition and
access to cheaper intermediate goods, triggering changes in industrial
production in member countries. As such, PTAs among ``South'' countries
can reduce intra-South barriers and lead to industrialization of the
region (Puga and Venables, 1998). In this context, what matters are not
static gains from PTAs, but dynamic gains in industrial development. If
South-South PTAs truly promote industrial development of member
countries, they might be desirable even if there are short-term losses
due to trade diversion (Dahi and Demir, 2013). Other arguments in the
development literature emphasize the asymmetries in bargaining power
between ``North'' and ``South'' countries, which could lead to worse
outcomes for developing countries if their policy space gets restricted
(Thrasher and Gallagher, 2008). To the extent that these arguments hold
true, developing countries could be better off entering into South-South
rather than North-South agreements, or at least should pursue both kinds
of agreements.

\textbf{Empirical Evidence}

The preference of a type of partner in a PTAs then becomes an empirical
question. Do South-South PTAs promote trade and industrial development
among their members? The empirical literature overall reports positive
effects of PTAs on the trade of member countries, but with considerable
heterogeneity on the estimation coefficients. For example, a
meta-analysis of research papers on the effects of PTAs on member trade,
encompassing 85 papers and 1827 estimates, finds an average of 0.59 (an
80\% increase in trade), with a median of 0.38 (a 46\% increase in
trade), a wide range of coefficient estimates (-9.01 to 15.41), and only
312 out of 1827 estimates reported as negative (Cipollina and Salvatici,
2010). Furthermore, a survey of the empirical research on the effect of
economic integration agreements on international trade flows, as well as
using the most modern econometric techniques to address biases, found an
increase of 50\% on international trade, but with significant variation
in the effects of specific agreements (Kohl, 2014). However, much of the
empirical research is focused on the effects of PTAs on or including the
most advanced economies. Empirical research focused exclusively on the
effects of South-South PTAs or comparing them to the effects of
North-North or North-South PTAs, is much less prevalent in the
literature. However, several research papers do control for the type of
agreement (North-South or South-South) and have found positive and
significant effects of South-South PTAs (Medvedev, 2006; Mayda and
Steinberg, 2007; Dahi and Demir, 2013; Deme and Ndrianasy, 2017), but
these articles tend to be limited in their scope, sample size or only
focus on trade volumes.

\textbf{Significance of Exports}

%
\section{Methodology}%
\label{sec:Methodology}%
\textbf{The Gravity Model of Trade}

Often referred as the ``workhorse'' of international trade, the gravity
model is prominent in the empirical literature of applied international
trade analysis. Among the arguments that could explain/support the use
of the gravity model, there are four that are particularly relevant for
our purposes. First, the gravity model of trade is intuitive to
understand. Following the metaphor of Newton's Law of Universal
Gravitation, it predicts that international trade between two countries
is directly proportional to the product of their economic size, and
inversely proportional to trade frictions between them. In simpler
words, the bigger (smaller) the economies of two countries, and the
easier (harder) it is for them to trade with each other, the more (less)
we expect them to trade. Second, it is referred to as a structural model
with solid theoretical foundations, which makes it appropriate for
counterfactual analysis, such as measuring the effects of trade policies
as we aim to do with the effects of South-North versus South-South
agreements. Third, model has a flexible structure, which will allow us
to construct a specification tailored to our research. Finally, fourth,
it holds consistent and remarkable predictive power, both with aggregate
and sectoral data (Yotov et al. 2016).

Through the decades, the gravity equation has been regularly upgraded in
the theoretical and empirical literature. Of relevance, the simple
intuition of the gravity model was theoretically extended by Anderson to
note that, after controlling for size, the increase or decrease is
\emph{relative} to the average barriers of the two countries with all
their partners, which are referred as ``multilateral resistance''
(Anderson 1979). The more trade barriers or resistance to trade exists
with other countries relative to a given partner, the more a country is
pushed to trade with said partner. Anderson also introduced the
assumptions of product differentiation by place of origin, and Constant
Elasticity of Substitution (CES) expenditures ,or the Armington-CES
assumption (Yotov et al. 2016; Chatzilazarou and Dadakas 2023), which
led us to today's generalized form of the gravity equation, as developed
and popularised by Anderson and van Wincoop (Anderson and van Wincoop
2003).

Equally important, several empirical developments have strengthened the
gravity model and inform our choice of methodology: Exporter-time and
importer-time fixed effects are used to account for the multilateral
resistance terms in a gravity estimation with panel data (Olivero and
Yotov 2012); As the gravity model is often estimated with an OSL
estimator, zero-trade flows were dropped from the sample when trade was
transformed into a logarithmic form. Also, trade data is recognized to
suffer from heteroscedasticity (Yotov et al. 2016). To solve for
zero-trade flows and heteroscedasticity, the Poisson Pseudo Maximum
Likelihood (PPML) estimator has been proposed to estimate the gravity
model, avoiding potential biases (Silva and Tenreyro 2006; Santos Silva
and Tenreyro 2011); Country-pair fixed effects has been proposed to
account for the unobserved endogeneity of trade policy (Baier and
Bergstrand 2007). It is worth nothing that the inclusion of
exporter-time and importer-time fixed effects will absorb all observable
and unobservable time-varying country-specific characteristics that
could affect the dependent variable, while the country-pair fixed
effects will absorb observable and unobservable bilateral time-invariant
characteristics that could affect trade costs; The inclusion of
intra-trade flows as well as international trade flows is proposed to
correctly estimate the effects of non-discriminatory trade policy,
allowing for consumers to choose products from both international and
domestic sources (Dai, Yotov, and Zylkin 2014; Heid, Larch, and Yotov
2017); Year-intervals instead of data pooled over consecutive years
should be used to allow for adjustment of trade flows to policies that
might not have immediate effects (Baier and Bergstrand 2007; Anderson
and Yotov 2016); And finally, to account for the effects of
globalization forces that may biased the estimates of trade policies, a
set of globalization dummies are recommended to control for the effects
of globalization in the gravity model (Yotov 2012; Bergstrand, Larch,
and Yotov 2015). Based on the theoretical and empirical best-practices
found in the relevant literature, we employ the following gravity
equation using a PPML estimator and a balanced panel data approach with
multiple exporters, multiple importers and time as our benchmark model:

% \[(1)\ \ \ \ \ X_{ij,t} = \ exp(\eta_{i,t} + \psi_{j,t} + \gamma_{\binom{-}{ij}} + \beta_{1}{PTA}_{ij,t} + \beta_{2}{PTA}_{ij,t - 5} + \sum_{t}^{}b_{t}) + \epsilon_{ij,t}\]
\begin{multline}
    X_{ij,t} = \exp\left(\eta_{i,t} + \psi_{j,t} + \gamma_{\binom{-}{ij}} + \beta_{1} \, PTA_{ij,t} \right. + \beta_{2} \, PTA_{ij,t-5} + \left. \sum_{t} b_{t} \right) + \epsilon_{ij,t}
\end{multline}

Where \(X_{ij,t}\) denotes the value of exports from an origin country
\(i\) to a destination country \(j\); \(\eta_{i,t}\) and \(\psi_{j,t}\)
are, respectively, exporter-time and importer-time fixed-effects;
\(\gamma_{\binom{-}{ij}}\) is a country-pair fixed-effect;
\({PTA}_{ij,t}\) and \({PTA}_{ij,t - 5}\) are our main variables of
interest, which, respectively indicate if \(i\) and \(j\) are members of
a PTA at time \(t\) and, to account for potential ``phase-in'' effects
over time of the PTA, at time \(t - 5\); \(\sum_{t}^{}b_{t}\) is a set
of dummies that equal 1 for international trade and 0 for domestic trade
observations at each time \(t\); and \(\epsilon_{ij,t}\) is an error
term.

In contrast with our main interest of research, which are the potential
heterogenous effects of PTAs on different members for different types of
agreements, this benchmark model, specifically
\(\beta = \beta_{1} + \beta_{2}\) , would provide the average ``total''
partial effect of PTAs on trade after accounting for lagged effects, but
it cannot provide the effects for a given agreement, country-pairs o
specific country members to a specific agreement. As such, three
successive expansions can be implemented to capture heterogeneity in PTA
effects as proposed by Baier \emph{et al}. (Baier, Yotov, and Zylkin
2019):

% \[(2)\ \ \ \ \ X_{ij,t} = \ exp(\eta_{i,t} + \psi_{j,t} + \gamma_{\binom{-}{ij}} + \sum_{A}^{}{\beta_{1,A}{PTA}_{ij,t}} + \sum_{A}^{}{\beta_{2,A}{PTA}_{ij,t - 5}} + \sum_{t}^{}b_{t}) + \epsilon_{ij,t}\]
\begin{multline}
    X_{ij,t} = \exp\left(\eta_{i,t} + \psi_{j,t} + \gamma_{\binom{-}{ij}} + \sum_{A} \beta_{1,A} \, PTA_{ij,t} \right. + \sum_{A} \beta_{2,A} \, PTA_{ij,t-5} + \left. \sum_{t} b_{t} \right) + \epsilon_{ij,t}
\end{multline}

Equation (2) can be implemented to account for heterogeneous effects of
PTAs at the level of the specific agreement, by allowing for distinct
average partial effects for each individual agreement, using superscript
\(A\) to index by agreement and also allowing for agreement-specific
lags: \(\beta_{A} = \beta_{1,A} + \beta_{2,A}\).

In order to analyse the differentiated effects of North-North,
North-South and South-South PTAs, we extend both models to get estimates
for each type of PTA. Our benchmark model is extended as follows:

% \[(3)\ \ \ \ \ X_{ij,t} = \ exp(\eta_{i,t} + \psi_{j,t} + \gamma_{\binom{-}{ij}} + \beta_{1NN}{PTA\_ NN}_{ij,t} + \beta_{2NN}{PTA\_ NN}_{ij,t - 5} + \beta_{1NS}{PTA\_ NS}_{ij,t} + \beta_{2NS}{PTA\_ NS}_{ij,t - 5} + \beta_{1SS}{PTA\_ SS}_{ij,t} + \beta_{2SS}{PTA\_ SS}_{ij,t - 5} + \sum_{t}^{}b_{t}) + \epsilon_{ij,t}\]
\begin{multline}
    X_{ij,t} = \exp\left(\eta_{i,t} + \psi_{j,t} + \gamma_{\binom{-}{ij}} + \beta_{1NN} \, PTA\_NN_{ij,t} + \beta_{2NN} \, PTA\_NN_{ij,t-5} \right. \\
    + \beta_{1NS} \, PTA\_NS_{ij,t} + \beta_{2NS} \, PTA\_NS_{ij,t-5} + \beta_{1SS} \, PTA\_SS_{ij,t} + \beta_{2SS} \, PTA\_SS_{ij,t-5} \\
    + \left. \sum_{t} b_{t} \right) + \epsilon_{ij,t}
\end{multline}

Where \(X_{ij,t}\)\hspace{0pt} denotes the value of exports from country
\(i\) to country \(j\) at time \(t\); \(\eta_{i,t}\) and
\(\psi_{j,t}\ \)are exporter-time and importer-time fixed effects,
respectively; \(\gamma_{\binom{-}{ij}}\) is a country-pair fixed effect;
\hspace{0pt}\(\beta_{1NN}\) and \(\beta_{2NN}\) are the coefficients for
the immediate and lagged effects of a North-North PTA (\(PTA\_ NN\));
\hspace{0pt}\hspace{0pt}\(\beta_{1NS}\) and \(\beta_{2NS}\) are the
coefficients for the immediate and lagged effects of a North-South PTA
(\(PTA\_ SN\)); \hspace{0pt}\hspace{0pt}\(\beta_{1SS}\) and
\(\beta_{2SS}\) are the coefficients for the immediate and lagged
effects of a South-South PTA (\(PTA\_ SS\)); \(\sum_{t}^{}b_{t}\) is a
set of time dummies accounting for international trade-specific effects
at each time \(t\); and \(\epsilon_{ij,t}\) is the error term.

Equation (2) also gets extended to capture the heterogeneous effects of
the different types of PTAs as follows:

% \[(4)\ \ \ \ \ X_{ij,t} = \ exp(\eta_{i,t} + \psi_{j,t} + \gamma_{\binom{-}{ij}} + \sum_{A}^{}{(\beta_{1,A,NN}{PTA\_ NN}_{ij,t}\  + \ \beta_{2,A,NN}{PTA\_ NN}_{ij,t - 5})} + \sum_{A}^{}{(\beta_{1,A,NS}{PTA\_ NS}_{ij,t}\  + \ \beta_{2,A,NS}{PTA\_ NS}_{ij,t - 5})} + \sum_{A}^{}{(\beta_{1,A,SS}{PTA\_ SS}_{ij,t}\  + \ \beta_{2,A,SS}{PTA\_ SS}_{ij,t - 5})} + \sum_{t}^{}b_{t}) + \epsilon_{ij,t}\]

\begin{multline}
    X_{ij,t} = \exp\left(\eta_{i,t} + \psi_{j,t} + \gamma_{\binom{-}{ij}} + \sum_{A}\left(\beta_{1,A,NN} \, PTA\_NN_{ij,t} + \beta_{2,A,NN} \, PTA\_NN_{ij,t-5}\right) \right. \\
    + \sum_{A}\left(\beta_{1,A,NS} \, PTA\_NS_{ij,t} + \beta_{2,A,NS} \, PTA\_NS_{ij,t-5}\right) + \sum_{A}\left(\beta_{1,A,SS} \, PTA\_SS_{ij,t} + \beta_{2,A,SS} \, PTA\_SS_{ij,t-5}\right) \\
    + \left. \sum_{t} b_{t} \right) + \epsilon_{ij,t}
\end{multline}

Where \(X_{ij,t}\)\hspace{0pt} denotes the value of exports from country
\(i\) to country \(j\) at time \(t\); \(\eta_{i,t}\) and
\(\psi_{j,t}\ \)are exporter-time and importer-time fixed effects,
respectively; \(\gamma_{\binom{-}{ij}}\) is a country-pair fixed effect;
The summations \hspace{0pt}\(\sum_{}^{}A\) denote the sum over different
agreements \(A\) for: \(\beta_{1,A,NN}\) and \(\beta_{2,A,NN}\):
Coefficients for the immediate and lagged effects of North-North PTAs
\hspace{0pt}(\(PTA\_ NN\)); \(\beta_{1,A,NS}\) and \(\beta_{2,A,NS}\):
Coefficients for the immediate and lagged effects of North-South PTAs
(\(PTA\_ SN\)); \(\beta_{1,A,SS}\) and \(\beta_{2,A,SS}\): Coefficients
for the immediate and lagged effects of South-South PTAs (\(PTA\_ SS\));
\(\sum_{t}^{}b_{t}\) is a set of time dummies accounting for
trade-specific effects at each time \(t\); and \(\epsilon_{ij,t}\) is
the error term.

For both extended models we use the following variables:
\({PTA\_ NN}_{ij,t}\) is a dummy variable that takes the value of 1 if
the trade pair \((i,j)\) is part of a North-North PTA at time \(t\), and
0 otherwise; \({PTA\_ NN}_{ij,t - 5}\) is a dummy variable that takes
the value of 1 if the trade pair \((i,j)\) was part of a North-North PTA
at time \(t\)\emph{-5}, and 0 otherwise; \({PTA\_ NS}_{ij,t}\) is a
dummy variable that takes the value of 1 if the trade pair \((i,j)\) is
part of a North-South PTA at time \(t\), and 0 otherwise;
\({PTA\_ NS}_{ij,t - 5}\) is a dummy variable that takes the value of 1
if the trade pair \((i,j)\) was part of a North-South PTA at time
\(t\)\emph{-5}, and 0 otherwise; \({PTA\_ SS}_{ij,t}\) is a dummy
variable that takes the value of 1 if the trade pair \((i,j)\) is part
of a South-South PTA at time \(t\), and 0 otherwise;
\({PTA\_ SS}_{ij,t - 5}\) is a dummy variable that takes the value of 1
if the trade pair \((i,j)\) was part of a South-South PTA at time
\(t\)\emph{-5}, and 0 otherwise;

The extended models allow us to capture the differentiated effects of
PTAs on bilateral exports depending on whether they are between two
``North'' countries (NN), between a ``North'' and a ``South'' country
(NS), or between two ``South'' countries (SS).

\textbf{Defining South and North}

\textbf{Data}

%
\section{Findings}%
\label{sec:Findings}%
These are the findings

%
\subsection{Benchmark Estimation Results by Region}%
\label{subsec:BenchmarkEstimationResultsbyRegion}%
\subsubsection{Benchmark Model Results by Region}%
\label{ssubsec:BenchmarkModelResultsbyRegion}%
\begin{table}[htbp]
    \centering
    \caption{Benchmark Model Regional Results}
    \label{tab:benchmark_region_analysis} % This allows you to reference the table in the text with \ref{tab:pta_analysis}
    \begin{adjustbox}{max width=\textwidth}
    \begin{tabular}{l@{\extracolsep{1pt}}ccccc}
    \hline
    & \multicolumn{1}{c}{(1)} & \multicolumn{1}{c}{(2)} & \multicolumn{1}{c}{(3)} & \multicolumn{1}{c}{(4)} & \multicolumn{1}{c}{(5)} \\
    \hline
    \textbf{Variables} &  &  &  &  &  \\
    \hline
     & PPML & PPML & PPML & PPML & PPML \\
     & Africa & Americas & Asia & Europe & Intercontinental \\
    \hline
    PTA & 0.578$^{\ast\ast\ast}$ & 0.287$^{\ast\ast\ast}$ & 0.064 & 0.237$^{\ast\ast\ast}$ & 0.015 \\
    & (0.154) & (0.071) & (0.083) & (0.019) & (0.093) \\

    PTA Lag & -0.278 & 0.146 & -0.167$^{\ast\ast\ast}$ & 0.238$^{\ast\ast\ast}$ & 0.188$^{\ast\ast\ast}$ \\
    & (0.300) & (0.149) & (0.056) & (0.022) & (0.043) \\

    PTA + PTA Lag & 0.301 & 0.433$^{\ast\ast\ast}$ & -0.103 & 0.475$^{\ast\ast\ast}$ & 0.203$^{\ast}$ \\
    & (0.295) & (0.140) & (0.094) & (0.025) & (0.106) \\
    \hline
    Exporter-Year FE & Yes & Yes & Yes & Yes & Yes \\
    Importer-Year FE & Yes & Yes & Yes & Yes & Yes \\
    Country-Pair FE & Yes & Yes & Yes & Yes & Yes \\
    R-Squared & 0.997 & 0.999 & 0.999 & 0.997 & 0.998 \\
    Observations & 5838 & 10997 & 25308 & 28168 & 73930 \\
    \hline
    \multicolumn{6}{l}{\footnotesize{Notes: Robust standard errors clustered at the country-pair in parentheses. Significance levels are indicated as follows: $^{\ast}$p$<$0.1; $^{\ast\ast}$p$<$0.05; $^{\ast\ast\ast}$p$<$0.01.}} \\
    \end{tabular}
    \end{adjustbox}
\end{table}    

%
\subsection{PTA Estimation Results by Region}%
\label{subsec:PTAEstimationResultsbyRegion}%
\subsubsection{PTA Model Results by Region}%
\label{ssubsec:PTAModelResultsbyRegion}%
\begin{table}[htbp]
    \centering
    \caption{TA + TA Lag Coefficients for Africa Region}
    \label{tab:pta_africa}
    \begin{adjustbox}{max width=\textwidth}
    \begin{tabular}{lcc}
    \hline
    \textbf{Statistically Insignificant} &  &  \\
    \hline
    TA ID & Estimate & SE \\
    \hline
    670 & 0.326 & (0.410) \\
    787 & 0.304 & (0.233) \\
    \hline
    Exporter-Year FE & Yes \\
    Importer-Year FE & Yes \\
    Country-Pair FE & Yes \\
    R-Squared & 0.997 \\
    Observations & 5838 \\
    \hline
    \multicolumn{3}{l}{\footnotesize{Notes: Robust standard errors clustered at the country-pair level in parentheses.}} \\
    \multicolumn{3}{l}{\footnotesize{Significance levels are indicated as follows: $^{\ast}$p$<$0.1; $^{\ast\ast}$p$<$0.05; $^{\ast\ast\ast}$p$<$0.01.}} \\
    \end{tabular}
    \end{adjustbox}
\end{table}
%
\begin{table}[htbp]
    \centering
    \caption{TA + TA Lag Coefficients for Americas Region}
    \label{tab:pta_americas}
    \begin{adjustbox}{max width=\textwidth}
    \begin{tabular}{lcc}
    \hline
    \textbf{Positive and Statistically Significant} &  &  \\
    \hline
    TA ID & Estimate & SE \\
    \hline
    213 & 1.342$^{\ast\ast\ast}$ & (0.434) \\
    218 & 0.879$^{\ast\ast\ast}$ & (0.173) \\
    239 & 0.571$^{\ast\ast\ast}$ & (0.173) \\
    616 & 0.488$^{\ast\ast\ast}$ & (0.044) \\
    168 & 0.410$^{\ast\ast\ast}$ & (0.113) \\
    163 & 0.342$^{\ast\ast\ast}$ & (0.096) \\
    141 & 0.265$^{\ast\ast\ast}$ & (0.024) \\
    716 & 0.732$^{\ast\ast}$ & (0.358) \\
    201 & 0.545$^{\ast\ast}$ & (0.265) \\
    612 & 0.515$^{\ast\ast}$ & (0.251) \\
    \hline
    \textbf{Statistically Insignificant} &  &  \\
    \hline
    TA ID & Estimate & SE \\
    \hline
    185 & 0.291 & (0.376) \\
    645 & 0.117 & (0.141) \\
    \hline
    \textbf{Negative and Statistically Significant} &  &  \\
    \hline
    TA ID & Estimate & SE \\
    \hline
    188 & -0.774$^{\ast\ast\ast}$ & (0.144) \\
    \hline
    Exporter-Year FE & Yes \\
    Importer-Year FE & Yes \\
    Country-Pair FE & Yes \\
    R-Squared & 0.999 \\
    Observations & 10997 \\
    \hline
    \multicolumn{3}{l}{\footnotesize{Notes: Robust standard errors clustered at the country-pair level in parentheses.}} \\
    \multicolumn{3}{l}{\footnotesize{Significance levels are indicated as follows: $^{\ast}$p$<$0.1; $^{\ast\ast}$p$<$0.05; $^{\ast\ast\ast}$p$<$0.01.}} \\
    \end{tabular}
    \end{adjustbox}
\end{table}
%
\begin{table}[htbp]
    \centering
    \caption{PTA + PTA Lag Coefficients for Asia Region}
    \label{tab:pta_asia}
    \begin{adjustbox}{max width=\textwidth}
    \begin{tabular}{lcc}
    \hline
    \textbf{Positive and Statistically Significant} &  &  \\
    \hline
    PTA ID & Estimate & SE \\
    \hline
    683 & 1.080$^{\ast\ast\ast}$ & (0.237) \\
    70  & 0.472$^{\ast\ast\ast}$ & (0.150) \\
    100 & 0.376$^{\ast\ast\ast}$ & (0.105) \\
    67  & 0.342$^{\ast\ast\ast}$ & (0.125) \\
    675 & 1.360$^{\ast\ast}$ & (0.655) \\
    475 & 0.636$^{\ast\ast}$ & (0.298) \\
    598 & 0.166$^{\ast\ast}$ & (0.083) \\
    474 & 0.419$^{\ast}$ & (0.243) \\
    \hline
    \textbf{Statistically Insignificant} &  &  \\
    \hline
    PTA ID & Estimate & SE \\
    \hline
    72  & 0.254 & (0.178) \\
    116 & 0.256 & (0.703) \\
    492 & 0.041 & (0.180) \\
    640 & 0.183 & (0.217) \\
    223 & -0.014 & (0.203) \\
    71  & -0.138 & (0.091) \\
    456 & -0.209 & (0.165) \\
    534 & -0.165 & (0.370) \\
    667 & -0.049 & (0.241) \\
    \hline
    \textbf{Negative and Statistically Significant} &  &  \\
    \hline
    PTA ID & Estimate & SE \\
    \hline
    221 & -2.955$^{\ast\ast\ast}$ & (0.727) \\
    220 & -1.215$^{\ast\ast\ast}$ & (0.093) \\
    599 & -0.967$^{\ast\ast\ast}$ & (0.191) \\
    1   & -0.732$^{\ast\ast}$ & (0.359) \\
    \hline
    Exporter-Year FE & Yes \\
    Importer-Year FE & Yes \\
    Country-Pair FE & Yes \\
    R-Squared & 0.999 \\
    Observations & 25308 \\
    \hline
    \multicolumn{3}{l}{\footnotesize{Notes: Robust standard errors clustered at the country-pair level in parentheses.}} \\
    \multicolumn{3}{l}{\footnotesize{Significance levels are indicated as follows: $^{\ast}$p$<$0.1; $^{\ast\ast}$p$<$0.05; $^{\ast\ast\ast}$p$<$0.01.}} \\
    \end{tabular}
    \end{adjustbox}
\end{table}
%
\begin{table}[htbp]
    \centering
    \caption{PTA + PTA Lag Coefficients for Europe Region}
    \label{tab:pta_europe}
    \begin{adjustbox}{max width=\textwidth}
    \begin{tabular}{lcc}
    \hline
    \textbf{Positive and Statistically Significant} &  &  \\
    \hline
    PTA ID & Estimate & SE \\
    \hline
    5   & 3.812$^{\ast\ast\ast}$ & (0.278) \\
    128 & 2.712$^{\ast\ast\ast}$ & (0.211) \\
    13  & 2.256$^{\ast\ast\ast}$ & (0.262) \\
    132 & 2.241$^{\ast\ast\ast}$ & (0.252) \\
    192 & 1.107$^{\ast\ast\ast}$ & (0.163) \\
    7   & 1.153$^{\ast\ast\ast}$ & (0.272) \\
    328 & 0.671$^{\ast\ast\ast}$ & (0.175) \\
    8   & 0.667$^{\ast\ast\ast}$ & (0.161) \\
    621 & 0.618$^{\ast\ast\ast}$ & (0.186) \\
    135 & 0.615$^{\ast\ast\ast}$ & (0.217) \\
    254 & 0.565$^{\ast\ast\ast}$ & (0.084) \\
    394 & 0.745$^{\ast\ast\ast}$ & (0.202) \\
    335 & 0.472$^{\ast\ast\ast}$ & (0.025) \\
    9   & 0.580$^{\ast\ast}$ & (0.285) \\
    11  & 0.656$^{\ast\ast}$ & (0.307) \\
    131 & 0.615$^{\ast\ast}$ & (0.281) \\
    129 & 0.553$^{\ast\ast\ast}$ & (0.206) \\
    \hline
    \textbf{Statistically Insignificant} &  &  \\
    \hline
    PTA ID & Estimate & SE \\
    \hline
    6   & 0.355 & (0.358) \\
    150 & 0.247 & (0.687) \\
    153 & 0.614 & (0.633) \\
    154 & 0.592 & (0.409) \\
    255 & 0.167 & (0.237) \\
    389 & 0.412 & (0.323) \\
    331 & 0.142 & (0.201) \\
    594 & 0.474$^{\ast}$ & (0.251) \\
    12  & -0.246 & (1.208) \\
    156 & -0.441 & (0.445) \\
    \hline
    \textbf{Negative and Statistically Significant} &  &  \\
    \hline
    PTA ID & Estimate & SE \\
    \hline
    133 & -0.772$^{\ast\ast\ast}$ & (0.248) \\
    \hline
    \multicolumn{3}{l}{\footnotesize{Notes: Robust standard errors clustered at the country-pair level in parentheses.}} \\
    \multicolumn{3}{l}{\footnotesize{Significance levels are indicated as follows: $^{\ast}$p$<$0.1; $^{\ast\ast}$p$<$0.05; $^{\ast\ast\ast}$p$<$0.01.}} \\
    \end{tabular}
    \end{adjustbox} 
\end{table}
%
\begin{center}
\small
\begin{longtable}{lcc}
    \caption{PTA+PTALag Coefficients for Intercontinental} \label{tab:pta_intercontinental} \\
    
    \hline
    \textbf{Positive and Statistically Significant} &  &  \\
    \hline
    PTA ID & Estimate & SE \\
    \hline
    \endfirsthead
    
    \multicolumn{3}{c}{{\bfseries \tablename\ \thetable{} -- continued from previous page}} \\
    \hline
    PTA ID & Estimate & SE \\
    \hline
    \endhead
    
    \hline \multicolumn{3}{r}{{Continued on next page}} \\ \hline
    \endfoot
    
    \hline
    \endlastfoot
    
    627 & 2.372$^{\ast\ast\ast}$ & (0.345) \\
    415 & 1.853$^{\ast\ast\ast}$ & (0.201) \\
    206 & 1.539$^{\ast\ast\ast}$ & (0.180) \\
    75  & 1.366$^{\ast\ast\ast}$ & (0.493) \\
    263 & 1.426$^{\ast\ast\ast}$ & (0.115) \\
    4   & 1.254$^{\ast\ast\ast}$ & (0.268) \\
    626 & 1.099$^{\ast\ast\ast}$ & (0.121) \\
    657 & 0.705$^{\ast\ast\ast}$ & (0.082) \\
    637 & 0.667$^{\ast\ast\ast}$ & (0.102) \\
    202 & 0.658$^{\ast\ast\ast}$ & (0.123) \\
    208 & 0.763$^{\ast\ast\ast}$ & (0.129) \\
    136 & 0.744$^{\ast\ast\ast}$ & (0.185) \\
    490 & 0.843$^{\ast\ast\ast}$ & (0.181) \\
    17  & 0.811$^{\ast\ast\ast}$ & (0.242) \\
    466 & 0.710$^{\ast\ast\ast}$ & (0.147) \\
    304 & 0.770$^{\ast\ast\ast}$ & (0.120) \\
    628 & 0.484$^{\ast\ast\ast}$ & (0.142) \\
    207 & 0.516$^{\ast\ast\ast}$ & (0.114) \\
    518 & 0.627$^{\ast\ast\ast}$ & (0.135) \\
    330 & 0.314$^{\ast\ast\ast}$ & (0.086) \\
    164 & 0.288$^{\ast\ast\ast}$ & (0.073) \\
    96 & 0.271$^{\ast\ast\ast}$ & (0.055) \\
    181 & 0.392$^{\ast\ast}$ & (0.178) \\
    624 & 0.388$^{\ast\ast}$ & (0.163) \\
    521 & 0.101$^{\ast\ast}$ & (0.045) \\
    384  & 0.645$^{\ast}$ & (0.355) \\
    15  & 0.313$^{\ast}$ & (0.179) \\
    227 & 0.348$^{\ast}$ & (0.186) \\
    \hline
    \textbf{Statistically Insignificant} &  &  \\
    \hline
    PTA ID & Estimate & SE \\
    \hline
    641 & 2.028 & (1.255) \\
    543 & 1.090 & (0.707) \\
    509 & 0.210 & (0.216) \\
    252 & 0.192 & (0.357) \\
    508 & 0.140 & (0.122) \\
    376 & 0.172 & (0.228) \\
    416 & 0.424 & (0.295) \\
    401 & 0.407 & (0.288) \\
    152 & 0.110 & (0.266) \\
    242 & 0.050 & (0.294) \\
    390 & 0.0471 & (0.181) \\
    396 & 0.019 & (0.379) \\
    205 & 0.0012 & (0.178) \\
    602 & -0.076 & (0.918) \\
    383 & -0.202 & (0.152) \\
    386 & -0.092 & (0.168) \\
    84  & -0.059 & (0.120) \\
    979  & -0.126 & (0.294) \\
    644  & -0.189 & (0.122) \\
    658  & -0.303 & (0.349) \\
    \hline
    \textbf{Negative and Statistically Significant} &  &  \\
    \hline
    PTA ID & Estimate & SE \\
    \hline
    399 & -0.473$^{\ast\ast\ast}$ & (0.127) \\
    104 & -0.338$^{\ast\ast\ast}$ & (0.112) \\
    677 & -1.366$^{\ast\ast\ast}$ & (0.385) \\
    679 & -1.429$^{\ast\ast\ast}$ & (0.430) \\
    323 & -0.338$^{\ast\ast}$ & (0.138) \\
    512 & -0.458$^{\ast}$ & (0.266) \\
    \hline
    Exporter-Year FE & Yes \\
    Importer-Year FE & Yes \\
    Country-Pair FE & Yes \\
    R-Squared & 0.998 \\
    Observations & 73930 \\
    \hline
    \multicolumn{3}{l}{\footnotesize{Notes: Robust standard errors clustered at the country-pair level in parentheses.}} \\
    \multicolumn{3}{l}{\footnotesize{Significance levels are indicated as follows: $^{\ast}$p$<$0.1; $^{\ast\ast}$p$<$0.05; $^{\ast\ast\ast}$p$<$0.01.}} \\
\end{longtable}
\end{center}


% \begin{table}[htbp]
%     \centering
%     \caption{PTA + PTA Lag Coefficients for Intercontinental Region}
%     \label{tab:pta_intercontinental}
%     \begin{minipage}[t]{0.48\textwidth} % Adjust the width as needed
%         \centering
%         \begin{adjustbox}{max width=\textwidth}
%         \begin{tabular}{lcc}
%         \hline
%         \textbf{Positive and Statistically Significant} &  &  \\
%         \hline
%         PTA ID & Estimate & SE \\
%         \hline
%         627 & 2.372$^{\ast\ast\ast}$ & (0.345) \\
%         415 & 1.853$^{\ast\ast\ast}$ & (0.201) \\
%         206 & 1.539$^{\ast\ast\ast}$ & (0.180) \\
%         75  & 1.366$^{\ast\ast\ast}$ & (0.493) \\
%         263 & 1.426$^{\ast\ast\ast}$ & (0.115) \\
%         4   & 1.254$^{\ast\ast\ast}$ & (0.268) \\
%         626 & 1.099$^{\ast\ast\ast}$ & (0.121) \\
%         657 & 0.705$^{\ast\ast\ast}$ & (0.082) \\
%         637 & 0.667$^{\ast\ast\ast}$ & (0.102) \\
%         202 & 0.658$^{\ast\ast\ast}$ & (0.123) \\
%         208 & 0.763$^{\ast\ast\ast}$ & (0.129) \\
%         136 & 0.744$^{\ast\ast\ast}$ & (0.185) \\
%         490 & 0.843$^{\ast\ast\ast}$ & (0.181) \\
%         17  & 0.811$^{\ast\ast\ast}$ & (0.242) \\
%         466 & 0.710$^{\ast\ast\ast}$ & (0.147) \\
%         304 & 0.770$^{\ast\ast\ast}$ & (0.120) \\
%         628 & 0.484$^{\ast\ast\ast}$ & (0.142) \\
%         207 & 0.516$^{\ast\ast\ast}$ & (0.114) \\
%         518 & 0.627$^{\ast\ast\ast}$ & (0.135) \\
%         330 & 0.314$^{\ast\ast\ast}$ & (0.086) \\
%         164 & 0.288$^{\ast\ast\ast}$ & (0.073) \\
%         96 & 0.271$^{\ast\ast\ast}$ & (0.055) \\
%         181 & 0.392$^{\ast\ast}$ & (0.178) \\
%         624 & 0.388$^{\ast\ast}$ & (0.163) \\
%         521 & 0.101$^{\ast\ast}$ & (0.045) \\
%         384  & 0.645$^{\ast}$ & (0.355) \\
%         15  & 0.313$^{\ast}$ & (0.179) \\
%         227 & 0.348$^{\ast}$ & (0.186) \\
%         \hline
%         \end{tabular}
%         \end{adjustbox}
%     \end{minipage}%
%     \hfill
%     \begin{minipage}[t]{0.48\textwidth} % Adjust the width as needed
%         \centering
%         \begin{adjustbox}{max width=\textwidth}
%         \begin{tabular}{lcc}
%         \hline
%         \textbf{Statistically Insignificant} &  &  \\
%         \hline
%         PTA ID & Estimate & SE \\
%         \hline
%         641 & 2.028 & (1.255) \\
%         543 & 1.090 & (0.707) \\
%         509 & 0.210 & (0.216) \\
%         252 & 0.192 & (0.357) \\
%         508 & 0.140 & (0.122) \\
%         376 & 0.172 & (0.228) \\
%         416 & 0.424 & (0.295) \\
%         401 & 0.407 & (0.288) \\
%         152 & 0.110 & (0.266) \\
%         242 & 0.050 & (0.294) \\
%         390 & 0.0471 & (0.181) \\
%         396 & 0.019 & (0.379) \\
%         205 & 0.0012 & (0.178) \\
%         602 & -0.076 & (0.918) \\
%         383 & -0.202 & (0.152) \\
%         386 & -0.092 & (0.168) \\
%         84  & -0.059 & (0.120) \\
%         979  & -0.126 & (0.294) \\
%         644  & -0.189 & (0.122) \\
%         658  & -0.303 & (0.349) \\
%         \hline
%         \textbf{Negative and Statistically Significant} &  &  \\
%         \hline
%         PTA ID & Estimate & SE \\
%         \hline
%         399 & -0.473$^{\ast\ast\ast}$ & (0.127) \\
%         104 & -0.338$^{\ast\ast\ast}$ & (0.112) \\
%         677 & -1.366$^{\ast\ast\ast}$ & (0.385) \\
%         679 & -1.429$^{\ast\ast\ast}$ & (0.430) \\
%         323 & -0.338$^{\ast\ast}$ & (0.138) \\
%         512 & -0.458$^{\ast}$ & (0.266) \\
%         \hline
%         \end{tabular}
%         \end{adjustbox}
%     \end{minipage}
%     \vspace{0.5cm} % Optional: Add space between the tables and notes
%     \begin{adjustbox}{max width=\textwidth}
%     \begin{tabular}{l}
%     \hline
%     Exporter-Year FE & Yes \\
%     Importer-Year FE & Yes \\
%     Country-Pair FE & Yes \\
%     R-Squared & 0.998 \\
%     Observations & 73930 \\
%     \hline
%     \multicolumn{3}{l}{\footnotesize{Notes: Robust standard errors clustered at the country-pair level in parentheses.}} \\
%     \multicolumn{3}{l}{\footnotesize{Significance levels are indicated as follows: $^{\ast}$p$<$0.1; $^{\ast\ast}$p$<$0.05; $^{\ast\ast\ast}$p$<$0.01.}} \\
%     \end{tabular}
%     \end{adjustbox}
% \end{table}

%
\subsection{NS Estimation Results by Region}%
\label{subsec:NSEstimationResultsbyRegion}%
\subsubsection{NS Model Results by Region}%
\label{ssubsec:NSModelResultsbyRegion}%
\begin{table}[htbp]
    \centering
    \caption{Regional Results by PTA Type}
    \label{tab:pta_types}
    \begin{adjustbox}{max width=\textwidth}
    \begin{tabular}{lccccc}
    \hline
     & \multicolumn{1}{c}{Africa} & \multicolumn{1}{c}{Americas} & \multicolumn{1}{c}{Asia} & \multicolumn{1}{c}{Europe} & \multicolumn{1}{c}{Intercontinental} \\
    \hline
    \textbf{Variables} &  &  &  &  &  \\
    \hline
    NN PTA &  &  &  & 0.207$^{\ast\ast\ast}$ & 0.013 \\
     &  &  &  & (0.021) & (0.072) \\
    NN PTA Lag &  &  &  & 0.192$^{\ast\ast\ast}$ & 0.016 \\
     &  &  &  & (0.023) & (0.073) \\
    NN PTA + NN PTA Lag &  &  &  & 0.399$^{\ast\ast\ast}$ & 0.029 \\
     &  &  &  & (0.026) & (0.102) \\
    \hline
    NS PTA &  & 0.199$^{\ast\ast\ast}$ & -0.089 & 0.374$^{\ast\ast\ast}$ & 0.013 \\
     &  & (0.069) & (0.089) & (0.041) & (0.144) \\
    NS PTA Lag &  & 0.234 & -0.067 & 0.349$^{\ast\ast\ast}$ & 0.231$^{\ast\ast\ast}$ \\
     &  & (0.190) & (0.060) & (0.041) & (0.061) \\
    NS PTA + NS PTA Lag &  & 0.434$^{\ast\ast}$ & -0.156$^{\ast}$ & 0.723$^{\ast\ast\ast}$ & 0.244 \\
     &  & (0.200) & (0.090) & (0.046) & (0.156) \\
    \hline
    SS PTA & 0.578$^{\ast\ast\ast}$ & 0.476$^{\ast\ast\ast}$ & 0.153 & 0.530$^{\ast\ast\ast}$ & 0.004 \\
     & (0.154) & (0.139) & (0.117) & (0.107) & (0.121) \\
    SS PTA Lag & -0.278 & -0.023 & -0.208$^{\ast\ast\ast}$ & 0.575$^{\ast\ast\ast}$ & 0.204$^{\ast\ast\ast}$ \\
     & (0.300) & (0.133) & (0.063) & (0.119) & (0.073) \\
    SS PTA + SS PTA Lag & 0.301 & 0.453$^{\ast\ast\ast}$ & -0.055 & 1.105$^{\ast\ast\ast}$ & 0.208 \\
     & (0.295) & (0.112) & (0.130) & (0.092) & (0.128) \\
    \hline
    Exporter-Year FE & Yes & Yes & Yes & Yes & Yes \\
    Importer-Year FE & Yes & Yes & Yes & Yes & Yes \\
    Country-Pair FE & Yes & Yes & Yes & Yes & Yes \\
    R-Squared & 0.997 & 0.999 & 0.999 & 0.997 & 0.998 \\
    Observations & 5838 & 10997 & 25308 & 28168 & 73930 \\
    \hline
    \multicolumn{6}{l}{\footnotesize{Notes: Robust standard errors clustered at the country-pair level in parentheses.}} \\
    \multicolumn{6}{l}{\footnotesize{Significance levels are indicated as follows: $^{\ast}$p$<$0.1; $^{\ast\ast}$p$<$0.05; $^{\ast\ast\ast}$p$<$0.01.}} \\
    \end{tabular}
    \end{adjustbox}
\end{table}


%
\subsection{NS PTA Estimation Results by Region}%
\label{subsec:NSPTAEstimationResultsbyRegion}%
\subsubsection{NS PTA Model Results by Region}%
\label{ssubsec:NSPTAModelResultsbyRegion}%
\begin{table}[htbp]
    \centering
    \caption{Africa TA + TA Lag Coefficients by Type}
    \label{tab:africa_pta}
    \begin{adjustbox}{max width=\textwidth}
    \begin{tabular}{lccc}
    \hline
    \textbf{TA ID} & \textbf{NS TA+Lag} & \textbf{SS TA+Lag} & \textbf{NN TA+Lag} \\
    \hline
    \textbf{NS and SS (or only NS)} &  &  &  \\
    \hline
    \multicolumn{4}{c}{No agreements in this category} \\
    \hline
    \textbf{Only SS} &  &  &  \\
    \hline
    670 &  & 0.326 &  \\
     &  & (0.410) &  \\
    787 &  & 0.304 &  \\
     &  & (0.233) &  \\
    \hline
    \textbf{Agreements with NN and NS} &  &  &  \\
    \hline
    \multicolumn{4}{c}{No agreements in this category} \\
    \hline
    Exporter-Year FE & Yes \\
    Importer-Year FE & Yes \\
    Country-Pair FE & Yes \\
    R-Squared & 0.997 \\
    Observations & 5838 \\
    \hline
    \multicolumn{4}{l}{\footnotesize{Notes: Robust standard errors clustered at the country-pair level in parentheses.}} \\
    \multicolumn{4}{l}{\footnotesize{Significance levels are indicated as follows: $^{\ast}$p$<$0.1; $^{\ast\ast}$p$<$0.05; $^{\ast\ast\ast}$p$<$0.01.}} \\
    \end{tabular}
    \end{adjustbox}
\end{table}
%
\begin{table}[htbp]
    \centering
    \caption{Americas PTA + PTA Lag Coefficients by Type}
    \label{tab:americas_pta}
    \begin{adjustbox}{max width=\textwidth}
    \begin{tabular}{lccc}
    \hline
    \textbf{PTA ID} & \textbf{NS PTA + Lag} & \textbf{SS PTA + Lag} & \textbf{NN PTA + Lag} \\
    \hline
    \textbf{Agreements with NS and SS (or only NS)} &  &  &  \\
    \hline
    163 & 0.346$^{\ast\ast\ast}$ &  &  \\
    168 & 0.410$^{\ast\ast\ast}$ &  &  \\
    188 & -0.811$^{\ast\ast\ast}$ & 0.685$^{\ast\ast}$ &  \\
    218 & 0.879$^{\ast\ast\ast}$ &  &  \\
    645 & 0.117 &  &  \\
    \hline
    \textbf{Agreements with only SS} &  &  &  \\
    \hline
    141 &  & 0.265$^{\ast\ast\ast}$ &  \\
    201 &  & 0.545$^{\ast\ast}$ &  \\
    213 &  & 1.342$^{\ast\ast\ast}$ &  \\
    239 &  & 0.572$^{\ast\ast\ast}$ &  \\
    612 &  & 0.517$^{\ast\ast}$ &  \\
    616 &  & 0.488$^{\ast\ast\ast}$ &  \\
    716 &  & 0.732$^{\ast\ast}$ &  \\
    \hline
    \textbf{Agreements with NN and NS} &  &  &  \\
    \hline
    \multicolumn{4}{c}{No agreements in this category} \\
    \hline
    \multicolumn{4}{l}{\footnotesize{Notes: Robust standard errors clustered at the country-pair level in parentheses.}} \\
    \multicolumn{4}{l}{\footnotesize{Significance levels are indicated as follows: $^{\ast}$p$<$0.1; $^{\ast\ast}$p$<$0.05; $^{\ast\ast\ast}$p$<$0.01.}} \\
    \end{tabular}
    \end{adjustbox}
\end{table}
%
\begin{table}[htbp]
    \centering
    \caption{Asia PTA + PTA Lag Coefficients by Type}
    \label{tab:asia_pta}
    \begin{adjustbox}{max width=\textwidth}
    \begin{tabular}{lccc}
    \hline
    \textbf{PTA ID} & \textbf{NS PTA + Lag} & \textbf{SS PTA + Lag} & \textbf{NN PTA + Lag} \\
    \hline
    \textbf{Agreements with NS and SS (or only NS)} &  &  &  \\
    \hline
    1 &  & -0.732$^{\ast\ast}$ &  \\
    67 &  & 0.342$^{\ast\ast\ast}$ &  \\
    70 &  & 0.472$^{\ast\ast\ast}$ &  \\
    71 & -0.138 &  &  \\
    72 & 0.254 &  &  \\
    220 &  & -1.215$^{\ast\ast\ast}$ &  \\
    221 &  & -2.955$^{\ast\ast\ast}$ &  \\
    \hline
    \textbf{Agreements with only SS} &  &  &  \\
    \hline
    100 &  & 0.376$^{\ast\ast\ast}$ &  \\
    475 &  & 0.636$^{\ast\ast}$ &  \\
    640 &  & 0.183 &  \\
    675 &  & 1.360$^{\ast\ast}$ &  \\
    683 &  & 1.080$^{\ast\ast\ast}$ &  \\
    \hline
    \textbf{Agreements with NN and NS} &  &  &  \\
    \hline
    \multicolumn{4}{c}{No agreements in this category} \\
    \hline
    \multicolumn{4}{l}{\footnotesize{Notes: Robust standard errors clustered at the country-pair level in parentheses.}} \\
    \multicolumn{4}{l}{\footnotesize{Significance levels are indicated as follows: $^{\ast}$p$<$0.1; $^{\ast\ast}$p$<$0.05; $^{\ast\ast\ast}$p$<$0.01.}} \\
    \end{tabular}
    \end{adjustbox}
\end{table}
%
\begin{table}[htbp]
    \centering
    \caption{Europe PTA + PTA Lag Coefficients by Type}
    \label{tab:europe_pta}
    \begin{adjustbox}{max width=\textwidth}
    \begin{tabular}{lccc}
    \hline
    \textbf{PTA ID} & \textbf{NS PTA + Lag} & \textbf{SS PTA + Lag} & \textbf{NN PTA + Lag} \\
    \hline
    \textbf{Agreements with NS and SS (or only NS)} &  &  &  \\
    \hline
    8   & 0.663$^{\ast\ast\ast}$ & 0.783$^{\ast\ast\ast}$ &  \\
    9   & 0.581$^{\ast\ast}$ &  &  \\
    254 & 0.568$^{\ast\ast\ast}$ & 0.323 &  \\
    328 & 0.738$^{\ast\ast\ast}$ & 0.354 &  \\
    335 & 0.727$^{\ast\ast\ast}$ & 1.099$^{\ast\ast\ast}$ & 0.399$^{\ast\ast\ast}$ \\
    \hline
    \textbf{Agreements with only SS} &  &  &  \\
    \hline
    5   &  & 3.811$^{\ast\ast\ast}$ &  \\
    7   &  & 1.153$^{\ast\ast\ast}$ &  \\
    11  &  & 0.663$^{\ast\ast}$ &  \\
    13  &  & 2.303$^{\ast\ast\ast}$ &  \\
    128 &  & 2.773$^{\ast\ast\ast}$ &  \\
    129 &  & 0.556$^{\ast\ast\ast}$ &  \\
    132 &  & 2.241$^{\ast\ast\ast}$ &  \\
    135 &  & 0.696$^{\ast\ast\ast}$ &  \\
    150 &  & 0.444 &  \\
    153 &  & 0.817 &  \\
    192 &  & 1.199$^{\ast\ast\ast}$ &  \\
    621 &  & 0.614$^{\ast\ast\ast}$ &  \\
    394 & 0.747$^{\ast\ast\ast}$ &  &  \\
    \hline
    \textbf{Agreements with NN and NS} &  &  &  \\
    \hline
    335 & 0.727$^{\ast\ast\ast}$ & 1.099$^{\ast\ast\ast}$ & 0.399$^{\ast\ast\ast}$ \\
    6   & 0.411 &  &  \\
    132 & 0.738$^{\ast\ast\ast}$ & 2.241$^{\ast\ast\ast}$ &  \\
    \hline
    \multicolumn{4}{l}{\footnotesize{Notes: Robust standard errors clustered at the country-pair level in parentheses.}} \\
    \multicolumn{4}{l}{\footnotesize{Significance levels are indicated as follows: $^{\ast}$p$<$0.1; $^{\ast\ast}$p$<$0.05; $^{\ast\ast\ast}$p$<$0.01.}} \\
    \end{tabular}
    \end{adjustbox}
\end{table}
%
\begin{table}[htbp]
    \centering
    \caption{Intercontinental PTA + PTA Lag Coefficients by Type}
    \label{tab:intercontinental_pta}
    \begin{adjustbox}{max width=\textwidth}
    \begin{tabular}{lccc}
    \hline
    \textbf{PTA ID} & \textbf{NS PTA + Lag} & \textbf{SS PTA + Lag} & \textbf{NN PTA + Lag} \\
    \hline
    \textbf{Agreements with NS and SS (or only NS)} &  &  &  \\
    \hline
    17  & 0.800$^{\ast\ast\ast}$ & 1.055$^{\ast\ast\ast}$ &  \\
    75  & 1.366$^{\ast\ast\ast}$ &  &  \\
    96  & 0.271$^{\ast\ast\ast}$ &  &  \\
    202 & 0.660$^{\ast\ast\ast}$ & 0.612$^{\ast\ast}$ &  \\
    207 & 0.516$^{\ast\ast\ast}$ &  &  \\
    208 &  & 0.763$^{\ast\ast\ast}$ &  \\
    323 & -0.335$^{\ast\ast}$ & -0.360 &  \\
    330 & 0.286$^{\ast\ast\ast}$ & 0.662$^{\ast\ast\ast}$ &  \\
    512 & -0.458$^{\ast}$ &  &  \\
    518 & 0.627$^{\ast\ast\ast}$ &  &  \\
    624 &  & 0.384$^{\ast\ast}$ &  \\
    626 &  & 1.099$^{\ast\ast\ast}$ &  \\
    627 &  & 2.372$^{\ast\ast\ast}$ &  \\
    628 & 0.484$^{\ast\ast\ast}$ &  &  \\
    637 & 0.667$^{\ast\ast\ast}$ &  &  \\
    657 &  & 0.705$^{\ast\ast\ast}$ &  \\
    679 & 0.546$^{\ast}$ & -1.636$^{\ast\ast\ast}$ &  \\
    \hline
    \textbf{Agreements with only SS} &  &  &  \\
    \hline
    4   &  & 1.255$^{\ast\ast\ast}$ &  \\
    104 &  & -0.338$^{\ast\ast\ast}$ &  \\
    136 &  & 0.744$^{\ast\ast\ast}$ &  \\
    164 &  & 0.288$^{\ast\ast\ast}$ &  \\
    181 &  & 1.288$^{\ast\ast\ast}$ &  \\
    206 &  & 1.540$^{\ast\ast\ast}$ &  \\
    263 &  & 1.426$^{\ast\ast\ast}$ &  \\
    304 & 0.787$^{\ast\ast\ast}$ & 0.591$^{\ast\ast}$ &  \\
    466 &  & 0.710$^{\ast\ast\ast}$ &  \\
    490 &  & 0.843$^{\ast\ast\ast}$ &  \\
    521 & 0.102$^{\ast\ast}$ &  &  \\
    543 & 1.090 &  &  \\
    677 &  & -1.366$^{\ast\ast\ast}$ &  \\
    679 & -1.636$^{\ast\ast\ast}$ &  &  \\
    \hline
    \textbf{Agreements with NN and NS} &  &  &  \\
    \hline
    84  &  & -0.059 & -0.059 \\
    15  &  & 0.313 &  \\
    17  &  & 0.800$^{\ast\ast\ast}$ &  \\
    164 &  & 0.288$^{\ast\ast\ast}$ & 0.343$^{\ast\ast\ast}$ \\
    715 &  & 0.102$^{\ast}$ & 0.516$^{\ast\ast\ast}$ \\
    \hline
    \multicolumn{4}{l}{\footnotesize{Notes: Robust standard errors clustered at the country-pair level in parentheses.}} \\
    \multicolumn{4}{l}{\footnotesize{Significance levels are indicated as follows: $^{\ast}$p$<$0.1; $^{\ast\ast}$p$<$0.05; $^{\ast\ast\ast}$p$<$0.01.}} \\
    \end{tabular}
    \end{adjustbox}
\end{table}

%
\newpage%
\subsubsection{Additional Findings}%
\label{ssubsec:AdditionalFindings}%
Here you can include additional findings and discussions.

%
\section{Analysis and Discussion}%
\label{sec:AnalysisandDiscussion}%
This is the analysis and discussion

``Several themes emerge from this newly bourgeoning literature. First,
South--South trade and finance is now a significant economic and
political force for South countries as well as for the global economy.
There is a near consensus therefore that South--South economic relations
do matter and that they have the potential to have a significant
developmental impact. Moreover, this impact may be positive or negative,
that is, that it may help or hinder the long-term developmental goals of
exchanging parties. Second, much of South--South manufactures trade is
concentrated in high-technology-and-skill content, opening the door for
potential long-run dynamic gains from trade. However, these gains are
being increasingly concentrated within a small number of South
countries. The global South is, in fact, splitting into two groups,
which we refer to as the Emerging South and the Rest of South with very
different outcomes. While there is evidence for gains through
South--South trade, there is also evidence that the Emerging South is
rising at the expense of the Rest of South. Finally, the South--South
exchanges have expanded significantly to cover issues including
financial flows and technology transfer, among other topics. The overall
conclusion of this diverse literature is that while it does matter who
is exchanging what and with whom, South--South trade is not a panacea
for the development challenges in Southern countries. On the contrary,
South--South exchange themselves may become a potential threat for
development for some of the Southern countries.'' (Dahi \& Demir, 2017)

References

Dahi, O. S., \& Demir, F. (2017). South-South and North-South Economic
Exchanges: Does It Matter Who Is Exchanging What and with Whom?
\emph{Journal of Economic Surveys}, \emph{31}(5), 1449--1486.
https://doi.org/10.1111/joes.12225

%
\section{Conclusion}%
\label{sec:Conclusion}%
This is the conclusion

%
\newpage%
%TC:ignore%
\section{References}%
\label{sec:References}%
\printbibliography

%
%TC:endignore%
\newpage%
%TC:ignore%
\section{Appendix}%
\label{sec:Appendix}%
\subsection{Subsection in Appendix}%
\label{subsec:SubsectioninAppendix}%
Content in the appendix should not be counted in the word count.

%
%TC:endignore%
\end{document}